\section{From regular expressions to \eNFA{s}}

We left behind the regular expressions when we informally introduced
the transition diagrams for token recognition. Now let us show that
regular expressions when used in lexers to specify tokens can be
converted to \eNFA{s}, and therefore to DFAs. This proves that regular
languages are recognisable languages. Actually, it is possible to
prove that any \eNFA can be converted to a regular expression denoting
the same language, but we will not do so. Therefore, keep in mind that
the regular languages are the same as the recognisable languages. In
other words, the choice of using a regular expression or a finite
automaton is only a matter of convenience.

The construction we present here to build an \eNFA from a regular
expression is called \emph{Thompson's construction}. Let us first
associate an \eNFA to the basic regular expressions.
\begin{itemize}

  \item For the expression \(\epsilon\), construct the following NFA,
    where~\(i\) and~\(f\) are new states:
  \begin{center}
    \includegraphics[bb=48 710 135 730]{thompson_epsilon}
  \end{center}

  \item For \(a \in \Sigma\), construct the following NFA, where~\(i\)
    and~\(f\) are new states:
  \begin{center}
    \includegraphics[bb=48 710 135 730]{thompson_symbol}
  \end{center}

\end{itemize}
Now let us associate NFAs to complex regular expressions. In the
following, let us assume that~\(N(s)\) and~\(N(t)\) are the NFAs for
regular expressions \(s\)~and~\(t\).
\begin{itemize}

  \item For the regular expression \(st\), construct the following NFA
    \(N(st)\), where no new state is created:
\begin{center}
\includegraphics[bb=65 660 295 714]{thompson_conc}
\end{center}
  The final state of \(N(s)\) becomes a normal state, as well as the
  initial state of \(N(t)\). This way only remains a unique initial
  state~\(i\) and a unique final state~\(f\).

  \item For the regular expression \(s\) \disj \(t\), construct the
  following NFA \(N(s \, \text{\disj} \, t)\)
\begin{center} 
\includegraphics[bb=65 590 272 715,scale=0.9]{thompson_disj}
\end{center}
where \(i\) and \(f\) are new states. Initial and final
states of \(N(s)\) and \(N(t)\) become normal.

  \item For the regular expression \(s\)\kleene, construct the following
    NFA \(N(s\text{\kleene})\), where~\(i\) and~\(f\) are new
    states:
\begin{center}
\includegraphics[bb=50 620 255 718]{thompson_kleene}
\end{center}
Note that we added two \(\epsilon\) transitions and that the initial
and final states of \(N(s)\) become normal states.

\end{itemize}
How do we apply these simple rules when we have a complex regular
expression, having many level of nested parentheses and other
constructs? Actually, the abstract syntax tree of the regular
expression directs the application of the rules. If the syntax tree
has the shape shown in \fig~\vref{fig:re_ast_conc},
\begin{figure}[b]
\centering
\subfloat[\(s \cdot t\)\label{fig:re_ast_conc}]{
  \includegraphics{re_ast_conc}
}
\qquad
\subfloat[\(s \,\text{\disj}\, t\)\label{fig:re_ast_disj}]{
\includegraphics{re_ast_disj}
}
\qquad
\subfloat[\(s\text{\kleene}\)\label{fig:re_ast_kleene}]{
\includegraphics{re_ast_kleene}
}
\caption{Three tree patterns for three regular expressions}
\end{figure}
then we construct first \(N(s)\), \(N(t)\) and finally \(N(s \cdot
t)\). If the syntax tree has the shape found in
\fig~\vref{fig:re_ast_disj}, then we construct first \(N(s)\),
\(N(t)\) and finally \(N (s \, \text{\disj} \, t)\). If the syntax
tree has the shape shown in \fig~\vref{fig:re_ast_kleene}, then we
construct first \(N(s)\) and finally \(N(s\text{\kleene})\). These
pattern matchings are applied first at the root of the abstract syntax
tree of the regular expression.
