\section{Equivalence of DFAs and NFAs}

NFAs are easier to build than DFAs because one does not have to worry,
for any state, of having out-going edges carrying a unique label. The
surprising fact is that NFAs and DFAs actually have the same
expressiveness, that is, all that can be defined by means of a NFA can
also be defined with a DFA (the converse is trivial since a DFA is
already a NFA). More precisely, there is a procedure, called \emph{the
  subset construction}, which converts any NFA to a DFA.

Consider that, in a NFA, from a state~\(q\) with several out-going
edges with the same label~\(a\), the transition function~\(\delta_N\)
leads, in general, to several states. The idea of the \emph{subset
  construction} is to create a new automaton where these edges are
merged. So we create a state~\(p\) which corresponds to the set of
states \(\delta_N (q,a)\) in the NFA. Accordingly, we create a
state~\(r\) which corresponds to the set~\(\{q\}\) in the NFA. We
create an edge labelled~\(a\) between~\(r\) and~\(p\). The important
point is that \emph{this edge is unique}. This is the first step to
create a DFA from a NFA.

Graphically, instead of the non-determinism of
\fig~\vref{fig:nfa_edges},
\begin{figure}
\centering
\subfloat[Non-determinism\label{fig:nfa_edges}]{
\includegraphics[bb=62 660 222 730]{nfa_edges}
}
\qquad
\subfloat[Determinism\label{fig:nfa_subset}]{
\includegraphics[bb=60 712 144 730]{nfa_subset}
}
\caption{From non-determinism to determinism}
\end{figure}
where we have \(\delta_N (q, a) = \{p_0, p_1, \dots, p_n\}\), we get
the determinism of \fig~\vref{fig:nfa_subset}.

Let us present the complete algorithm for the subset construction.

Let us start from a NFA \(\mathcal{N} = (Q_N, \Sigma, \delta_N, q_0,
F_N)\). The goal is to construct a DFA \(\mathcal{D} = (Q_D, \Sigma,
\delta_D, \{q_0\}, F_D)\) such that \(L(\mathcal{D}) =
L(\mathcal{N})\). Notice that the input alphabet of the two automata
are the same and the initial state of~\(\mathcal{D}\) if the set
containing only the initial state of~\(\mathcal{N}\).

The other components of~\(\mathcal{D}\) are constructed as
follows. First, \(Q_D\)~is the set of subsets of~\(Q_N\); that is,
\(Q_D\)~is the \emph{power set} of~\(Q_N\). Thus, if~\(Q_D\) has~\(n\)
states, \(Q_D\)~has~\(2^n\) states. Fortunately, often not all these
states are accessible from the initial state of~\(Q_D\), so these
inaccessible states can be discarded.

\label{state_explosion}
Why is~\(2^n\) the number of subsets of a finite set of
cardinal~\(n\)?

Let us order the~\(n\) elements and represent each subset by an
\(n\)-bit string where bit \(i\) corresponds to the \(i\)th element:
it is~\(1\) if the \(i\)th element is present in the subset and~\(0\)
if not. This way, we counted all the subsets and not more (a bit
cannot always be \(0\) since all elements are used to form subsets and
cannot always be \(1\) if there is more than one element). There
are~\(2\) possibilities, \(0\)~or~\(1\), for the first bit;
\(2\)~possibilities for the second bit etc. Since the choices are
independent, we multiply all of them: \(\underbrace{2 \times 2 \times
  \dots \times 2}_{n \; \text{times}} = 2^n\), yielding the number of
subsets of an \(n\)-element set.

Resuming the definition of DFA~\(\mathcal{D}\), the remaining
components are defined as follows.
\begin{itemize}

  \item \(F_D\)~is the set of subsets~\(S\) of~\(Q_N\) such that \(S
    \cap F_N \neq \varnothing\). That is, \(F_D\)~is all sets of
    \(N\)'s states that include at least one final state
    of~\(\mathcal{N}\).

  \item for each set \(S \subseteq Q_N\) and for each input \(a \in
    \Sigma\),
  \begin{equation*}
    \delta_D(S, a) = \bigcup_{q \in S}{\delta_N (q, a)}.
  \end{equation*}
  In other words, to compute~\(\delta_D (S, a)\), we look at all the
  states~\(q\) in~\(S\), see what states of~\(\mathcal{N}\) are
  reached from~\(q\) on input~\(a\) and take the union of all those
  states to make the next state of~\(\mathcal{D}\).

\end{itemize}
Let us reconsider the NFA given by its transition table in
\fig~\vref{fig:initial_nfa_table}
\begin{figure}
\centering
\(\begin{array}{r@{}l||c|c}
    \multicolumn{2}{c||}{\text{NFA} \; \mathcal{N}} & 0 & 1\\
    \hhline{==::==}
    \rightarrow & q_0 & \{q_0, q_1\} & \{q_0\}\\
                & q_1 & \varnothing  & \{q_2\}\\
    \#          & q_2 & \varnothing  & \varnothing
\end{array}\)
\caption{Initial NFA table\label{fig:initial_nfa_table}}
\end{figure}
and let us create an equivalent DFA. Firstly, we form all the subsets
of the sets of the NFA and put them in the first column in
\fig~\vref{fig:first_stage}.
\begin{figure}
\centering
\subfloat[First stage\label{fig:first_stage}]{
\(\begin{array}{r@{}l||c|c}
    \multicolumn{2}{c||}{\text{DFA} \; \mathcal{D}} & 0 & 1\\
    \hhline{==::==}
    & \varnothing & &\\
    & \{q_0\}           & &\\
    & \{q_1\}           & &\\
    & \{q_2\}           & &\\
    & \{q_0, q_1\}      & &\\
    & \{q_0, q_2\}      & &\\
    & \{q_1, q_2\}      & &\\
    & \{q_0, q_1, q_2\} & &
\end{array}\)
}
\qquad
\subfloat[Second stage\label{fig:second_stage}]{
\(\begin{array}{r@{}l||c|c}
    \multicolumn{2}{c||}{\text{DFA} \; \mathcal{D}} & 0 & 1\\
    \hhline{==::==}
    & \varnothing & &\\
    \rightarrow & \{q_0\}  & &\\
       & \{q_1\}           & &\\
    \# & \{q_2\}           & &\\
       & \{q_0, q_1\}      & &\\
    \# & \{q_0, q_2\}      & &\\
    \# & \{q_1, q_2\}      & &\\
    \# & \{q_0, q_1, q_2\} & &
\end{array}\)
}
\caption{First two stages of the subset construction}
\end{figure}

Secondly, we annotate in this first column the states
with~\(\rightarrow\) if and only if they contain the initial state of
the NFA, here~\(q_0\), and we add a~\(\#\) if and only if the states
contain at least a final state of the NFA, here~\(q_2\). See
\fig~\vref{fig:second_stage}.

Thirdly, we form the subsets as follows:
\begin{equation*}
\small
\begin{array}{@{}r@{}l@{\,}||@{\,}c@{\,}|@{\,}c@{}}
\multicolumn{2}{c||}{\text{DFA} \; \mathcal{D}} & 0 & 1\\
\hhline{==::==}
            & \varnothing & \varnothing & \varnothing\\
\rightarrow & \{q_0\} & \delta_N(q_0, 0) & \delta_N(q_0, 1)\\
            & \{q_1\} & \delta_N(q_1, 0) & \delta_N(q_1, 1)\\
         \# & \{q_2\} & \delta_N(q_2, 0) & \delta_N(q_2, 1)\\
            & \{q_0, q_1\} & \delta_N(q_0,0) \cup \delta_N(q_1,0) 
                           & \delta_N(q_0,1) \cup \delta_N(q_1,1)\\
         \# & \{q_0, q_2\} & \delta_N(q_0,0) \cup \delta_N(q_2,0) 
                           & \delta_N(q_0,1) \cup \delta_N(q_2,1)\\
         \# & \{q_1, q_2\} & \delta_N(q_1,0) \cup \delta_N(q_2,0) 
                           & \delta_N(q_1,1) \cup \delta_N(q_2,1)\\
         \# & \{q_0, q_1, q_2\} 
            & \delta_N(q_0,0) \cup \delta_N(q_1,0) \cup \delta_N(q_2,0)
            & \delta_N(q_0,1) \cup \delta_N(q_1,1) \cup \delta_N(q_2,1)
\end{array}
\end{equation*}
Finally, we compute those subsets and obtain the table in
\fig~\ref{fig:first_dfa_table}.
\begin{figure}[t]
\centering
\(\begin{array}{r@{}l||c|c}
\multicolumn{2}{c||}{\text{DFA} \; \mathcal{D}} & 0 & 1\\
\hhline{==::==}
            & \varnothing & \varnothing & \varnothing\\
\rightarrow & \{q_0\} & \{q_0, q_1\} & \{q_0\}\\
            & \{q_1\} & \varnothing & \{q_2\}\\
         \# & \{q_2\} & \varnothing & \varnothing\\
            & \{q_0, q_1\} & \{q_0,q_1\} & \{q_0,q_2\}\\
         \# & \{q_0, q_2\} & \{q_0,q_1\} & \{q_0\}\\
         \# & \{q_1, q_2\} & \varnothing & \{q_2\}\\
         \# & \{q_0, q_1, q_2\} & \{q_0, q_1\} & \{q_0,q_2\}
\end{array}\)
\caption{First DFA obtained\label{fig:first_dfa_table}}
\end{figure}
The transition diagram of the DFA~\(\mathcal{D}\) is showed in
\fig~\vref{fig:dfa_from_nfa}
\begin{figure}[b]
\centering
\includegraphics[bb=42 628 291 758]{dfa_from_nfa}
\caption{Transition diagram of \fig~\vref{fig:first_dfa_table}
\label{fig:dfa_from_nfa}}
\end{figure}
where states with out-going edges which have no end are final
states. If we look carefully at the transition diagram, we see that
the DFA is actually made of two disconnected sub\hyp{}automata. In
particular, since we have only one initial state, this means that one
part is not accessible, therefore its states are never used to
recognise or reject an input word, and we can remove this part, as
shown in \fig~\ref{fig:dfa_from_nfa_simple1}.
\begin{figure}
\centering
\subfloat[Simplification of \fig~\ref{fig:dfa_from_nfa}
\label{fig:dfa_from_nfa_simple1}]{
\includegraphics[bb=43 628 162 758]{dfa_from_nfa_simple1}
}
\qquad
\subfloat[State renamings of \fig~\ref{fig:dfa_from_nfa_simple1}
\label{fig:dfa_from_nfa_simple}]{
\includegraphics[bb=48 625 150 755]{dfa_from_nfa_simple}
}
\caption{Simplifications}
\end{figure}
It is important to understand that the states of the DFA are subsets
of the NFA states. This is due to the construction and, when finished,
it is possible to hide this by renaming the states. For example, we
can rename the states of the previous DFA in the following manner:
\(\{q_0\}\) into~\(A\), \(\{q_0, q_1\}\) in~\(B\) and \(\{q_0, q_2\}\)
in~\(C\). So the transition table changes:
\begin{equation*}
  \begin{array}{r@{}l||c|c}
    \multicolumn{2}{c||}{\text{DFA} \; \mathcal{D}} & 0 & 1\\
    \hhline{==::==}
    \rightarrow & \{q_0\}     & \{q_0, q_1\} & \{q_0\}\\
                & \{q_0,q_1\} & \{q_0,q_1\}  & \{q_0,q_2\}\\
             \# & \{q_0,q_2\} & \{q_0,q_1\}  & \{q_0\}
  \end{array}
\qquad
  \begin{array}{r@{}l||c|c}
    \multicolumn{2}{c||}{\text{DFA} \; \mathcal{D}} & 0 & 1\\
    \hhline{==::==}
    \rightarrow & A & B & A\\
                & B & B & C\\
             \# & C & B & A
  \end{array}
\end{equation*}
So, finally, the DFA is simply as in \fig~\vref{fig:dfa_from_nfa_simple}.

\subsection*{Optimisation}

Even if in the worst case the resulting DFA has an exponential number
of states of the corresponding NFA, it is in practice often possible
to avoid the construction of inaccessible states.
\begin{itemize*}

  \item The singleton containing the initial state (in our example,
  \(\{q_0\}\))~is accessible;

  \item let us assume we have a set~\(S\) of accessible states; then
    for each input symbol~\(a\), we compute \(\delta_D(S,a)\): this
    new set is also accessible;

  \item let us repeat the last step, starting with~\(\{q_0\}\), until
    no new (accessible) sets are found.

\end{itemize*}
Let us reconsider the NFA given by the transition table
\begin{equation*}
  \begin{array}{r@{}l||c|c}
    \multicolumn{2}{c||}{\text{NFA} \; \mathcal{N}} & 0 & 1\\
    \hhline{==::==}
    \rightarrow & q_0 & \{q_0, q_1\} & \{q_0\}\\
                & q_1 & \varnothing  & \{q_2\}\\
    \#          & q_2 & \varnothing  & \varnothing
  \end{array}
\end{equation*}
Initially, the sole subset of accessible states is \(\{q_0\}\):
\begin{equation*}
  \begin{array}{r@{}l||c|c}
    \multicolumn{2}{c||}{\text{DFA} \; \mathcal{D}} & 0 & 1\\
    \hhline{==::==}
    \rightarrow & \{q_0\} & \delta_N(q_0,0) & \delta_N(q_0,1)
  \end{array}
\quad\text{that is}\quad
  \begin{array}{r@{}l||c|c}
    \multicolumn{2}{c||}{\text{DFA} \; \mathcal{D}} & 0 & 1\\
    \hhline{==::==}
    \rightarrow & \{q_0\} & \{q_0,q_1\} & \{q_0\}
  \end{array}
\end{equation*}
Therefore \(\{q_0,q_1\}\) and~\(\{q_0\}\) are accessible sets,
but~\(\{q_0\}\) is not a new set, so we only add to the table entries
\(\{q_0, q_1\}\) and compute the transitions from it:
\begin{equation*}
  \begin{array}{r@{}l||c|c}
    \multicolumn{2}{c||}{\text{DFA} \; \mathcal{D}} & 0 & 1\\
    \hhline{==::==}
    \rightarrow & \{q_0\}      & \{q_0, q_1\} & \{q_0\}\\
                & \{q_0, q_1\} & \{q_0, q_1\} & \{q_0, q_2\}
  \end{array}
\end{equation*}
This step uncovered a new set of accessible states, \(\{q_0, q_2\}\),
which we add to the table and repeat the procedure, and mark it as
final state since \(q_2 \in \{q_0, q_2\}\):
\begin{equation*}
  \begin{array}{r@{}l||c|c}
    \multicolumn{2}{c||}{\text{DFA} \; \mathcal{D}} & 0 & 1\\
    \hhline{==::==}
    \rightarrow & \{q_0\}      & \{q_0, q_1\} & \{q_0\}\\
                & \{q_0, q_1\} & \{q_0, q_1\} & \{q_0, q_2\}\\
             \# & \{q_0, q_2\} & \{q_0, q_1\} & \{q_0\}
  \end{array}
\end{equation*}
We are done since there are no more new accessible sets.

\subsection*{Tries}

Lexical analysis tries to recognise a prefix of the input character
stream (in other words, the first lexeme of the given
program). Consider the C~keywords \term{const} and \term{continue}:
\begin{center}
\includegraphics[bb=48 675 353 730]{nfa_kwd}
\end{center}
This example shows that a NFA is much more comfortable than a DFA for
specifying tokens for lexical analysis: we design separately the
automata for each token and then merge their initial states into one,
leading to one, possibly large NFA. It is possible to apply the subset
construction to this NFA.

After forming the corresponding NFA as in the previous example, it is
actually easy to construct an equivalent DFA by sharing their
prefixes, hence obtaining a tree-like automaton called \emph{trie}
(pronounced as the word `try'):
\begin{center}
\includegraphics[bb=48 675 353 730]{trie_kwd}
\label{trie_kwd}
\end{center}
Note that this construction only works for a list of constant words,
like keywords.

This technique can easily be generalised for searching constant
strings (like keywords) in a text, that is, not only as a prefix of a
text, but \emph{at any position}. It suffices to add a loop on the
initial state for each possible input symbol. If we note~\(\Sigma\)
the language alphabet, we get
\begin{center}
\includegraphics[bb=48 675 355 738]{trie_kwd_search}
\end{center}
It is possible to apply the subset construction to this NFA or to use
it directly for searching the two keywords at any position in a
text. In case of direct use, the difference between this NFA and the
trie one page~\pageref{trie_kwd} is that there is no need here to
`restart' by hand the recognition process once a keyword has been
recognised: we just continue. This works because of the loop on the
initial state, which always allows a new start. (The reader may try
for instance the input \texttt{constantcontinue}.)

The subset construction can lead, in the worst case, to a number of
states which is the total number of state subsets of the NFA. In other
words, if the NFA has~\(n\) states, the equivalent DFA by subset
construction can have~\(2^n\) states (see
page~\pageref{state_explosion} for the count of all the subsets of a
finite set). For instance, consider the following NFA, which
recognises all binary strings which have~\(1\) at the \(n\)th
position from the end:
\begin{center}
\includegraphics[bb=48 711 305 758]{subset_bad_case}
\end{center}
The language recognised by this NFA is \(\Sigma^{*} 1 \Sigma^{n-1}\),
where \(\Sigma=\{0,1\}\), that is: all words of length greater or
equal to~\(n\) are accepted as long as the \(n\)th bit from the
\emph{end} is~\(1\). Therefore, in any equivalent DFA, all the
prefixes of length~\(n\) should not lead to a stuck state, because the
automaton must wait until the \emph{end} of the word to accept or
reject it. If the states reached by these prefixes are all different,
then there are at least \(2^n\)~states in the DFA. Equivalently (by
contraposition), if there are less than \(2^n\)~states, then some
states can be reached by several strings of length~\(n\):
\begin{center}
\includegraphics[bb=48 643 235 730]{dfa_bad_case}
\end{center}
where words \(x1w\) and \(x'0w\) have length~\(n\). Let us define the
DFA as follows: \(\mathcal{D} = (Q_D, \Sigma, \delta_D, q_D, F_D)\),
where \(q_D=\{q_0\}\). The extended transition function is noted
\(\hat{\delta}_D\) as usual. The situation of the previous picture can
be formally expressed as
\begin{gather}
\hat{\delta}_D (q_D, x1) = \hat{\delta}_D (q_D, x'0) = q, \label{ext:1}\\
\lvert{x1w}\lvert = \lvert{x'0w}\lvert = n,\notag
\end{gather}
where \(\lvert{u}\lvert\) is the length of~\(u\). Let~\(y\) be a any
string of~0 and~1 such that \(\lvert{wy}\lvert = n - 1\). Then
\(\hat{\delta}_D(q_D, x1wy) \in F_D\) since there is a~1 at the
\(n\)th position from the end:
\begin{center}
\includegraphics[bb=48 643 287 730]{dfa_bad_case_complete}
\end{center}
Also, \(\hat{\delta}_D(q_D,x'0wy) \not\in F_D\) because there is a 0
at the \(n\)th position from the end. On the other hand,
equation~\eqref{ext:1} implies
\begin{equation*}
\hat{\delta}_D (q_D, x1wy) = \hat{\delta}_D (q_D, x'0wy) = p.
\end{equation*}
So we stumble upon a contradiction because a state (\(p\)) must be
either final or not final, it cannot be both. As a consequence, we
must reject our initial assumption: there are at least \(2^n\)~states
in the equivalent DFA. This is a very bad case, even if it is not the
worst case (\(2^{n+1}\) states).
