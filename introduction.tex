\chapter{Introduction}
\setcounter{page}{1}

This is an overview of the topics developed in the rest of the book.

\section{Rewrite systems}

\paragraph{String-rewriting}

Let us assume that we own a string of white and black beads, like
\(\circ \bullet \bullet \bullet \circ \circ \bullet\), and the
game \citep{VanLeeuwen_1990a,Dershowitz_1993} consists in removing two
adjacent beads and replace them with only one according to some rules,
for example
\begin{equation*}
\bullet \; \circ   \xrightarrow{\smash{\alpha}} \bullet\qquad\qquad
\circ   \; \bullet \xrightarrow{\smash{\beta}} \bullet\qquad\qquad
\bullet \; \bullet \xrightarrow{\smash{\gamma}} \circ
\end{equation*}
The rules~\(\alpha\), \(\beta\)~and~\(\gamma\) make up a simple
\emph{string\hyp{}rewriting system}\index{rewrite system}. Rules
\(\alpha\)~and~\(\beta\) can be conceived as `A black bead absorbs the
white bead next to it.' The goal of this game is to end up with as few
beads as possible, so our example may lead to the \emph{rewrites}
\begin{equation*}
\circ \bullet \bullet \, \fbox{\(\bullet\; \circ\)} \circ \bullet
\xrightarrow{\smash{\alpha}} \circ \bullet \bullet \,
\fbox{\(\bullet\; \circ\)} \; \bullet \xrightarrow{\smash{\alpha}}
\fbox{\(\circ \; \bullet\)} \; \bullet \bullet \,\bullet
\xrightarrow{\smash{\beta}} \bullet \bullet \, \fbox{\(\bullet \;
  \bullet\)} \xrightarrow{\smash{\gamma}} \bullet \, \fbox{\(\bullet
  \; \circ\)} \xrightarrow{\smash{\alpha}} \fbox{\(\bullet \;
  \bullet\)} \xrightarrow{\smash{\gamma}} \circ
\end{equation*}
where the part of the string to be rewritten next is framed.

Other compositions of the rules lead to the same result~\(\circ\) as
well. Some others bring all\hyp{}white strings, the simplest being
\(\circ \, \circ\). Some others lead to~\(\bullet\). Strings that can
not be further rewritten, or \emph{reduced}, are called \emph{normal
  forms}\index{rewrite system!normal form}. These observations induce
us to wonder whether all strings have a normal form; if so, if it is
unique and, furthermore, if it is either all\hyp{}white or
black\hyp{}only.

First, let us note that the system is  
\emph{terminating}\index{termination}, that is, there is no infinite
chain of rewrites, because the number of beads strictly decreases in
all the rules, although this is not a necessary condition in general,
for instance, \(\circ \; \bullet \xrightarrow{\smash{\beta}} \bullet
\circ \circ\) would preserve termination because the composition
\(\beta\alpha\alpha\) would be equivalent to the original
rule~\(\beta\). In particular, this means that any string has a normal
form. Furthermore, notice how the parity of the number of black beads
is invariant through each rule and how there is no rewrite rule for
two adjacent white beads. Therefore, if there are \(2p\) initial black
beads, then composing rules \(\alpha\)~and~\(\beta\) lead to an
all\hyp{}black string, like \(\bullet \bullet \bullet \, \bullet\)
above, which can be reduced by applying rule~\(\gamma\) to contiguous
pairs of beads into an all\hyp{}white string made of
\(p\)~beads. Otherwise, the same all\hyp{}black string can be reduced
by applying alternatively \(\gamma\)~and~\(\beta\) on the left end or
\(\gamma\)~and~\(\alpha\) on the right end,
yielding~\(\circ\). Similarly, if there is an initial odd number of
black beads, we always end up with one black bead. It suffices to
consider the rewrites \(\circ \circ \xleftarrow{\smash{\gamma}}
\bullet \bullet \circ \xrightarrow{\smash{\alpha}} \bullet \bullet
\xrightarrow{\smash{\gamma}} \circ\) to see that normal forms are not
unique. Systems where normal forms are unique are said
\emph{confluent}\index{rewrite system!confluence}.

If we add the rule \(\circ\; \circ \xrightarrow{\smash{\delta}}
\circ\), the result of the game is always one bead, whose colour
depends on the original parity of the black beads as before, and any
strategy is successful. To see why, let us consider first that two
non\hyp{}overlapping parts of a string can be rewritten in parallel,
so they are not of concern. The interesting cases occur when two
applications of rules (maybe of the same rule) lead to different
strings because they overlap. For instance, \(\circ \, \circ
\xleftarrow{\smash{\gamma}} \bullet \bullet \circ
\xrightarrow{\smash{\alpha}} \bullet \, \bullet\). The important point
is that \(\circ \, \circ\) and \(\bullet \, \bullet\) can be rewritten
into~\(\circ\) at the next step by \(\delta\)~and~\(\gamma\),
respectively. In general, what matters is that all pairs of strings
resulting from the application of overlapping rules, called
\emph{critical pairs}\index{rewrite system!critical pair}, can be
rewritten to the same string, to wit, they are \emph{joinable}. In our
example, all interactions occur on substrings made of three beads, so
we must examine in \fig~\vref{fig:bullets}
\begin{figure}[b]
\centering
\includegraphics{bullets}
\caption{The critical pairs are all joinable\label{fig:bullets}}
\end{figure}
eight cases, which we can order as if counting in binary from
\(0\)~to~\(7\), (\(\circ\))~being interpreted as~\(0\) and
(\(\bullet\))~as~\(1\). In all the cases, the divergences are joinable
in one step at most. In general, it is not necessary for critical
pairs to be joinable in one rewrite just after the divergence, but,
more generally, to be joinable. This property is called \emph{local
  confluence}\index{rewrite system!confluence!local $\sim$}. Together
with termination it implies that \emph{every} string has exactly one
normal form (a strong property entailing confluence).

The system we defined is \emph{ground}\index{rewrite system!ground
  $\sim$}, that is, it involves no variables. These allow a finite
system to denote an infinite number of ground rules or, simply, to
reduce the size of rewrite systems, for instance, the previous example
is equivalent to
\begin{equation*}
\bullet \; \circ \xrightarrow{\smash{\alpha}} \bullet
\qquad\qquad
\circ \; x \xrightarrow{\smash{\beta+\delta}} x
\qquad\qquad
\bullet \; \bullet \xrightarrow{\smash{\gamma}} \circ
\end{equation*}
If we accept multiple occurrences of a variable on the left\hyp{}hand
side of a rule, a so\hyp{}called \emph{non left\hyp{}linear
  rule}\index{rewrite system!linear $\sim$}, we can further decrease
the size of the system:
\begin{equation*}
x  \; x \xrightarrow{\smash{\gamma+\delta}} \circ\qquad\qquad
x  \; y \xrightarrow{\smash{\alpha+\beta}} \bullet
\end{equation*}
Notice that there is now an implicit order over the rules: the rule
\(\gamma+\delta\) must be examined first for a \emph{match} with a
part of the current string, because it is included in the second (set
\(x=y\) in \(\alpha+\beta\)). In general, the order in which the rules
are written on the page is their logical order. Moreover, let us
remark that the system does not specify that \(x\)~must either
be~\(\circ\) or~\(\bullet\). Generally speaking, this means that the
\emph{type} of the variables has to be defined elsewhere or inferred
from the variable uses.

\paragraph{Term-rewriting}

Up to now, only string\hyp{}rewriting systems have been played
with. More general are the \emph{term\hyp{}rewriting
  systems} \citep{BaaderNipkow_1998}\index{rewrite system}, where a
\emph{term}\index{term} is a mathematical object possibly featuring
tuples, integers and variables. Let us consider the following totally
ordered system
\begin{equation}
(0,m) \rightarrow m;\qquad\qquad
(n,m) \rightarrow (n-1,n \cdot m);\qquad\qquad
n     \rightarrow (n,1).
\label{eq:fact_tf}
\end{equation}
where rules are separated by a semi\hyp{}colon and the last one is
ended by a period. Arithmetic operators \((-)\)~and~\((\cdot)\) are
defined out of the system and \(m\)~and~\(n\) are variables denoting
natural numbers. Would the rules not be ordered as they are actually
laid out, the second rule would match any pair. Instead, it can be
assumed that \(n \neq 0\) when matching with it. We can easily see
that all the compositions of rewrites starting with a natural
number~\(n\) end with the factorial\index{factorial} of~\(n\):
\begin{equation*}
n \rightarrow (n,1) \rightarrow \dots \rightarrow (0,n!) \rightarrow
n!, \quad \text{for \(n \in \mathbb{N}\)}.
\end{equation*}
Let us note \((\xrightarrow{\smash{n}})\) the composition of
\((\rightarrow)\) repeated \(n-1\)~times:
\begin{equation*}
  (\xrightarrow{\smash{1}})   := (\rightarrow);\qquad
  (\xrightarrow{\smash{n+1}}) :=
     (\rightarrow) \circ (\xrightarrow{\smash{n}}),
\quad \text{with \(n > 0\)}.
\end{equation*}
The \emph{transitive closure}\index{transitive
  closure} \label{transitive_closure} of \((\rightarrow)\) is defined
as \((\twoheadrightarrow) := \bigcup_{i >
  0}{(\xrightarrow{\smash{i}})}\). In the present case, the
factorial\index{factorial} coincides with the transitive closure of
\((\rightarrow)\), namely, \(n \twoheadrightarrow n!\). Let
\((\xrightarrow{\smash{*}})\) be the reflexive\hyp{}transitive closure
of \((\rightarrow)\), that is, \((\xrightarrow{\smash{*}}) := (=) \cup
(\twoheadrightarrow)\). 

A confluent system defines a \emph{function} and it is then convenient
to name it; for example, \(\fun{c}(1, \fun{d}(n))\)~is a term
constructed with \emph{function names} \fun{c}~and~\fun{d}, as well as
variable~\(n\). A tuple tagged with a function name, like
\(\fun{f}(x,y)\), is called a \emph{function call}. The components of
the tuples are then called \emph{arguments}, for example
\(\fun{d}(n)\) is the second argument of the call \(\fun{c}(1,
\fun{d}(n))\). It is possible for a function call to hold no
arguments, like \(\fun{d}()\). We restrict the left\hyp{}hand sides of
rules to be function calls.

\section{Trees for depicting terms}\index{term|see{tree}} 
\label{def:tree}

% Wrapping figure better declared before a paragraph
%
\begin{wrapfigure}[9]{r}[0pt]{0pt}
% [9] vertical lines
% {r} mandatory right placement
% [0pt] of margin overhang
\centering
\includegraphics[bb=71 645 192 710]{tree_for_term}% [...720]
\caption{Shape of a tree\label{fig:tree_for_term}}
\end{wrapfigure}
The topological understanding of a function call or a tuple is the
finite \emph{tree}\index{tree}. A tree is a hierarchical layout of
information and \fig~\vref{fig:tree_for_term} shows the shape of
one. The disks are called \emph{nodes}\index{tree!node} and the
segments which connect two nodes are called
\emph{edges}\index{tree!edge}. The topmost node (with a diameter) is
called the \emph{root}\index{tree!node!root} and the bottommost ones
(\(\bullet\)) are called the \emph{leaves}\index{tree!node!leaf}. All
nodes except the leaves are seen to downwardly connect to some other
nodes, called \emph{children}\index{tree!node!child}. Upwardly, each
node but the root is connected to another node, called its
\emph{parent}\index{tree!node!parent}. Depending on the context, a
node can also denote the whole tree of which it is the root. Any node
except the root is the root of a \emph{proper
  subtree}\index{tree!subtree!proper $\sim$}. A tree is its own
subtree. The children of a node~\(x\) are the roots of \emph{immediate
  subtrees}\index{tree!subtree!immediate $\sim$} with respect to the
tree rooted at~\(x\). Any two different immediate subtrees are
disjoint, that is, no node from one connects to a node in the other. A
group of trees is a \emph{forest}\index{tree!forest}.

% Wrapping figure better declared before a paragraph
%
\begin{wrapfigure}[8]{r}[0pt]{0pt}
% [8] vertical lines
% {r} mandatory right placement
% [0pt] of margin overhang
\centering
\includegraphics[bb=71 652 123 710]{tree}% normal is [.. .. .. 715]
\caption{\label{fig:tree}}
\end{wrapfigure}
Trees can be used to depict terms as follows. A function call is a
tree whose root is the function name and the children are the trees
denoting the arguments. A tuple can be considered as having an
invisible function name represented by a node with a period
(\texttt{.}) in the tree, in which case the components of the tuple
are its children. For example, the tree in \fig~\vref{fig:tree} has
root~\fun{f} and leaves~\(0\), \(x\), \(1\) and~\(y\). Note how
variables \(x\)~and~\(y\) are set in italics to differentiate them
from function names \fun{x}~and~\fun{y}, set in sans\hyp{}serif. For
example \(\fun{d}((), \fun{e}([1]))\) can be interpreted as a tree of
root~\fun{d} and whose first immediate subtree is the representation
of the empty tuple \(()\) and second immediate subtree corresponds to
\(\fun{e}([1])\). \Fig~\ref{fig:tree} represents the tree
corresponding to \(\fun{f}((\fun{g}(0),(x,1)),(),\fun{g}(y))\). Note
that tuples, except the empty tuple, are not represented in the tree,
since they encode the structure, which is already visible. The number
of arguments of a function is called \emph{arity}\index{functional
  language!arity}. Functions with the same name but different arities
are permitted; for instance, we could have both \(\fun{c}(\fun{a}())\)
and \(\fun{c}(\fun{a}(x),0)\). This is called
\emph{overloading}\index{functional language!overloading}. To
distinguish the different uses, the arity of the function should be
indicated after a slash, for example \fun{c/1} and \fun{c/2}.

\section{Purely functional languages}
\label{sec:functional}

We only want to consider confluent systems because they define
functions. This property can be trivially enforced by setting an order
on the rules, as we did in previous examples. Another restriction we
impose is for normal forms to be \emph{values}\index{functional
  language!value}, to wit, they do not contain any function call which
can be further reduced. These two constraints define a \emph{purely
  functional
  language} \citep{Hughes_1989,Hinsen_2009}\index{functional
  language}. Notice that we do not require by construction that a
system terminates, although that is a desirable property. Not doing so
enables more expressivity, at the expense of some more caution, as we
shall soon realise.

We would like to further constrain the computation of function calls
by imposing that arguments are rewritten before the call is. This
strategy is named \emph{call\hyp{}by\hyp{}value}\index{functional
  language!call-by-value}.\label{def:call-by-value} Unfortunately, it
enables otherwise terminating programs to not terminate. For instance,
let us consider
\begin{equation*}
\fun{f}(x) \xrightarrow{\smash{\alpha}} 0.\qquad 
\fun{g}() \xrightarrow{\smash{\beta}} \fun{g}().
\end{equation*}
We have \(\fun{f}(\fun{g}()) \xrightarrow{\smash{\alpha}} 0\) but
\(\fun{f}(\fun{g}()) \xrightarrow{\smash{\beta}} \fun{f}(\fun{g}())
\xrightarrow{\smash{\beta}} \dots\) Despite this inconvenience, we
shall retain call\hyp{}by\hyp{}value because it facilitates some
analyses. (As an illustration of a more powerful strategy, the purely
functional language \Haskell \citep{DoetsVanEijck_2004} features
\emph{lazy evaluation}.) Also it allows us to restrict the shape of
the left\hyp{}hand sides, called \emph{patterns}\index{functional
  language!pattern}, to one, outermost function call. For instance, we
can then disallow for being useless a rule like
\begin{equation*}
\fun{plus}(x,\fun{plus}(y,z)) \rightarrow
\fun{plus}(\fun{plus}(x,y),z).
\end{equation*}
If the system is also terminating, we say that
\((\twoheadrightarrow)\) defines an \emph{evaluation}\index{functional
  language!evaluation}, or \emph{interpretation}\index{functional
  language!interpretation|see{evaluation}}, of the terms. For example,
the factorial\index{factorial|(}
\fun{fact/1}\index{fact@\textsf{fact/1}} can be defined by the ordered
system
\begin{equation}
\fun{fact}(0) \rightarrow 1;\qquad
\fun{fact}(n) \rightarrow n \cdot \fun{fact}(n-1).
\label{def:fact}
\end{equation}
Thus \(\fun{fact}(n) \twoheadrightarrow n!\) and the system details
how to reduce step by step \(\fun{fact}(n)\) into its value.

Most functional languages allow \emph{higher-order
  function}\index{functional language!higher-order functions}
definitions, whereas standard term\hyp{}rewriting systems do not. Such
an example would be the following higher\hyp{}order functional
program, where \(n \in \mathbb{N}\):
\begin{equation*}
\fun{f}(g,0) \rightarrow 1;
\qquad
\fun{f}(g,n) \rightarrow n \cdot g(g,n-1).
\qquad
\fun{fact}_1(n) \rightarrow \fun{f}(\fun{f},n).
\end{equation*}
Note that these two definitions are not recursive, yet
\fun{fact\(_1\)/1}\index{fact1@\textsf{fact$_1$/1}} computes the
factorial\index{factorial}.\index{factorial|)} An adequate theoretical
framework to understand higher\hyp{}order functions is
\emph{\(\lambda\)-calculus}\hspace*{-2.1pt}
\citep{HindleySeldin_2008,VanLeeuwen_1990b}.
\index{lambda-calculus@$\lambda$-calculus} In fact,
\(\lambda\)-calculus features prominently in the semantics of
programming languages, even not functional ones
\citep{Winskel_1993,Reynolds_1998,Pierce_2002,FriedmanWand_2008,TurbakGifford_2008}. We
prefer to work with rewrite systems because they offer native pattern
matching, whilst in
\(\lambda\)-calculus\index{lambda-calculus@$\lambda$-calculus} we
would have to encode it as a cascade of conditionals, which have
themselves to be encoded.

In the following, we show how to express linear data structures in a
purely functional language and how to run our programs on a computer.

\paragraph{Stacks}
\label{par:stacks}

Let us consider the abstract program\index{cat@\fun{cat/2}|(}
\index{cons@\fun{cons/2}|(}\index{nil@\fun{nil/0}|(}
\begin{equation*}
\fun{cat}(\fun{nil}(),t)     \xrightarrow{\smash{\alpha}} t;\qquad
\fun{cat}(\fun{cons}(x,s),t) \xrightarrow{\smash{\beta}}
                                \fun{cons}(x,\fun{cat}(s,t)).
\end{equation*}
It defines the function \fun{cat/2}\index{cat@\fun{cat/2}|)} that
catenates \index{stack!catenation} two \emph{stacks}. Functions
\fun{nil/0} and \fun{cons/2} are \emph{data
  constructors}\index{functional language!data constructor}, that is,
functions that are \emph{not} defined by the system: their irreducible
calls model data, so they are values and are allowed in patterns. The
function call \(\fun{nil}()\) denotes the empty stack and
\(\fun{cons}(x,s)\) is the stack obtained by putting the item~\(x\) on
top of the stack~\(s\), an action commonly referred as
\textsl{pushing~\(x\) on~\(s\)}. A non\hyp{}empty stack can be thought
of as a finite series of items that can only be accessed sequentially
from the top, as suggested by the analogy with a stack of material
objects. Let~\(T\) be the set of all possible terms and \(S \subseteq
T\) be the set of all stacks. Formally, \(S\)~can by defined by
\emph{induction}\index{induction!definition by
  $\sim$}\index{stack!inductive definition}\index{inductive
  definition|see{induction}} as the smallest set~\(\mathcal{S}\) such
that
\begin{itemize}

  \item \(\fun{nil}() \in \mathcal{S}\);\label{def:stack}

  \item if \(x \in T\) and \(s \in \mathcal{S}\), then
    \(\fun{cons}(x,s) \in \mathcal{S}\).

\end{itemize}

Note that, in rule~\(\beta\), if \(s\)~is not a stack, the recursion
implies that the function \fun{cat/2} is partial, not because the
rewrites never end, but because the normal form\index{rewrite
  system!normal form} is not a value. In operational terms, the
interpreter fails to rewrite the call for some arguments.

Let us set the abbreviations \(\el :=
\fun{nil}()\)\index{nil@\fun{nil/0}|)} and \(\cons{x}{s} :=
\fun{cons}(x,s)\)\index{cons@\fun{cons/2}|)}, after the convention of
the programming language
\Prolog \citep{SterlingShapiro_1994,Bratko_2000}. For instance, we may
write \([1|[2|[3|\el]]]\) in stead of
\(\fun{cons}(1,\fun{cons}(2,\fun{cons}(3,\fun{nil}())))\).  We can
further abbreviate the notation \(\cons{x_1}{\cons{x_2}{\dots
    \cons{x_n}{s}}}\) into \(\cons{x_1,x_2,\dots,x_n}{s}\) and
\(\cons{x}{\el}\) into \([x]\). For example, \([1|[2|[3|\el]]]\) is
more compactly written as \([1,2,3]\). Our system for
\fun{cat/2}\index{cat@\fun{cat/2}} now becomes a bit more legible:
\begin{equation}
\fun{cat}(        \el,t) \xrightarrow{\smash{\alpha}} t;\qquad
\fun{cat}(\cons{x}{s},t) \xrightarrow{\smash{\beta}}
\cons{x}{\fun{cat}(s,t)}.
\label{def:cat}\index{stack!catenation!definition}
\end{equation}
Finally, let us illustrate it with the following
evaluation:\index{stack!catenation!example}
\begin{equation*}
\fun{cat}([1,2],[3,4])
\!\xrightarrow{\smash{\beta}}\!
\cons{1}{\fun{cat}([2],[3,4])}
\!\xrightarrow{\smash{\beta}}\!
\cons{1}{\cons{2}{\fun{cat}(\el,[3,4])}}
\!\xrightarrow{\smash{\alpha}}\!
[1,2,3,4].
\end{equation*}

\paragraph{Abstract syntax trees}

Depending on the context, we may use the arborescent depiction of
terms to bring to the fore certain aspects of a computation. For
example, it may be interesting to show how parts of the output (the
right\hyp{}hand side) are actually \emph{shared}\index{sharing} with
the input (the left\hyp{}hand side), in other words, how much of the
data remains invariant through a given rule. This concept supposes
that terms reside in some sort of space and that they can be referred
to from different other terms. This abstract space serves as a model
of a computer \emph{memory}\index{memory}. Consider for instance in
\fig~\vref{fig:cat_dag} the same definition of~\fun{cat/2} as given
in~\eqref{def:cat}. The arrows on certain edges denote some data
sharing. When trees are used to visualise terms, they are called
\emph{abstract syntax trees}\index{tree!abstract syntax $\sim$}. When
some trees share subtrees, the whole forest is called a \emph{directed
  acyclic graph}.\index{directed acyclic graph}
\index{tree|see{directed acyclic graph}}
\begin{figure}[h]
\centering
\includegraphics[bb=73 654 300 720]{cat_dag}
\caption{Definition of \fun{cat/2} with directed acyclic graphs
\label{fig:cat_dag}}
\end{figure}

\section{Analysis of algorithms}

The branch of theoretical computer science devoted to the mathematical
study of the efficiency of programs has been pioneered by Donald
Knuth, who named it \emph{analysis of algorithms}
\citep{SedgewickFlajolet_1996,Knuth_1997}. Given a function
definition, this approach consists basically in three steps: defining
a measure on the arguments, which represents their size; defining a
measure on time, which abstracts the wall\hyp{}clock time; expressing
the abstract time needed to compute calls to that function in terms of
the size of its arguments. This function models the efficiency and is
called the \emph{cost}\index{cost} (the lower the cost, the higher the
efficiency). For example, when sorting objects, also called
\emph{keys}\index{key|see{sorting}} in this context, by comparing
them, the input size is the number of keys and the abstract unit of
time is often one comparison, so the cost is the mathematical function
which associates the number of keys and the number of comparisons
to sort them.


\mypar{Exact cost}

Rewrite systems enable a rather natural notion of cost for functional
programs: it is the number of rewrites to reach the value of a
function call, assuming that the arguments are values. In other words,
it is the number of calls needed to compute the value. To gain some
generality, we need to relate the cost to a measure of the size of the
input. In the case of stacks, this is the number of items it
contains. For instance, let us recall the catenation of two stacks in
definition~\eqref{def:cat}\index{cat@\fun{cat/2}}:
\begin{equation*}
\fun{cat}(        \el,t) \xrightarrow{\smash{\alpha}} t;\qquad
\fun{cat}(\cons{x}{s},t) \xrightarrow{\smash{\beta}}
\cons{x}{\fun{cat}(s,t)}.
\end{equation*}
We observe that \(t\)~is invariant, so the cost depends only on the
size of the first argument. Let
\(\C{\fun{cat}}{n}\)\index{cat@$\C{\fun{cat}}{n}$} be the cost of the
call \(\fun{cat}(s,t)\)\index{cat@\fun{cat/2}}, where \(n\)~is the
size of~\(s\). Rule~\(\alpha\) leads to the equation
\(\C{\fun{cat}}{0} \eqn{\smash{\alpha}} 1\) and rule~\(\beta\) to
\(\C{\fun{cat}}{n+1} \eqn{\smash{\beta}} 1 + \C{\fun{cat}}{n}\),
yielding together \(\C{\fun{cat}}{n} = n +
1\).\label{cost:cat}\index{stack!catenation!cost}


\mypar{Extremal costs}
\index{cost!extremal $\sim$}

When considering sorting programs based on comparisons, the cost
varies depending on the algorithm and it also often depends on the
original partial ordering of the keys, thus size does not capture all
aspects needed to assess efficiency. This quite naturally leads to
consider bounds on the cost: for a given input size, we seek the
configurations of the input that minimise and maximise the cost,
respectively called \emph{best case} and \emph{worst case}. For
example, some sorting algorithms have their worst case when the keys
are already sorted, others when they are sorted in reverse order, etc.


\mypar{Average cost}
\label{par:mean_sort}
\index{cost!average $\sim$}
\index{cost!mean $\sim$|see{cost, average}}

Once we obtain bounds on a cost, the question about the \emph{average}
or \emph{mean cost} \citep{VitterFlajolet_1990}
\citep[\S{}1.2.10]{Knuth_1997} arises as well. It is computed by
taking the arithmetic mean of the costs for all possible inputs of a
given size. Some care is necessary, as there must be a finite number
of such inputs. For instance, to assess the mean cost of sorting
algorithms based on comparisons, it is usual to assume that the input
is a series of \(n\)~\emph{distinct
  keys}\index{sorting!key!uniqueness} and that the sum of the costs is
taken over all its \emph{permutations}\index{permutation}, thus
divided by~\(n!\), the number of permutations of size~\(n\). The
uniqueness constraints actually allows the analysis to equivalently,
and more simply, consider the permutations of \((1,2,\dots,n)\). Some
sorting algorithms, like \emph{merge sort}
\cite[\S{}5.2.4]{Knuth_1998} \cite[\S{}2.3]{CLRS_2009} or
\emph{insertion sort} \cite[\S{}5.2.1]{Knuth_1998}
\cite[\S{}2.1]{CLRS_2009}, have their average cost
\emph{asymptotically equivalent} to their maximum cost, that is, for
increasingly large numbers of keys, the ratio of the two costs become
arbitrarily close to~\(1\)\index{cost!asymptotic $\sim$}. Some others,
like \emph{Hoare's sort}, also known as \emph{quicksort}
\cite[\S{}5.2.2]{Knuth_1998} \cite[\S{}7]{CLRS_2009}, have the growth
rate of their average cost being of a lower magnitude than the maximum
cost, on an asymptotic scale \cite[\S{}9]{GrahamKnuthPatashnik_1994}.

\paragraph{On-line vs. off-line}
\label{par:online_vs_offline}

Sorting algorithms can be distinguished depending on whether they
operate on the whole series of keys, or key by key. The former are
said \emph{off\hyp{}line}\index{off-line algorithm}, as keys are not
sorted while they are coming in, and the latter are called
\emph{on\hyp{}line}\index{on-line algorithm}, as the sorting process
can be temporally interleaved with the input process. For example,
insertion sort is an on\hyp{}line algorithm, whereas Hoare's sort is
not because it is an instance of the divide\hyp{}and\hyp{}conquer
strategy that splits the data. This distinction is pertinent in other
contexts as well, like with algorithms that are intrinsically
\emph{sequential}, instead of enabling some degree of
\emph{parallelism}, e.g., a database is updated by a series of atomic
requests, but requests on non\hyp{}overlapping parts of the data might
be performed in parallel.

\mypar{Amortised cost}
\label{par:amortised_cost}

Sometimes an update is costly because it is delayed by an imbalance in
the data structure that calls for an immediate remediation, but this
remediation itself may lead to a state such that subsequent operations
are faster than if the costly update had not happen. Therefore, when
considering a series of updates, it may be overly pessimistic to
cumulate the maximum costs of all the operations considered in
isolation. Instead, \emph{amortised
  analysis} \citep{Okasaki_1998a} \citep[\S{}17]{CLRS_2009}\index{cost!amortised
  $\sim$}\index{amortised analysis|see{cost, amortised}} takes into
account the interactions between updates, so a lower maximum bound on
the cost is derived. Note that this kind of analysis is inherently
different from the average case analysis, as its object is the
composition of different functions instead of independent calls to the
same function on different inputs. Amortised analysis is a worst case
analysis of a sequence of updates, not of a single one.


\paragraph{Aggregate analysis}
\label{par:aggregate}
\index{aggregate cost|see{cost, amortised}}
\index{enumerative combinatorics}

Let us consider a counter enumerating the integers from~\(0\) to~\(n\)
in binary by updating an array containing
bits \cite[\S{}17.1]{CLRS_2009}. In the worst case, an increment leads
to inverting all the bits. The number~\(m\) of bits of~\(n\) can be
found by setting \(n := \sum_{i=0}^{m-1}{b_i2^i}\), where the~\(b_i\)
are the bits and \(b_{m-1}=1\). Then
\begin{equation}
2^{m-1} \leqslant n < 2^m \Rightarrow m - 1 \leqslant \lg n
< m \Rightarrow m = \floor{\lg n} + 1,
\label{eq:num_of_bits}
\end{equation}
where \(\floor{x}\) (\textsl{floor
  of~\(x\)})\index{floor@$\floor{x}$|see{floor function}}\index{floor
  function} is the greatest integer less than or equal to~\(x\) and
\(\lg n\)~is the binary logarithm of~\(n\). The cost of the
\(n\)~increments is thus bounded from above by \(n\lg n + n \sim n\lg
n\), as \(n \rightarrow \infty\).

\hspace*{-0.25pt} A little observation reveals that this upper bound
is overly pessimistic, as carry propagation clears a series of
rightmost bits to~\(0\), so the next addition will flip only one bit,
the following two etc. as shown in \fig~\vref{fig:flips}, where bits
about to be flipped at the next increment are set in boldface type.
\begin{figure}[t]
\centering
\subfloat[Bit flips\label{fig:flips}]{\includegraphics{flips}}
\qquad
\subfloat[$F(n) = \sum_{i \geqslant 0}\floor{n/2^i}$, with $n=22$\label{fig:ruler}]%
{\includegraphics{ruler}}
\caption{Counting bits vertically and diagonally}
\end{figure}
Counting the flips \emph{vertically} reveals that the bit
corresponding to~\(2^0\), that is, the rightmost bit, flips every
time. The bit of~\(2^1\) flips once every two increments, so, from
\(0\)~to~\(n\), it flips \(\floor{n/2^1}\)~times. In general, the bit
of~\(2^k\) flips \(\floor{n/2^k}\)~times. Therefore, the total number
of flips~\(F(n)\) in a sequence of \(n\)~increments is
\begin{equation}
F(n) := \sum_{k \geqslant 0}{\left\lfloor\frac{n}{2^k}\right\rfloor}.
\label{eq:F}
\end{equation}
The sum is actually always finite, as illustrated by the example in
\fig~\vref{fig:ruler}. There, we can see \emph{diagonally} that
\(1\)-bits at position~\(j\) appear in positions~\(j-1\) down
to~\(0\), so account for \(2^j + 2^{j-1} + \dots + 2^0 = 2^{j+1}-1\).
In all generality, let \(n := 2^{e_r} + \dots + 2^{e_1} + 2^{e_0} >
0\), with \(e_r > \dots > e_1 > e_0 \geqslant 0\) and \(r \geqslant
0\). The naturals \(e_i\)~are the positions of the \(1\)-bits in the
binary notation of~\(n\). The power \(2^{e_r}\) corresponds to the
leftmost bit in the binary expansion of~\(n\), so \(e_r+1\)~is equal
to the number of bits of~\(n\), which is known from
equation~\eqref{eq:num_of_bits}:
\begin{equation}
e_r = \floor{\lg n}.\label{eq:e_r}
\end{equation}
We can now give a closed form for \(F(n)\) as follows:
\begin{equation}
F(n) = \sum_{i=0}^{r}(2^{e_i+1} - 1) = 2n - \nu_n,
\label{eq:ruler_nu}
\end{equation}
where \(\nu_n := r + 1\)\index{bit sum@$\nu_n$|see{bit sum}} is the
sum of the bits of~\(n\), or, equivalently, the number of
\(1\)-bits. It is called many names, like \emph{population count},
\emph{sideways sum}, \emph{bit sum}\index{bit sum} or \emph{Hamming
  weight}; for example, in \fig~\vref{fig:ruler}, we can read \(F(22)
= 41 = 2 \cdot 22 - 3\). Furthermore, we have the following intuitive
tight bounds for any \(n>0\):
\begin{equation*}
1 \leqslant \nu_{n} \leqslant \floor{\lg n} + 1,
\end{equation*}
because equality~\eqref{eq:e_r} establishes that \(\floor{\lg n} + 1\)
is the number of bits of~\(n\). Therefore, \(2n - \floor{\lg n} - 1
\leqslant F(n) \leqslant 2n\). By l'Hospital rule, \(\lim_{n \to
  +\infty}{(\lg n/n)} = \lim_{n \to +\infty}(1/n\ln 2) = 0\), where
\(\ln n\)~is the \emph{natural logarithm}. Therefore,
\begin{equation*}
F(n) \sim 2n,\;\, \text{as \(n \rightarrow \infty\)}.
\end{equation*}
Two enumerations (counting vertically and diagonally) have shown that
the exact total number of flips is of a lower magnitude than expected.

This example resorts to a particular kind of amortised analysis called
\emph{aggregate analysis}, because it relies on enumerative
combinatorics \citep{Stanley_1999a,Stanley_1999b,Martin_2001} to reach
its result (it aggregates positive partial amounts, often in different
manners, to obtain the total cost). A visually appealing variation on
the previous example consists in determining the average number of
\(1\)-bits in the binary notation of the integers from~\(0\) to~\(n\)
\citep{Bush_1940}.

\section{Inductive proofs}

Let us notice that \(\fun{cat}([1], [2,3,4]) \twoheadrightarrow
[1,2,3,4] \twoheadleftarrow \fun{cat}([1,2],
[3,4])\)\index{cat@\fun{cat/2}|(}. It is enlightening to create
\emph{equivalence classes} of terms that are joinable, based
on~\((\equiv)\) \index{equivalence@$a \equiv b$|see{equivalence of
    expressions}} \index{equivalence!$\sim$ of expressions} defined
as: \(a \equiv b\) if there exists a value~\(v\) such that \(a
\xrightarrow{\smash[t]{*}} v\) and \(b \xrightarrow{\smash{*}}
v\). For instance, \(\fun{cat}([1,2], [3,4]) \equiv \fun{cat}([1],
[2,3,4])\). The relation (\(\equiv\)) is indeed an equivalence
because it is
\begin{itemize*}
 
  \item \emph{reflexive}: \(a \equiv a\);

  \item \emph{symmetric}: if \(a \equiv b\), then \(b \equiv a\);

  \item \emph{transitive}: if \(a \equiv b\) and \(b \equiv c\), then
    \(a \equiv c\).

\end{itemize*}
Of some interest are the following facts. If \(f(x)\) and~\(f(y)\)
have a value, then \(x \equiv y\) implies \(f(x) \equiv f(y)\). If
\(x_1 \leftarrow x_2 \twoheadrightarrow x_3 \equiv x_4 \rightarrow x_5
\twoheadleftarrow x_6\), then \(x_1 \equiv x_2 \equiv x_3 \equiv x_4
\equiv x_5 \equiv x_6\). In the case \(x \twoheadrightarrow z
\xrightarrow{\smash{\alpha}} t \twoheadleftarrow y\), we use the
special notation \(x \Rra{\alpha} y\) to underline the role played by
rule~\(\alpha\) in the equivalence, instead of simply \(x \equiv
y\). If we want to prove equivalences with variables ranging over
infinite sets, like \(\fun{cat}(s,\fun{cat}(t,u)) \equiv
\fun{cat}(\fun{cat}(s,t),u)\)\index{cat@\fun{cat/2}|)}, we need some
induction principle.


\mypar{Well-founded induction}
\label{par:well-founded}
\index{induction!well-founded $\sim$}

We define a \emph{well\hyp{}founded order}
\citep{Winskel_1993}\index{induction!well-founded order} on a
set~\(A\) as being a binary relation \((\succ)\) which does not have
any \emph{infinite descending chains}\index{induction!infinite
  descending chain}, to wit, no \(a_0 \succ a_1 \succ \dots\) The
\emph{well\hyp{}founded induction principle} then states that, for any
predicate~\(\aleph\), \(\forall a \in A.\aleph(a)\) is implied by
\(\forall a.(\forall b.a \succ b \Rightarrow \aleph(b)) \Rightarrow
\aleph(a)\). Because there are no infinite descending chains, any
subset \(B \subseteq A\) contains minimal elements \(M \subseteq B\),
that is, there is no \(b \in B\) such that \(a \succ b\), if \(a \in
M\). In this case, proving by well\hyp{}founded induction degenerates
into proving \(\aleph(a)\) for all \(a \in M\). When \(A=\mathbb{N}\),
this principle is called \emph{mathematical (complete) induction}
\citep{Buck_1963}\index{induction!mathematical
  $\sim$}. \emph{Structural induction}\index{induction!structural
  $\sim$} is another particular case where \(t \succ s\) holds if, and
only if, \(s\)~is a proper
\emph{subterm}\index{term!subterm}\index{term!proper subterm}
of~\(t\), namely, the abstract syntax tree of~\(s\) is included in the
tree of~\(t\) and \(s \neq t\). Sometimes, a restricted form is
enough. For instance, we can define \(\cons{x}{s} \succ
s\)\index{cons@\fun{cons/2}}, for any term~\(x\) and any stack~\(s \in
S\). (Both \(x\)~and~\(s\) are \emph{immediate
  subterms}\index{term!immediate subterm} of
\(\cons{x}{s}\)\index{cons@\fun{cons/2}}.) There is no infinite
descending chain since \(\el\)~is the unique minimal element of~\(S\):
no~\(s\) satisfies \(\el \succ s\); so the basis is \(t=\el\) and
\(\forall t.(\forall s.t \succ s \Rightarrow \aleph(s)) \Rightarrow
\aleph(t)\) degenerates into \(\aleph(\el)\).

\paragraph{Associativity}
\label{proof:assoc_cat}

Let us prove the associativity\index{stack!catenation!associativity}
\index{induction!example} of \fun{cat/2}\index{cat@\fun{cat/2}}, which
we express formally by \(\pred{CatAssoc}{s,t,u} \colon
\fun{cat}(s,\fun{cat}(t,u)) \equiv
\fun{cat}(\fun{cat}(s,t),u)\)\index{CatAssoc@\predName{CatAssoc}|(},
where \(s\), \(t\) and~\(u\) are stack values. We apply the
well\hyp{}founded induction principle to the structure of~\(s\), so we
must establish
\begin{itemize*}

  \item the basis \(\forall t,u \in S.\pred{CatAssoc}{\el,t,u}\);

  \item step \(\forall s,t,u \in S.\pred{CatAssoc}{s,t,u}
    \Rightarrow \forall x \in T.\pred{CatAssoc}{\cons{x}{s},t,u}\).

\end{itemize*}
The base case is direct: \(\fun{cat}(\el,\fun{cat}(s,t))
\xrightarrow{\smash{\alpha}} \fun{cat}(s,t)
\xleftarrow{\smash{\alpha}}
\fun{cat}(\fun{cat}(\el,t),u)\)\index{cat@\fun{cat/2}}. Let us assume
now \(\pred{CatAssoc}{s,t,u}\), called the \emph{induction
  hypothesis}\index{induction!$\sim$ hypothesis}, and let us prove
\(\pred{CatAssoc}{\cons{x}{s},t,u}\)\index{CatAssoc@\predName{CatAssoc}|)},
for any term~\(x\). We have
\begin{equation*}
\begin{array}{r@{\;}l@{\;}l@{\qquad}r@{}}
  \fun{cat}(\cons{x}{s},\fun{cat}(t,u))
& \Rra{\beta}
& \cons{x}{\fun{cat}(s,\fun{cat}(t,u))}\\
& \equiv
& \cons{x}{\fun{cat}(\fun{cat}(s,t),u)}
& (\pred{CatAssoc}{s,t,u})\\
& \Lla{\beta}
& \fun{cat}(\cons{x}{\fun{cat}(s,t)},u)\\
& \Lla{\beta}
& \fun{cat}(\fun{cat}(\cons{x}{s},t),u).\index{cat@\fun{cat/2}}
\end{array}
\end{equation*}
Thus\index{CatAssoc@\predName{CatAssoc}}
\(\pred{CatAssoc}{\cons{x}{s},t,u}\) holds and \(\forall s,t,u \in
S.\pred{CatAssoc}{s,t,u}\)\index{CatAssoc@\predName{CatAssoc}}.\hfill\(\Box\)

Writing \(\fun{cat}(\cons{x}{s},\fun{cat}(t,u))
\xrightarrow{\smash{\beta}} \cons{x}{\fun{cat}(s,\fun{cat}(t,u))}\) is
incorrect because the semantics of our functional language requires
that arguments are evaluated before the call is. Nevertheless,
if~\(s\), \(t\) and~\(u\) are not values, it is sufficient that they
have one for \(\pred{CatAssoc}{s,t,u}\) to hold as well; for example,
if \(f(s)\) has a value~\(v\), then \(\pred{CatAssoc}{f(s),t,u}\) is
true because \(\fun{cat}(f(s),\fun{cat}(t,u)) \twoheadrightarrow
\fun{cat}(v,\fun{cat}(t,u)) \equiv \fun{cat}(\fun{cat}(v,t),u)
\twoheadleftarrow \fun{cat}(\fun{cat}(f(s),t),u)\). Therefore,
termination is an important property we need to check.
 
\mypar{Termination}
\label{par:ackermann}
When defining our purely functional language, we allowed programs to
not terminate. We could actually have imposed some syntactic
restrictions on recursive definitions in order to guarantee the
termination of all functions. A well\hyp{}known class of such
terminating functions makes exclusively use of a bridled form of
recursion called \emph{primitive recursion}
\citep{Robinson_1947,Robinson_1948}\index{termination!primitive
  recursion}. Unfortunately, many useful functions do not fit easily
in this framework and, as a consequence, most functional languages
leave to the programmers the responsibility to check the termination
of their programs. For theoretical reasons, it is not possible to
provide a general criterion for termination, but many standard rules
exists that cover many usages. Consider the following example where
\(m,n \in \mathbb{N}\):
\begin{equation*}
\begin{array}{@{}r@{\;}l@{\;}l@{}}
\fun{ack}(0,n)     & \xrightarrow{\smash{\theta}} & n+1;\\
\fun{ack}(m+1,0)   & \xrightarrow{\smash{\iota}}  & \fun{ack}(m,1);\\
\fun{ack}(m+1,n+1) & \xrightarrow{\smash{\kappa}} 
                   & \fun{ack}(m,\fun{ack}(m+1,n)).
\end{array}
\end{equation*}
This is a simplified form of Ackermann's
function\index{termination!Ackermann's
  function}\index{ack@\fun{ack/2}|(}, an early example of a total
computable function which is not primitive recursive. It makes use of
double recursion and two parameters to grow values as towers of
exponents, for example, \(\fun{ack}(4,3) \twoheadrightarrow
2^{2^{65536}} - 3\). It is not obviously terminating, because if the
first argument does decrease, the second largely increases. Let us
define a well\hyp{}founded ordering on pairs, called
\emph{lexicographic order}\index{induction!lexicographic order}. Let
\((\succ_A)\) and \((\succ_B)\) be well\hyp{}founded orders on the
sets \(A\) and~\(B\). Then \((\succ_{A \times B})\) defined as follows
on \(A \times B\) is well\hyp{}founded:
\begin{equation}
(a_0,b_0) \succ_{A \times B} (a_1,b_1) :\Leftrightarrow \text{\(a_0
    \succ_A a_1\) or (\(a_0 = a_1\) and \(b_0 \succ_B b_1\)).}
\label{def:lexico}
\end{equation}
If \(A=B=\mathbb{N}\) then \((\succ_A) = (\succ_B) = (>)\). To prove
that \(\fun{ack}(m,n)\) terminates for all \(m,n \in \mathbb{N}\),
first, we must find a well\hyp{}founded order on the calls
\(\fun{ack}(m,n)\), that is, the calls must be totally ordered without
any infinite descending chain. In this case, a lexicographic order on
\((m,n) \in \mathbb{N}^2\) extended to \(\fun{ack}(m,n)\) works:
\begin{equation*}
\fun{ack}(a_0,b_0) \succ \fun{ack}(a_1,b_1) :\Leftrightarrow 
\text{\(a_0 > a_1\) or (\(a_0 = a_1\) and \(b_0 > b_1\)).}
\end{equation*}
Clearly, \(\fun{ack}(0,0)\) is the minimum element. Second, we must
prove that \fun{ack/2} rewrites to smaller calls. We are only
concerned with rules \(\iota\)~and~\(\kappa\). With the former, we
have \(\fun{ack}(m+1,0) \succ \fun{ack}(m,1)\). With the latter, we
have \(\fun{ack}(m+1,n+1) \succ \fun{ack}(m+1,n)\) and
\(\fun{ack}(m+1,n+1) \succ \fun{ack}(m,p)\), for all values~\(p\), in
particular when \(\fun{ack}(m+1,n) \twoheadrightarrow
p\)\index{ack@\fun{ack/2}|)}.\hfill\(\Box\)

A series of examples of termination\index{termination}\index{rewrite
  system!termination} proofs for term\hyp{}rewriting systems has been
published by \cite{Dershowitz_1995,ArtsGiesl_2001}. An accessible
survey is provided by \cite{Dershowitz_1987}. \cite{Knuth_2000a}
analysed some famously involved recursive functions.

\section{Implementation}
\label{sec:implementation}

\mypar{Translation to \Erlang}
\index{functional language!Erlang@\Erlang}

It is always enjoyable to have computers actually evaluate our
function calls. We briefly introduce here \Erlang, a functional
language that contains a pure core \citep{Armstrong_2007}. A
\emph{module} is a collection of function definitions. The syntax of
\Erlang is very close to our formalism and our previous rewrite
systems become\index{stack!catenation!$\sim$ in
  \Erlang}\index{factorial}
\begin{verbatim}
-module(mix).
-export([cat/2,fact/1]).

cat(   [],T) -> T;
cat([X|S],T) -> [X|cat(S,T)].

fact(N) -> f(fun f/2,N).

f(_,0) -> 1;
f(G,N) -> N * G(G,N-1).
\end{verbatim}
The differences are the headers and the lexical conventions of setting
variables in big capitals and to mute unused variables in patterns
with an underscore (\verb|_|). Moreover, the expression \verb|fun f/2|
denotes~\fun{f/2} when used in stead of a value. From the \Erlang
shell, we can compile and run some examples:
\begin{verbatim}
1> c(mix).
{ok,mix}
2> mix:cat([1,2,3],[4,5]).
[1,2,3,4,5]
3> mix:fact(30).
265252859812191058636308480000000
\end{verbatim}
Note that \Erlang features exact integer arithmetic and that the order
of the definitions is irrelevant.

\mypar{Translation to \Java}
\label{par:java}
\index{Java@\Java|(}

Functional programs on stacks can be systematically translated into
\Java, following designs similar to those initially published
by \cite{FelleisenFriedman_1997}, \cite{Bloch_2003}
and \cite{Sher_2004}. This operation should transfer some interesting
properties proved on the source to the target language: the whole
point hinges on how the mathematical approach presented earlier, both
with structural induction and functional programming, leads to trusted
\Java programs and, therefore, constitutes a solid bridge between
mathematics and computer science.

Of course, the programs discussed in this book are extremely short and
the topic at hand thus resorts to `programming in the small,' but,
from the vantage point of software engineering, these functional
programs can then be considered as \emph{formal specifications} of the
\Java programs, and inductive proofs may be thought as instances of
\emph{formal methods}, like the ones used to certify telecommunication
protocols and critical embedded systems. Therefore, this book may be
used as a prerequisite to a software engineering course, but also to
an advanced programming course.

\paragraph{Design Pattern}

The design pattern in \Java which models a stack \index{stack!$\sim$
  in \Java} relies on polymorphic methods, recursive classes and
generics. A generic and abstract class \texttt{Stack} captures the
essence of a stack as follows:
\begin{alltt}
// Stack.java
\public \abstractX \class Stack<Item> \{
  \public \final NStack<Item> push(\final Item item) \{
    \return \new NStack<Item>(item,\this); \}
\}
\end{alltt}
A stack is empty or not and the class \texttt{Stack} is abstract
because it asserts that both are stacks and that they share common
functionalities, like the method \texttt{push}, which is a wrapper
around the constructor of non\hyp{}empty stacks, \texttt{NStack}, so
both empty and non\hyp{}empty stacks are augmented by the same
method. The argument \texttt{item} of \texttt{push} is declared
\final{} because we want it to be constant in the body of the method,
following the functional paradigm. The empty stack~\(\el\) is mapped
to an extension \texttt{EStack} of \texttt{Stack}, capturing the
relationship `An empty stack is a stack.' The class \texttt{EStack}
contains no data.
\begin{alltt}
// EStack.java
\public \final \class EStack<Item> \extends Stack<Item> \{\}
\end{alltt}
The non\hyp{}empty stack is logically encoded by \texttt{NStack},
another subclass of \texttt{Stack}:
\begin{alltt}
// NStack.java
\public \final \class NStack<Item> \extends Stack<Item> \{
  \private \final Item head;
  \private \final Stack<Item> tail;

  \public NStack(\final Item item, \final Stack<Item> stack) \{
    head = item; tail = stack; \}
\}
\end{alltt}
The field \texttt{head} models the first item of a stack and
\texttt{tail} corresponds to the rest of the stack (so \texttt{NStack}
is a recursive class). The constructor just initialises
these. Importantly, both are declared \texttt{final} to express that
we do not expect reassignments after the first instantiation. Just as
in the functional language, every time a new stack is needed, instead
of modifying another with a side\hyp{}effect, a new one is created,
perhaps reusing others as constant components.

\paragraph{Catenation of stacks}

As an illustration\index{stack!catenation!$\sim$ in \Java|(}, let us
recall definition~\eqref{def:cat} \vpageref{def:cat}
\index{cat@\fun{cat/2}|(} for catenating two stacks:
\begin{equation*}
\fun{cat}(        \el,t) \xrightarrow{\smash{\alpha}} t;\qquad
\fun{cat}(\cons{x}{s},t) \xrightarrow{\smash{\beta}}
                                          \cons{x}{\fun{cat}(s,t)}.
\end{equation*}
The translation into our \Java class hierarchy is as follows. The
first argument of \fun{cat/2} is a stack, corresponding to
\texttt{this} inside our classes \texttt{EStack} and
\texttt{NStack}. Therefore, the translation of \fun{cat/2} is an
abstract \Java method in class \texttt{Stack}, with one parameter (the
second of \fun{cat/2}):
\begin{alltt}
\public \abstractX Stack<Item> cat(\final Stack<Item> t);
\end{alltt}
Rule~\(\alpha\) applies only if the current object represents~\(\el\),
so the corresponding translation is a method of
\texttt{EStack}. Dually, rule~\(\beta\) leads to a method of
\texttt{NStack}. The former returns its argument, so the translation
is
\begin{alltt}
\public Stack<Item> cat(\final Stack<Item> t) \{ \return t; \}
\end{alltt}
The latter returns an object of~\texttt{NStack} corresponding to
\(\cons{x}{\fun{cat}(s,t)}\). It is built by translating this stack
from the bottom up: translate
\(\fun{cat}(s,t)\)\index{cat@\fun{cat/2}|)} and then push~\(x\). Let
us recall that \(s\)~is the tail of \(\cons{x}{s}\) on the
left\hyp{}hand side of rule~\(\beta\), hence \(\cons{x}{s}\) is
\texttt{this} and \(s\)~corresponds to \texttt{this.tail}, or, simply,
\texttt{tail}. Similarly, \(x\)~is~\texttt{head}. Finally,
\begin{alltt}
\public NStack<Item> cat(\final Stack<Item> t) \{
  \return tail.cat(t).push(head);
\}
\end{alltt}
\index{stack!catenation!$\sim$ in \Java|)}
\index{Java@\Java|)}
