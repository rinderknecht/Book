%%-*-latex-*-

\chapter*{Foreword}
\addcontentsline{toc}{chapter}{Foreword}
\thispagestyle{empty}

This book addresses a priori different audiences whose common interest
is functional programming.

\emph{For undergraduate students}, we offer a very progressive
introduction to functional programming, with long developments about
algorithms on stacks and some kinds of binary trees. We also study
memory allocation through aliasing (dynamic data-sharing), the role of
the control stack and the heap, automatic garbage collection (GC), the
optimisation of tail calls and the total allocated memory. Program
transformation into tail form, higher\hyp{}order functions and
continuation\hyp{}passing style are advanced subjects presented in the
context of the programming language \Erlang. We give a technique for
translating short functional programs to \Java.

\emph{For postgraduate students}, each functional program is
associated with the mathematical analysis of its minimum and maximum
cost (efficiency), but also its average and amortised cost. The
peculiarity of our approach is that we use elementary concepts
(elementary calculus, induction, discrete mathematics) and we
systematically seek explicit bounds in order to draw asymptotic
equivalences. Indeed, too often textbooks only introduce Bachmann
notation \(\mathcal{O}(\cdot)\) for the dominant term of the cost,
which provides little information and may confuse
beginners. Furthermore, we cover in detail proofs of properties like
correctness, termination and equivalence. An introduction to
operational semantics is given in the context of the programming
language \OCaml, with a hint of type inference.

\emph{For the professionals} who do not know functional languages and
who must learn how to program with the language \XSLT, we propose an
introduction which dovetails the part dedicated to undergraduate
students. The reason of this unusual didactic choice lies on the
observation that \XSLT is rarely taught in college, therefore
programmers who have not been exposed to functional programming face
the two challenges of learning a new paradigm and use \XML for
programming: whereas the former puts forth recursion, the latter
obscures it because of the inherent verbosity of \XML. By learning
first an abstract functional language, and then \XML, we hope for a
transfer of skills towards the design and implementation in \XSLT
without mediation.

This book has also been written with the hope of enticing the reader
into theoretical computing, like programming language semantics,
formal logic, lattice path counting and analytic combinatorics.

I thank Nachum Dershowitz, Fran\c{c}ois Pottier, Sri Gopal Mohanty,
Walter B\"ohm, Philippe Flajolet, Francisco Javier Bar\'on L\'opez,
Ham Jun-Wu and Kim Sung~Ho for their technical help. Most of this book
has been written while I was working at Konkuk University (Seoul,
Republic of Korea), and some parts added while I was at E\"otv\"os
Lor\'and University (Budapest, Hungary).

Please inform me at \url{rinderknecht@free.fr} of any error.

\bigskip

\hfill{}Montpellier, France,

%\hfill\today.

\hfill{}14th January \oldstylenums{2015}.

\hfill{}Ch. Rinderknecht
