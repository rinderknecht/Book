\paragraph{Answer 1.}

The method to answer these questions is simply to try small words by
constructing them in order to satisfy the constraints.
\begin{enumerate}

  \item The \label{aba} shortest word \(x\) belonging to L(r) is found
    by taking \(\epsilon\) in place of \(\lparen a \disjM{}
    b\rparen\kleeneM\). So \(x = aba\). Let us check if \(x \in L(s)\)
    or not. \(L(s)\) is made of the union of four sub-languages
    (subsets). To make this clear, let us remove the useless
    parentheses on the right side:
    \begin{equation*}
    s = \lparen ab \rparen\kleeneM{} \, \disjM{} \, \lparen ba
    \rparen\kleeneM{} \, \disjM{} \, a\kleeneM{} \, \disjM{} \,
    b\kleeneM.
    \end{equation*}
    Therefore, membership tests on \(L(s)\) have to be split
    into four: one membership test on \(\lparen ab
    \rparen\kleeneM\), one on \(\lparen ba \rparen\kleeneM\),
    one on \(a\kleeneM\) and another one on
    \(b\kleeneM\). In other words, \(x \in L(s)\) is equivalent to
    \begin{equation*}
    x \in L(\lparen ab \rparen\kleeneM) \;
     \text{or} \; x \in L(\lparen ba \rparen\kleeneM) \;
     \text{or} \; x \in L(a\kleeneM) \; \text{or} \; x
     \in L(b\kleeneM).
    \end{equation*}
    Let us test the membership with \(x = aba\):
    \begin{enumerate}
  
      \item The words in \(L(\lparen ab \rparen\kleeneM)\) are
        \(\epsilon\), \(ab\), \(abab\ldots\) Thus \(aba \not\in
        L(\lparen ab \rparen\kleeneM)\).

      \item The words in \(L(\lparen ba \rparen\kleeneM)\) are
        \(\epsilon\), \(ba\), \(baba\ldots\) Hence \(aba \not\in
        L(\lparen ba \rparen\kleeneM)\).

      \item The words in \(L(a\kleeneM)\) are \(\epsilon\), \(a\),
        \(aa\ldots\) Therefore \(aba \not\in L(a\kleeneM)\).

      \item The words in \(L(b\kleeneM)\) are \(\epsilon\), \(b\),
        \(bb\ldots\) So \(aba \not\in L(b\kleeneM)\).
  
    \end{enumerate}
    The conclusion is \(aba \not\in L(s)\).

    \item What is the shortest word belonging to \(L(s)\)?  Since the
      four sub-languages composing \(L(s)\) are starred, it means that
      \(\epsilon \in L(s)\). Since we showed at the item~(\ref{aba})
      that \(aba\) is the shortest word of \(L(r)\), it means that
      \(\epsilon \not\in L(r)\) because \(\epsilon\) is of length
      \(0\).

    \item This question is a bit more difficult. After a few tries, we
      cannot find any~\(x\) such that \(x \in L(r)\) and \(x \in
      L(s)\). Then we may try to prove that \(L(r) \cap L(s) =
      \varnothing\), \emph{i.e.,} there is no such~\(x\). How should we
      proceed? The idea is to use the decomposition of \(L(s)\) into
      for sub-languages and try to prove
      \begin{align*}
        L(r) \cap L(\lparen ab \rparen\kleeneM) &= \varnothing,\\
        L(r) \cap L(\lparen ba \rparen\kleeneM) &= \varnothing,\\
        L(r) \cap L(a\kleeneM) &= \varnothing,\\
        L(r) \cap L(b\kleeneM) &= \varnothing.
      \end{align*}
      If all these four equations are true, they imply
        \(L(r) \cap L(s) = \varnothing\).
      \begin{enumerate}

        \item Any word in \(L(r)\) ends with \(a\) whereas any word in
          \(L(\lparen ab \rparen\kleeneM)\) finishes with \(b\) or is
          \(\epsilon\). Thus \(L(r) \cap L(\lparen ab \rparen\kleeneM)
          = \varnothing\).

        \item For the same reason, \(L(r) \cap L(b\kleeneM) =
          \varnothing\).
 
        \item Any word in \(L(r)\) contains both \(a\) and \(b\)
          whereas any word in \(L(a\kleeneM)\) contains only \(b\) or
          is \(\epsilon\). Therefore \(L(r) \cap L(a\kleeneM) =
          \varnothing\).

        \item Any word in \(L(r)\) starts with \(a\) whereas any word
          in \(L(\lparen ba \rparen\kleeneM)\) starts with \(b\) or is
          \(\epsilon\). Thus \(L(r) \cap L(\lparen ba \rparen\kleeneM)
          = \varnothing\).

      \end{enumerate}
      Finally, since all the four equations are false, they imply
      that
      \begin{equation*}
        L(r) \cap L(s) = \varnothing.
      \end{equation*}

    \item Let us construct letter by letter a word \(x\) which does
      not belong neither to \(L(r)\) not \(L(s)\). First, we note that
      all words in \(L(r)\) start with \(a\), so we can try to start
      \(x\) with \(b\): this way \(x \not\in L(r)\). So we have \(x =
      b\ldots\) and we have to fill the dots with some letters in such
      a way that \(x \not\in L(s)\).

      We use again the decomposition of~\(L(s)\) into four
      sub-languages and make sure that~\(x\) does not belong to any of
      those sub-languages. First, because \(x\) starts with \(b\), we
      have \(x \not\in L(a\kleeneM)\) and \(x \not\in L(\lparen
      ab \rparen\kleeneM)\). Now, we have to add some more letters
      such that \(x \not\in L(b\kleeneM)\) and \(x \not\in L(\lparen
      ba \rparen\kleeneM)\). Since any word in \(L(b\kleeneM)\) has a
      letter \(b\) as second letter or is \(\epsilon\), we can choose
      the second letter of \(x\) to be \(a\). This
      way \(x=ba\ldots \not\in L(b\kleeneM)\). Finally, we have to
      add more letters to make sure that
      \begin{equation*}
      x=ba\ldots \not\in L(\lparen ba\rparen\kleeneM).
      \end{equation*}
      Any word in \(L(\lparen ba\rparen\kleeneM)\) is
      either \(\epsilon\) or \(ba\) or \(baba\ldots\), hence the third
      letter is \(b\). Therefore, let us choose the letter \(a\) as
      the third letter of \(x\) and we thus have \(x=baa \not\in
      L(\lparen ba\rparen\kleeneM)\). In summary, \(baa \not\in L(r),
      baa \not\in L(b\kleeneM), baa \not\in L(\lparen
      ba\rparen\kleeneM), baa \not\in L(a\kleeneM), baa \not\in
      L(\lparen ab\rparen\kleeneM)\), which is equivalent
      to \(baa \not\in L(r)\) and \(baa \not\in L(\lparen
      ab \rparen\kleeneM) \cup L(\lparen ba \rparen\kleeneM) \cup
      L(a\kleeneM) \cup L(b\kleeneM) = L(s)\). Therefore, \(x=baa\) is
      one possible answer.

\end{enumerate}
