\section{Queueing}
\label{sec:queueing}
\index{queue}

Despite its didactic qualities, aggregate analysis (see
page~\pageref{par:aggregate}) is less frequently applied when the data
structures are not directly connected to numeration. We propose to
extend its scope by showing a compelling case study on
\emph{functional queues}
\citep{Burton_1982,Okasaki_1995,Okasaki_1998b}. A functional queue is
a linear data structure that is used in functional languages, whose
semantics force the programmer to model a \emph{queue} with two
stacks. Items can be pushed only on one stack and popped only on the
other:
\begin{equation*}
\begin{array}{@{}l@{}r@{\;}cc|c|c|c|c|c|cc@{\;}l@{}}
\cline{4-9}
& \text{Push, Pop (top)} 
& \leftrightsquigarrow & & a & b & c & d & e &&&\\
\cline{4-9}
\end{array}
\end{equation*}
A queue is like a stack where items are added, or \emph{enqueued}, at
one end, called \emph{rear}, but taken out, or \emph{dequeued}, at the
other end, called \emph{front}:
\begin{equation*}
\begin{array}{@{}l@{}r@{\;}cc|c|c|c|c|c|cc@{\;}l@{}}
\cline{4-10}
& \text{Enqueue (rear end)} 
                & \rightsquigarrow & & a & b & c & d & e &
& \rightsquigarrow & \text{Dequeue (front end).}\\
\cline{4-10}
\end{array}
\end{equation*}
Let us implement a queue with two stacks: one for enqueueing, called
the \emph{rear stack}\index{queue!rear stack}\index{stack!rear
  $\sim$|see{queue}}, and one for dequeueing, called the \emph{front
  stack}\index{queue!front stack}\index{stack!front
  $\sim$|see{queue}}. The previous ideal queue is equivalent to the
functional queue
\begin{equation*}
\begin{array}{r@{\;}cc|c|c|c|c|c|c|cc@{\;}l}
  \cline{3-6}\cline{8-10}
  \text{Enqueue (rear)} & \rightsquigarrow & & a & b & c & & d & e & &
  \rightsquigarrow & \text{Dequeue (front).}\\
  \cline{3-6}\cline{8-10}
\end{array}
\end{equation*}
Enqueueing is now pushing on the rear stack and dequeueing is popping
on the front stack. In the latter case, if the front stack is empty
and the rear stack is not, we swap the stacks and reverse the new
front stack. Graphically, dequeueing in the configuration
\begin{equation*}
\begin{array}{c|c|c|c|c|c} \cline{1-4}\cline{6-6} & a & b & c &
&\\ \cline{1-4}\cline{6-6}
\end{array}
\end{equation*}
requires first to make
\begin{equation*}
\begin{array}{c|c|c|c|c|c}
  \cline{1-1}\cline{3-6}
  & & a & b & c &\\
  \cline{1-1}\cline{3-6}
\end{array}
\end{equation*}
and then dequeue~\(c\).

Let us model a queue as we modelled the operation push by a function
\fun{cons/2}\index{cons@\fun{cons/2}} without a definition, on
page~\pageref{par:stacks}. We shall use the name
\fun{q/2}\index{q@\fun{q/2}} and the call \(\fun{q}(r,f)\) denotes a
functional queue with rear stack~\(r\) and front
stack~\(f\). Enqueueing is performed by the function
\fun{enq/2}\index{enq@\fun{enq/2}}:
\begin{equation}
\fun{enq}(x,\fun{q}(r,f)) \rightarrow \fun{q}(\cons{x}{r},f).
\label{def:enq}
\end{equation}
Dequeueing requires the result to be a \emph{pair}\index{functional
  language!pair} made of the dequeued item and the new queue without
it. Actually, the new queue is the first component of the pair, to fit
how the operation is depicted in the figures as a rightwards arrow. We
could denote the mathematical pair \((x,y)\) by \(\fun{pair}(x,y)\),
but a more symbolic notation is handy: \(\pair{x}{y}\). The angle
brackets allow us to avoid writing \(f((x,y))\) when we mean
\(f(\pair{x}{y})\). Let us call \fun{deq/1}\index{deq@\fun{deq/1}} the
dequeueing function:
\begin{equation}
\begin{array}{@{}r@{\;}l@{\;}l@{}}
  \fun{deq}(\fun{q}(\cons{x}{r},\el))
& \xrightarrow{\smash{\theta}}
& \fun{deq}(\fun{q}(\el,\fun{rcat}(r,[x])));\\
  \fun{deq}(\fun{q}(r,\cons{x}{f}))
& \xrightarrow{\smash{\iota}}
& \pair{\fun{q}(r,f)}{x}.
\end{array}
\label{def:deq}
\end{equation}
See page~\pageref{def:rev} for the definition~\eqref{def:rev}
of~\fun{rcat/2}\index{rcat@\fun{rcat/2}}. We shall say that the queue
has size~\(n\) if the total number of items in both stacks
is~\(n\). The cost of enqueueing\index{enq@$\C{\fun{enq}}{n}$} is
\(\C{\fun{enq}}{n} = 1\). The minimum
cost\index{deq@$\B{\fun{deq}}{n}$} for dequeueing is
\(\B{\fun{deq}}{n} = 1\), by rule~\(\iota\). The maximum cost is
\(\W{\fun{deq}}{n} = n+2\), as seen with the trace
\(\theta\eta^{n-1}\zeta\iota\). Let~\(\A{n}\) be the cost of a
sequence of \(n\)~updates on a functional queue originally empty. A
first attempt at assessing~\(\A{n}\) consists in ignoring any
dependence on previous operations and take the maximum individual
cost. Since \(\C{\fun{enq}}{k} \leqslant \C{\fun{deq}}{k}\), we
consider a series of \(n\)~dequeueings in their worst case, that is,
with all the items located in the rear stack. Besides, after
\(k\)~updates, there may be at most \(k\)~items in the queue, so
\begin{equation*}
\A{n} \leqslant \sum_{k=1}^{n-1}{\W{\fun{deq}}{k}} =
\frac{1}{2}{(n-1)(n+4)} \sim \frac{1}{2}{n^2}.
\end{equation*}

\paragraph{Aggregate analysis}
\index{cost!amortised $\sim$}
\index{queue!amortised cost}

Actually, this is overly pessimistic and even unrealistic. First, one
cannot dequeue on an empty queue, therefore, at any time, the number
of enqueueings since the beginning is always greater or equal than the
number of dequeueings and the series must start with one
enqueueing. Second, when dequeueing with the front being empty, the
rear stack is reversed onto the front stack, so its items cannot be
reversed again during the next dequeueing, whose cost will
be~\(1\). Moreover, as remarked above, \(\C{\fun{enq}}{k} \leqslant
\C{\fun{deq}}{k}\), so the worst case for a series of
\(n\)~operations occurs when the number of dequeueings is maximum,
that is, when it is \(\lfloor{n/2}\rfloor\). If we denote by~\(e\) the
number of enqueueings and by~\(d\) the number of dequeueings, we have
the relationship \(n = e + d\) and the two requisites for a worst case
become \(e=d\) (\(n\)~even) or \(e=d+1\) (\(n\)~odd). The former
corresponds graphically to a \emph{Dyck path}\index{Dyck path} and the
latter to a \emph{Dyck meander}\index{Dyck meander}.

\paragraph{Dyck path}
\index{Dyck path}

Let us depict updates as in \fig~\vref{fig:enq_deq}.
\begin{figure}
\centering
\subfloat[Enqueue\label{fig:enqueue}]{%
  \includegraphics[bb=69 662 132 721,scale=0.75]{enqueue}
}
\qquad
\subfloat[Dequeue\label{fig:dequeue}]{%
  \includegraphics[bb=69 662 132 721,scale=0.75]{dequeue}
}
\caption{Graphical representations of operations on
  queues\label{fig:enq_deq}}
\end{figure}
Textually, we represent an enqueueing as an opening parenthesis and a
dequeueing as a closing parenthesis. For example,
\texttt{((()()(()))())} can be represented in \fig~\ref{fig:dyck_path}
\begin{figure}
\centering
\includegraphics[bb=75 570 508 727,scale=0.7]{dyck_path}
\caption{Dyck path modelling queue operations (cost \(24\))
\label{fig:dyck_path}}
\end{figure}
as a Dyck path\index{Dyck path}. For a broken line to qualify
as a Dyck path of length~\(n\), it has to start at the origin
\((0,0)\) and end at coordinates \((n,0)\). In terms of a \emph{Dyck
  language}, an enqueueing is called a \emph{rise}\index{Dyck
  path!rise} and a dequeueing is called a \emph{fall}\index{Dyck
  path!fall}. A rise followed by a fall, that is, \texttt{()}, is
called a \emph{peak}\index{Dyck path!peak}. For instance, in
\fig~\ref{fig:dyck_path}, there are four peaks. The number near each
rise or fall is the cost incurred by the corresponding operation. The
abscissa axis bears the ordinal of each operation.

When \(e=d\), the line is a Dyck path of length \(n=2e=2d\). In order
to deduce the total cost in this case, we must find a
\emph{decomposition}\index{Dyck path!decomposition} of the path, by
which we mean to identify patterns whose costs are easy to calculate
and which make up any path, or to associate any path to another path
whose cost is the same but easy to find. \Fig~\ref{fig:dyck_eq}
\begin{figure}[!b]
\centering
\includegraphics[bb=75 570 508 727,scale=0.7]{dyck_eq}
\caption{Dyck path equivalent to \fig~\ref{fig:dyck_path}
\label{fig:dyck_eq}}
\end{figure}
shows how the previous path is mapped to an equivalent path only made
of a series of isosceles triangles whose bases belong to the abscissa
axis. Let us call them \emph{mountains}\index{Dyck path!mountain} and
their series a \emph{range}\index{Dyck path!range}. The mapping is
simple: after the first series of falls, if we are back to the
abscissa axis, we have a mountain and we proceed with the rest of the
path. Otherwise, the next operation is a rise and we exchange it with
the first fall after it. This brings us down by~\(1\) and the process
resumes until the abscissas are reached. We call this méthod
\emph{rescheduling}\index{Dyck path!rescheduling} because it amounts,
in operational terms, to reordering subsequences of operations a
posteriori.

For instance, \fig~\vref{fig:rescheduling}
\begin{figure}
\centering
\subfloat[Initial\label{fig:initial}]{%
  \includegraphics[scale=0.75,bb=48 588 218 726]{mount0}}
\qquad
\subfloat[Swapping \(4\nearrow 5\) and \(5\searrow 6\)]{%
  \includegraphics[scale=0.75,bb=48 588 218 726]{mount1}}\\
\subfloat[Swapping \(5\nearrow 6\) and \(8 \searrow 9\)\label{fig:valley}]{%
  \includegraphics[scale=0.75,bb=48 588 218 726]{mount2}}
\qquad
\subfloat[Last one]{%
  \includegraphics[bb=13 645 180 782,scale=0.75]{mount3}}
\caption{Rescheduling of \fig~\ref{fig:dyck_path} into
  \fig~\ref{fig:dyck_eq}\label{fig:rescheduling}}
\end{figure}
displays the rescheduling of \fig~\vref{fig:dyck_path}. Note that two
different paths can be rescheduled into the same path. What makes
\fig~\ref{fig:valley} equivalent to \fig~\ref{fig:initial} is the
invariance of the cost because all operations have cost~\(1\). This
always holds because enqueueings always have cost~\(1\) and the
dequeueings involved in a rescheduling have cost~\(1\), because they
found the front stack non\hyp{}empty after a peak. We proved that all
paths are equivalent to a range with the same cost, therefore, the
maximum cost can be found on ranges alone.

Let us note \(e_1, e_2, \dots, e_k\) the maximal subsequences of
rises; for example, in \fig~\ref{fig:dyck_eq}, we have \(e_1=3\),
\(e_2 = 3\) and \(e_3 = 1\). Of course, \(e = e_1 + e_2 + \dots +
e_k\). The fall making up the \(i\)th peak incurs the
cost\index{deq@$\W{\fun{deq}}{n}$} \(\W{\fun{deq}}{e_i} = e_i + 2\),
due to the front being empty because we started the rises from the
abscissa axis. The next \(e_i-1\) falls have all cost~\(1\), because
the front is not empty. For the \(i\)th mountain, the cost is thus
\(e_i+(e_i+2)+(e_i-1) = 3e_i+1\). Then \(\A{e,k} =
\sum_{i=1}^{k}{(3e_i+1)} = 3e + k\). The maximum cost is obtained by
maximising \(\A{e,k}\) for a given~\(e\):
\begin{equation*}
\W{}{e,e} := \max_{1 \leqslant k \leqslant e}{\A{e,k}} = \A{e,e} = 4e
= 2n,\,\; \text{with \(n=e+d=2e\)},
\end{equation*}
where \(\W{}{e,e}\) is the maximum cost when there are
\(e\)~enqueueings and \({d=e}\) dequeueings. In other words, the worst
case when \(e=d=7\) is the saw\hyp{}toothed Dyck path shown in
\fig~\vref{fig:max_dyck}. 
\begin{figure}
\centering
\includegraphics[bb=86 657 507 726,scale=0.7]{max_dyck}
\caption{Worst case when \(e=d=7\) (cost \(28\))
\label{fig:max_dyck}}
\end{figure}
Importantly, there are no other Dyck paths whose rescheduling lead to
this worst case and the reason is that the reverse transformation from
ranges to general Dyck paths works on dequeueings of cost~\(1\) and
the solution we found is the only one with no dequeueing equal
to~\(1\).

\paragraph{Dyck meander}
\index{Dyck meander}

Another worst case occurs if \(e = d + 1\) and the line is then a Dyck
meander whose extremity ends at ordinate \(e-d=1\). An example is
given in \fig~\ref{fig:meander1},
\begin{figure}[b]
\centering
\includegraphics[bb=75 570 480 727,scale=0.82]{meander1}
\caption{Dyck meander modelling queue operations (total cost
  \(21\))\label{fig:meander1}}
\end{figure}
where the last operation is a dequeueing. The dotted line delineates
the result of applying the rescheduling we used on Dyck paths. Here,
the last operation becomes an enqueueing.

Another possibility is shown in \fig~\ref{fig:meander2},
\begin{figure}[t]
\centering
\includegraphics[bb=75 570 480 727,scale=0.82]{meander2}
\caption{Dyck meander modelling queue operations (total cost
  \(23\))\label{fig:meander2}}
\end{figure}
where the last operation is left unchanged. The difference between the
two examples lies in the fact that the original last dequeueing has,
in the former case, a cost of~\(1\) (thus is changed) and, in the
latter case, a cost greater than~\(1\) (thus is invariant). The third
kind of Dyck meander is one ending with an enqueueing, but because
this enqueueing must start from the abscissa axis, this is the same
situation as the result of rescheduling a meander ending with a
dequeueing with cost~\(1\) (see dotted line in \fig~\ref{fig:meander1}
again). Therefore, we are left to compare the results of rescheduling
meanders ending with a dequeueing, that is, we envisage two cases.
\begin{itemize}

  \item If we have a range of \(n-1\)~operations followed by an
    enqueueing, the maximum cost of the range is the cost of a
    saw\hyp{}toothed Dyck path, that is,
    \(\W{}{e-1,e-1}=4(e-1)=2n-2\), because \(n=e+d=2e-1\), followed by
    an enqueueing, totalling \(2n-1\).

  \item Otherwise, we have a range of \(n-3\)~operations followed by
    two rises and one fall (of cost~\(6\)). The cost is
    \(\W{}{e-2,e-2}+6=2n\), which is marginally greater than the
    previous case.

\end{itemize}

\mypar{Amortised cost}
\index{cost!amortised $\sim$}
\index{queue!amortised cost}

The cost~\(\A{n}\) of a series of \(n\)~queue updates, starting on an
empty queue, is tightly bounded as\index{queue@$\A{n}$}
\begin{equation*}
n \leqslant \A{n} \leqslant 2n,
\end{equation*}
where the lower bound is tight if all updates are enqueueings and the
upper bound when a saw\hyp{}toothed range is followed by one
enqueueing or else two enqueueings and one dequeueing. By definition,
the amortised cost of one operation is~\(\A{n}/n\) and lies between
\(1\)~and~\(2\), which is less than our first analysis (\(\sim
n/2\)). We published a slightly different analysis on the same
examples \citep{Rinderknecht_2011}.

\paragraph{Side note}

We can gain some more abstraction by using a dedicated constructor for
the empty queue, \(\fun{nilq/0}\), and changing accordingly the
definition of \fun{enq/2} in~\eqref{def:enq} \vpageref{def:enq} so it
handles this case:
\begin{equation*}
\fun{enq}(x,\fun{nilq}()) \rightarrow \fun{q}([x],\el);
\quad
\fun{enq}(x,\fun{q}(r,f)) \rightarrow \fun{q}(\cons{x}{r},f).
\end{equation*}
We can improve this a little by pushing~\(x\) directly into the front
stack:
\begin{equation*}
\fun{enq}(x,\fun{nilq}()) \rightarrow \fun{q}(\el,[x]);
\quad
\fun{enq}(x,\fun{q}(r,f)) \rightarrow \fun{q}(\cons{x}{r},f).
\end{equation*}

\paragraph{Exercises}
\begin{enumerate}

  \item Let \(\fun{nxt}(q)\)\index{nxt@\fun{nxt/1}} be the next item
    to be dequeued from~\(q\):
    \begin{equation*}
      \begin{array}{r@{\;}l@{\;}l}
        \fun{nxt}(\fun{q}(\cons{x}{r},\el)) & \rightarrow 
        & \fun{nxt}(\fun{q}(\el,\fun{rcat}(r,[x])));\\
        \fun{nxt}(\fun{q}(r,\cons{x}{f})) & \rightarrow & x.     
        \index{q@\fun{q/2}}
      \end{array}
    \end{equation*}
    Modify \fun{enq/2}\index{enq@\fun{enq/2}},
    \fun{deq/1}\index{deq@\fun{deq/1}} and
    \fun{nxt/1}\index{nxt@\fun{nxt/1}} in such a way that
    \(\C{\fun{nxt}}{n} = 1\)\index{nxt@$\C{\fun{nxt}}{n}$}, where
    \(n\)~is the number of items in the queue.

    \item Find~\(\A{n}\) using the slightly different definition
      \begin{equation*}
        \begin{array}{@{}r@{\;}l@{\;}l@{}}
          \fun{deq}(\fun{q}(\cons{x}{r},\el))
          & \rightarrow
          & \fun{deq}(\fun{q}(\el,\fun{rev}(\cons{x}{r})));\\
          \fun{deq}(\fun{q}(r,\cons{x}{f}))
          & \rightarrow
          & \pair{\fun{q}(r,f)}{x}.
        \end{array}
      \end{equation*}

\end{enumerate}
