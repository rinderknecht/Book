\section{Deterministic finite automata}
 
Transition diagrams are useful graphical representations of instances
of the mathematical concept of \emph{deterministic finite automaton}
(\emph{DFA}). Formally, a DFA \(\mathcal{D}\) is a 5-tuple
\(\mathcal{D} = (Q, \Sigma, \delta, q_0, F)\) where
\begin{enumerate*}

  \item a finite set of \emph{states}, often noted \(Q\);

  \item an \emph{initial state} \(q_0 \in Q\);

  \item a set of \emph{final (\emph{or} accepting)
  states} \(F \subseteq Q\);

  \item a finite set of \emph{input symbols}, often noted \(\Sigma\);

  \item a \emph{transition function} \(\delta\) that takes a state and
    an input symbol and returns a state: if~\(q\) is a state with an
    edge labelled \(a\), the edge leads to the state \(\delta(q, a)\).

\end{enumerate*}

\subsection*{Recognised words}

Independently of the interpretation of the states, we can define how a
given word is accepted (or recognised) or rejected by a given DFA. For
example, the word \(a_1 a_2 \cdots a_n\), with \(a_i \in \Sigma\), is
recognised by the DFA \(\mathcal{D} = (Q, \Sigma, \delta, q_0, F)\)
if, for all \(0 \leqslant i \leqslant n-1\), there is a sequence of
states \(q_i \in Q\) such that \(\delta (q_i, a_{i+1}) = q_{i+1}\) and
\(q_n \in F\). The language recognised by \(\mathcal{D}\), noted
\(L(\mathcal{D})\) is the set of words recognised
by~\(\mathcal{D}\). For example, consider the DFA in
\fig~\vref{fig:trie_then}.
\begin{figure}[b]
\centering
\includegraphics[bb=48 644 218 730,scale=0.93]{trie_then}
\caption{A trie recognising \textsf{they}, \textsf{then},
  \textsf{this} and \textsf{thus} \label{fig:trie_then}}
\end{figure}
The word \textsf{then} is recognised because there is a sequence of
states \((q_0, q_1, q_2, q_4, q_5)\) connected by edges which
satisfies \(\delta (q_0, \textsf{t}) = q_1, \delta (q_1, \textsf{h}) =
q_2, \delta (q_2, \textsf{e}) = q_4 \,\text{and}\, \delta (q_4,
\textsf{n}) = q_5\), with \(q_5 \in F\), that is, \(q_5\)~is a final
state.

\subsection*{Recognised language}

It is easy to define formally \(L(\mathcal{D})\). Let \(\mathcal{D} =
(Q, \Sigma, \delta, q_0, F)\). First, let us extend~\(\delta\) to
words and let us call this extension~\(\hat{\delta}\):
\begin{itemize*}

  \item for all state \(q \in Q\), let \(\hat{\delta} (q, \varepsilon)
    = q\), where \(\varepsilon\) is the empty string;

  \item for all state \(q \in Q\), all word \(w \in \Sigma^{*}\), all
    input \(a \in \Sigma\), let us define \(\hat{\delta} (q, wa) =
    \delta (\hat{\delta}(q,w),a)\).

\end{itemize*}
Then the word \(w\) is recognised by \(\mathcal{D}\) if
\(\hat{\delta}(q_0, w) \in F\). The language \(L(\mathcal{D})\)
recognised by~\(\mathcal{D}\) is defined as \(L(\mathcal{D}) = \{w \in
\Sigma^{*} \; \lvert \; \hat{\delta}(q_0, w) \in F\}\). For example,
in our last example:
\begin{equation*}
\begin{aligned}
   \hat{\delta}(q_0, \epsilon)
&= q_0,\\
   \hat{\delta}(q_0, \textsf{t})
&= \delta (\hat{\delta}(q_0, \epsilon), \textsf{t})
= \delta (q_0, \textsf{t})
= q_1,\\
   \hat{\delta}(q_0, \textsf{th})
&= \delta (\hat{\delta}(q_0, \textsf{t}), \textsf{h})
= \delta (q_1, \textsf{h})
= q_2,\\
   \hat{\delta}(q_0, \textsf{the})
&= \delta (\hat{\delta}(q_0, \textsf{th}), \textsf{e})
= \delta (q_2, \textsf{e})
= q_4,\\
   \hat{\delta}(q_0, \textsf{then}) 
&= \delta (\hat{\delta}(q_0, \textsf{the}), \textsf{n})
= \delta (q_4, \textsf{n})
= q_5 \in F.
\end{aligned}
\end{equation*}

\subsection*{Transition diagrams}
 
We can also redefine transition diagrams in terms of the concept of
DFA. A transition diagram for a DFA \(\mathcal{D} = (Q, \Sigma,
\delta, q_0, F)\) is a graph defined as follows:
\begin{enumerate*}

  \item for each state \(q\) in \(Q\) there is a \emph{node}, \emph{i.e.,} a
    single circle with \(q\) inside;

  \item for each state \(q \in Q\) and each input symbol \(a \in
    \Sigma\), if \(\delta (q, a)\) exists, then there is an
    \emph{edge}, \emph{i.e.,} an arrow, from the node denoting~\(q\) to the
    node denoting \(\delta (q, a)\) labelled by \(a\); multiple edges
    can be merged into one and the labels are then separated by
    commas;

  \item there is an edge coming to the node denoting~\(q_0\) without
    origin;

  \item nodes corresponding to final states are doubly circled.

\end{enumerate*}
Here is a transition diagram for the language over alphabet \(\{0,
1\}\), called \emph{binary alphabet}, which contains the
string~\(01\):
\begin{center}
\includegraphics[bb=48 710 185 760]{dfa_01}
\end{center} 

\subsection*{Transition table}

There is a compact textual way to represent the transition function of
a DFA: a \emph{transition table}. The rows of the table correspond to
the states and the columns correspond to the inputs (symbols). In
other words, the entry for the row corresponding to state~\(q\) and
the column corresponding to input~\(a\) is the state~\(\delta (q,
a)\), as seen in \fig~\ref{fig:gen_table}.
\begin{figure}
\centering
\subfloat[General table\label{fig:gen_table}]{
\(\begin{array}{c||c|c|c}
\delta & \ldots & a & \ldots\\
\hhline{=::===}
\vdots & & &\\
\hline
q & & \delta (q, a)\\
\hline
\vdots & & &
\end{array}
\)}
\qquad
\subfloat[Example\label{fig:example_table}]{
\(\begin{array}{r@{}l||c|c}
\multicolumn{2}{c||}{\mathcal{D}} & 0 & 1\\
\hhline{==::==}
\rightarrow & q_0 & q_1 & q_0\\
            & q_1 & q_1 & q_2\\
         \# & q_2 & q_2 & q_2\\
            &     &     &
\end{array}\)
}
\caption{Transition tables}
\end{figure}
For instance, the transition table corresponding to the
function~\(\delta\) of our last example is found in
\fig~\vref{fig:example_table}. Actually, we added some extra
information in the table: the initial state is marked with
\(\rightarrow\) and the final states are marked
with~\(\#\). Therefore, it is not only~\(\delta\) which is defined by
means of the transition table here, but the whole DFA~\(\mathcal{D}\).

Let us consider another example. We want to define formally a DFA
which recognises the language \(L\) whose words contain an even number
of 0's and an even number of 1's (the alphabet is binary). We should
understand that the role of the states here is to \emph{not} count the
exact number of 0's and 1's that have been recognised before, but,
instead, this number \emph{modulo~2}. Therefore, there are four states
because there are four cases:
\begin{enumerate*}

  \item there has been an even number of 0's and 1's (state \(q_0\));

  \item there has been an even number of 0's and an odd number of 1's
    (state \(q_1\));

  \item there has been an odd number of 0's and an even number of 1's
    (state \(q_2\));

  \item there has been an odd number of 0's and 1's (state \(q_3\)).

\end{enumerate*}
What about the initial and final states?
\begin{itemize*}

  \item State~\(q_0\) is the initial state because before considering
  any input, the number of 0's and 1's is zero and zero is even;

  \item state~\(q_0\) is the lone final state because its definition
  matches exactly the characteristic of~\(L\) and no other state
  matches.

\end{itemize*}
We almost know now how to specify the DFA for the language~\(L\). It
is
\begin{equation*}
\mathcal{D} = (\{q_0,q_1,q_2,q_3\}, \{0,1\}, \delta,q_0, \{q_0\}),
\end{equation*}
where the transition function~\(\delta\) is described by the
transition diagram in \fig~\vref{fig:dfa_even01}.
\begin{figure}
\centering
\subfloat[The DFA\label{fig:dfa_even01}]{
\includegraphics[bb=48 653 138 737]{dfa_even01}
}
\qquad
\subfloat[The table\label{fig:table_even01}]{
\includegraphics{table_even01}
}
\caption{A deterministic finite automaton and its table}
\end{figure}
Notice how each input~0 causes the state to cross the horizontal
line. Thus, after seeing an even number of 0's we are always above the
horizontal line, in state~\(q_0\) or~\(q_1\), and after seeing an odd
number of 0's we are always below this line, in state~\(q_2\)
or~\(q_3\). There is a vertically symmetric situation for transitions
on~1. We can also represent this DFA by the transition table in
\fig~\vref{fig:table_even01}. We can use that table to illustrate the
construction of~\(\hat{\delta}\) from~\(\delta\). Suppose the input is
\(110101\). Since this string has even numbers of 0's and 1's, it
belongs to~\(L\), that is, we expect \(\hat{\delta}(q_0,110101) =
q_0\), since \(q_0\)~is the sole final state. We can check this by
computing step by step \(\hat{\delta}(q_0,\verb+110101+)\), from the
shortest prefix to the longest, which is the word \verb+110101+
itself:
\begin{align*}
  \hat{\delta} (q_0, \varepsilon) 
&= q_0,\\
   \hat{\delta} (q_0, \texttt{1}) 
&= \delta (\hat{\delta} (q_0, \varepsilon), \texttt{1})
= \delta (q_0, \texttt{1})
= q_1,\\
   \hat{\delta} (q_0, \texttt{11}) 
&= \delta (\hat{\delta} (q_0, \texttt{1}), \texttt{1}) 
= \delta (q_1, \texttt{1}) 
= q_0,\\
   \hat{\delta} (q_0, \texttt{110}) 
&= \delta (\hat{\delta} (q_0, \texttt{11}), \texttt{0}) 
= \delta (q_0, \texttt{0}) 
= q_2,\\
   \hat{\delta} (q_0, \texttt{1101}) 
&= \delta (\hat{\delta} (q_0, \texttt{110}), \texttt{1}) 
= \delta (q_2, \texttt{1}) 
= q_3,\\
   \hat{\delta} (q_0, \texttt{11010}) 
&= \delta (\hat{\delta} (q_0, \texttt{1101}), \texttt{0}) 
= \delta (q_3, \texttt{0}) 
= q_1,\\
   \hat{\delta} (q_0, \texttt{110101}) 
&= \delta (\hat{\delta} (q_0, \texttt{11010}), \texttt{1}) 
= \delta (q_1, \texttt{1}) 
= q_0 \in F.
\end{align*}
