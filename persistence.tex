\section{Persistence}
\label{sec:persistence}

\emph{Persistence}\index{persistence} is a distinctive feature of
purely functional languages, meaning that values are
constants. Functions update a data structure by creating a new version
of it, instead of modifying it in place and thereby erasing its
history. We saw in section~\vref{sec:functional} that subtrees common
to both sides of the same rule are shared. Such a sharing is sound
because of persistence: there is no logical way to tell apart a copied
subtree from the original.

\paragraph{Maximum sharing}
\index{sharing}

An obvious source of sharing is the occurrence of a variable in the
pattern of a rule and its right\hyp{}hand side, as seen in
\fig~\vref{fig:cat_dag} for instance. But this does not always lead to
maximum sharing as the definition of
\fun{red/1}\index{red@\fun{red/1}} (\emph{reduce}) in
\fig~\ref{fig:red} shows. This function copies a stack without its
consecutively repeated items. For example, we might have \(\fun{red}([4,1,2,2,2,1,1]) \twoheadrightarrow [4,1,2,1]\).
% Wrapping figure better declared before a paragraph
%
\setlength{\intextsep}{0pt}
\begin{wrapfigure}[6]{r}[0pt]{0pt}
% [6] vertical lines
% {r} mandatory right placement (better because of a list)
\centering
\includegraphics[bb=71 672 203 726]{red}
\caption{Reducing\label{fig:red}}
\end{wrapfigure}
A
directed acyclic graph of the second rule is depicted in
\fig~\vref{fig:red_dag1}.
\begin{figure}
\centering
\subfloat[Sharing variables\label{fig:red_dag1}]{%
  \includegraphics[bb=70 645 158 723]{red_dag1}
}
\qquad\qquad
\subfloat[Sharing subtrees\label{fig:red_dag2}]{%
  \includegraphics[bb=70 645 158 723]{red_dag2}
}
\caption{Directed acyclic graphs (DAG) of the second rule of
  \fun{red/1}}
\end{figure}
In this picture, sharing is based on commonly occurring variables, but
we can see that \(\cons{x}{s}\) is not completely shared. Consider the
same rule in \fig~\vref{fig:red_dag2} with maximum sharing, where a
whole subtree is shared.
\setlength{\intextsep}{12pt}

When discussing memory management, \emph{we assume that sharing is
  maximum for each rule}, so, for instance, \fig~\vref{fig:red_dag2}
would be the default. But this property is not enough to insure that
sharing is maximum between the arguments of a function call and its
value. For example,
\begin{equation*}
\fun{cp}(\el) \rightarrow \el;\qquad
\fun{cp}(\cons{x}{s}) \rightarrow \cons{x}{\fun{cp}(s)}.
\end{equation*}
makes a copy of its argument, but the value of \(\fun{cp}(s)\) only
shares its items with~\(s\), despite \(\fun{cp}(s) \equiv s\).

\paragraph{Revision control}

A simple idea to implement data structures enabling backtracking
consists in keeping all successive
versions\index{persistence!version-based}. A stack can be used to keep
such a record, called \emph{history}\index{persistence!history}, and
backtracking is reduced to linear search\index{linear search}. For
example, we may be interested in recording a series of stacks, each
one obtained from its predecessor by a push or a pop, like the history
\([[4,2,1],[2,1],[3,2,1],[2,1],[1],\el]\), where the initial stack was
empty; next, \(1\)~was pushed, followed by~\(2\) and~\(3\); then
\(3\)~was popped and \(4\)~pushed instead. This way, the \emph{last
  version} is the top of history, like \([4,2,1]\) in our
example. Furthermore, we want two successive versions to share as much
structure as possible. (We speak of `version' instead of `state',
because the latter applies when values are not persistent.)  These
requirements are at the heart of \emph{revision control} software,
used by programmers to keep track of the evolution of their source
code.

As a simple case study, we shall continue with our example of
recording the different versions of a stack, while keeping in mind
that the technique described further applies to any data structure. We
need to write two functions, \fun{push/2}\index{push@\fun{push/2}}
(different from the one defined in section~\ref{sec:cutting}) and
\fun{pop/1}\index{pop@\fun{pop/1}}, which, instead of processing one
stack, deal with a history of stacks. Consider the following
definitions:
\begin{equation}
\begin{array}{@{}r@{\;}c@{\;}l@{\quad}r@{\;}c@{\;}l@{}}
\fun{push}(x,\el) & \rightarrow & [[x],\el];
& \fun{pop}(\cons{\cons{x}{s}}{h}) & \rightarrow &
                                     \cons{s,\cons{x}{s}}{h}.\\
\fun{push}(x,\cons{s}{h}) & \rightarrow &
\cons{\cons{x}{s},s}{h}. & 
\fun{top}(\cons{\cons{x}{s}}{h}) & \rightarrow & x.
\end{array}
\label{eq:push_pop_persistence}
\end{equation}
The corresponding directed acyclic graphs are found in
\fig~\vref{fig:push_pop_dag}.
\begin{figure}
\centering
\includegraphics[bb=70 639 294 723]{push_pop_dag}
\caption{DAGs of \fun{push/2} and \fun{pop/1} with maximum sharing
\label{fig:push_pop_dag}}
\end{figure}
History \([[4,2,1],[2,1],[3,2,1],[2,1],[1],\el]\) is displayed in
\fig~\ref{fig:history}
\begin{figure}[!b]
\centering
\includegraphics[bb=70 575 247 723]{history}
\caption{History \([[4,2,1],[2,1],[3,2,1],[2,1],[1],\el]\)
\label{fig:history}}
\end{figure}
as a directed acyclic graph as well. It is obtained as the evaluation
of
\begin{equation}
%\abovedisplayskip=4pt
%\belowdisplayskip=4pt
\fun{push}(4, \fun{pop}(\fun{push}(3, \fun{push}(2,
\fun{push}(1, \el))))).
\label{ex:history}
\end{equation}

Let \(\fun{ver}(k,h)\) \index{ver@\fun{ver/2}} evaluate in the~\(k\)th
previous version in the history~\(h\), so \(\fun{ver}(0,h)\) is the
last version. As expected, \fun{ver/2} is just a linear search with a
countdown:
\begin{equation*}
%\abovedisplayskip=4pt
%\belowdisplayskip=4pt
\fun{ver}(0,\cons{s}{h}) \rightarrow s;\qquad
\fun{ver}(k,\cons{s}{h}) \rightarrow \fun{ver}(k-1,h).
\end{equation*}
Our encoding of history allows the last version to be easily modified,
but not older ones. When a data structure allows every version in its
history to be modified, it is said \emph{fully
  persistent}\index{persistence!full $\sim$}; if only the last version
can be replaced, it is said \emph{partially persistent}
\citep{MehlhornTsakalidis_1990}.\index{persistence!partial $\sim$}

\paragraph{Backtracking updates}
\index{persistence!backtracking}

In order to achieve full persistence, we should keep a history of
updates\index{persistence!update-based},
\(\ufun{push}(x)\)\index{push@\fun{push/1}} and
\(\ufun{pop}()\)\index{pop@\fun{pop/0}}, instead of successive,
maximally overlapping versions. In the following, the underlining
avoids confusion with the functions
\fun{push/2}\index{push@\fun{push/2}} and
\fun{pop/1}\index{pop@\fun{pop/1}}. Instead of
equation~\eqref{ex:history}, we have now the history
\begin{equation}
[\ufun{push}(4), \ufun{pop}(), \ufun{push}(3), \ufun{push}(2),
\ufun{push}(1)].
\label{ex:history1}
\end{equation}
But not all series of \(\ufun{push}(x)\) and \(\ufun{pop}()\) are
valid, as \([\ufun{pop}(), \ufun{pop}(), \ufun{push}(x)]\) and
\([\ufun{pop}()]\). In order to characterise valid histories, let us
consider a graphical representation of updates in
\fig~\vref{fig:push_pop1}.
\begin{figure}
\centering
\subfloat[$\protect\fun{push}(x)$]{%
  \includegraphics[bb=71 662 130 721,scale=0.8]{push}}
\qquad
\subfloat[$\protect\fun{pop}()$]{%
  \includegraphics[bb=71 662 130 721,scale=0.8]{pop}}
\caption{Stack updates\label{fig:push_pop1}}
\end{figure}
This is the same model we used in section~\ref{sec:queueing} where we
studied queueing (see in particular \fig~\vref{fig:enq_deq}), except
that we choose here a leftwards orientation to mirror the notation for
stacks, whose tops are laid out on the left of the page. Consider for
instance the history in \fig~\vref{fig:history1}.
\begin{figure}[b]
\centering
\includegraphics[bb=69 577 300 726,scale=0.9]{history1}
\caption{$[\ufun{pop}(), \ufun{pop}(), \ufun{push}(4),\ufun{push}(3),
      \ufun{pop}(), \ufun{push}(2), \ufun{push}(1)]$
\label{fig:history1}}
\end{figure}
It is clear that \emph{a valid history is a line which never
  crosses the absissa axis.}

Programming \fun{top/1}\index{top@\fun{top/1}} with a history of
updates is more difficult because we must find the top item of the
last version without constructing it. The idea is to walk back the
historical line and determine the last push leading to a version whose
length is equal to that of the last version. In
\fig~\ref{fig:history1}, the last version is~(\(\bullet\)). If we draw
a horizontal line from it rightwards, the first push ending on the
line is \(\ufun{push}(1)\), so the top of the last version
is~\(1\).

This thought experiment is depicted in \fig~\ref{fig:topmost}.
\begin{figure}
\centering
\includegraphics[bb=69 577 300 726,scale=0.9]{topmost}
\caption{Finding the top of the last version
\label{fig:topmost}}
\end{figure}
Notice that we do not need to determine the length of the last
version: the difference of length with the current version, while
walking back, is enough.

Let \fun{top\(_0\)/1}\index{top0@\fun{top\(_0\)/1}} and
\fun{pop\(_0\)/1}\index{pop0@\fun{pop\(_0\)/1}} be the equivalent of
\fun{top/1} and \fun{pop/1} operating on updates instead of versions.
Their definitions are found in \fig~\vref{fig:pop_top}.
\begin{figure}[h]
\begin{equation*}
\boxed{%
\begin{array}{@{}r@{\;}l@{\;}lr@{\;}l@{\;}l@{}}
\fun{pop}_0(h)   & \rightarrow          & \cons{\ufun{pop}()}{h}. &
\fun{top}_0(0,\cons{\ufun{push}(x)}{h}) & \rightarrow             & x;\\
\fun{top}_0(h)   & \rightarrow          & \fun{top}_0(0,h).       &
\fun{top}_0(k,\cons{\ufun{push}(x)}{h}) & \rightarrow             & \fun{top}_0(k-1,h);\\
                 & & &
\fun{top}_0(k,\cons{\ufun{pop}()}{h})   & \rightarrow            & \fun{top}_0(k+1,h).
\end{array}}
\end{equation*}
\caption{The top and the rest of an history of updates
\label{fig:pop_top}}
\end{figure}
An ancillary function \fun{top\(_0\)/2}\index{top0@\fun{top\(_0\)/2}}
keeps track of the difference between the lengths of the last and
current versions. We have found the item when the difference is~\(0\)
and the current update is a push.

As previously, we want the call
\(\fun{ver}_0(k,h)\)\index{ver0@\fun{ver\(_0\)/2}} to evaluate in
the~\(k\)th previous version in~\(h\). Here, we need to walk back
\(k\)~updates in the past,
\begin{equation*}
\fun{ver}_0(0,h)           \rightarrow \fun{lst}_0(h);\qquad
\fun{ver}_0(k,\cons{s}{h}) \rightarrow \fun{ver}_0(k-1,h).
\end{equation*}
and build the last version from the rest of the history with
\fun{lst\(_0\)/1}\index{lst0@\fun{lst\(_0\)/1}}:
\begin{mathpar}
\inferrule*{}{\fun{lst}_0(\el) \rightarrow \el};
\quad
\inferrule*{}{\fun{lst}_0(\cons{\ufun{push}(x)}{h})
              \rightarrow \cons{x}{\fun{lst}_0(h)}};
\quad
\inferrule
  {\fun{lst}_0(h)                      \twoheadrightarrow \cons{x}{s}}
  {\fun{lst}_0(\cons{\ufun{pop}()}{h}) \twoheadrightarrow s}.
\end{mathpar}
Let \(\C{\fun{ver}_0}{k,n}\)~be the cost of the call
\(\fun{ver}_0(k,h)\) and \(\C{\fun{lst}_0}{n}\)~the cost
of~\(\fun{lst}_0(h)\), where \(n\)~is the length of~\(h\):
\(\C{\fun{lst}_0}{i} = i + 1\) and \(\C{\fun{ver}_0}{k,n} = (k+1) +
\C{\fun{lst}_0}{n-k} = n + 2\).

What is the total amount of memory
allocated?\index{sharing}\index{memory|see{sharing}} More precisely,
we want to know the number of pushes performed. The only rule of
\fun{lst\(_0\)/1} featuring a push in its right\hyp{}hand side is the
second one, so the number of cons\hyp{}nodes is the number of push
updates. But this is a waste in certain cases, for example, when the
version built is empty, like the history
\([\ufun{pop}(),\ufun{push}(6)]\). The optimal situation is to
allocate only as much as the computed version actually contains.

% Wrapping figure better declared before a paragraph
%
\setlength{\intextsep}{0pt}
\begin{wrapfigure}[]{r}[0pt]{0pt}
% {r} mandatory right placement
\centering
\includegraphics[bb=71 644 257 721]{lst1}
\caption{Last version \label{fig:lst1}}
\end{wrapfigure}
\hspace*{-5.6pt} We can achieve this memory optimality with
\fun{lst\(_1\)/1} in \fig~\ref{fig:lst1} by retaining features of both
\fun{top\(_0\)/1}\index{top0@\fun{top\(_0\)/1}} and \fun{lst\(_0\)/1}.
We have\index{lst1@$\C{\fun{lst}_1}{n}$} \(\C{\fun{lst}_1}{n} =
\C{\fun{lst}_0}{n} = n + 1\), and the number of
cons\hyp{}nodes\index{sharing|(} created is now the length of the last
version itself. This is another instance of an output\hyp{}dependent
cost\index{cost!output-dependent $\sim$}, like
\fun{flat\(_0\)/1}\index{flat0@\fun{flat$_0$/1}} in
section~\vref{sec:flattening}. There is still room for improvement if
the historical line meets the absissa axis, as there is no need to
visit the updates \emph{before} a pop resulting into an empty version;
for instance, in \fig~\vref{fig:back2zero},
\begin{figure}[b]
\centering
\includegraphics[bb=69 606 300 726,scale=0.9]{back2zero}
\caption{Last version $[3]$ found in four steps\label{fig:back2zero}}
\end{figure}
it is useless to go past \(\ufun{push}(3)\)\index{push@\fun{push/1}}
to find the last version to be~\([3]\). But, in order to detect
whether the historical line meets the abscissa axis, we need to
augment the history~\(h\) with the length~\(n\) of the last version,
that is, work with \(\pair{n}{h}\), and modify
\fun{push/2}\index{push@\fun{push/2}}
and~\fun{pop/1}\index{pop@\fun{pop/1}} accordingly:
\begin{equation*}
\fun{push}_2(x,\pair{n}{h}) \! \rightarrow \!
\pair{n\!+\!1}{\cons{\ufun{push}(x)}{h}}.\quad
\fun{pop}_2(\pair{n}{h}) \! \rightarrow \!
\pair{n\!-\!1}{\cons{\ufun{pop}()}{h}}.
\end{equation*}
We have to rewrite~\fun{ver/2}\index{ver2@\fun{ver\(_2\)/2}} so it keeps
track of the length of the last version:
\begin{equation*}
\begin{array}{@{}r@{\;}l@{\;}l@{}}
\fun{ver}_2(0,\pair{n}{h}) & \rightarrow & \fun{lst}_1(0,h);\\
\fun{ver}_2(k,\pair{n}{\cons{\ufun{pop}()}{h}})
                      & \rightarrow & \fun{ver}_2(k-1,\pair{n+1}{h});\\
\fun{ver}_2(k,\pair{n}{\cons{\ufun{push}(x)}{h}})
                      & \rightarrow & \fun{ver}_2(k-1,\pair{n-1}{h}).
\end{array}
\end{equation*}
We can reduce the memory\index{sharing|)} footprint by separating the
length~\(n\) of the last version from the current history~\(h\), so no
pairs are allocated, and we can stop when the current version
is~\(\el\), as planned, in \fig~\ref{fig:ver_no_pair}.
\begin{figure}
\begin{equation*}
\boxed{%
\begin{array}{r@{\;}l@{\;}l}
\fun{ver}_3(k,\pair{n}{h}) & \rightarrow & \fun{ver}_3(k,n,h).\\
\fun{ver}_3(0,n,h) & \rightarrow & \fun{lst}_3(0,n,h);\\
\fun{ver}_3(k,n,\cons{\ufun{pop}()}{h})
                      & \rightarrow & \fun{ver}_3(k-1,n+1,h);\\
\fun{ver}_3(k,n,\cons{\ufun{push}(x)}{h})
                      & \rightarrow & \fun{ver}_3(k-1,n-1,h).\\
\fun{lst}_3(\pair{n}{h}) & \rightarrow & \fun{lst}_3(0,n,h).\\
\fun{lst}_3(k,0,h) & \rightarrow & \el;\\
\fun{lst}_3(0,n,\cons{\ufun{push}(x)}{h}) & \rightarrow
                      & \cons{x}{\fun{lst}_3(0,n-1,h)};\\
\fun{lst}_3(k,n,\cons{\ufun{push}(x)}{h}) & \rightarrow
                      & \fun{lst}_3(k-1,n-1,h);\\
\fun{lst}_3(k,n,\cons{\ufun{pop}()}{h}) & \rightarrow
                      & \fun{lst}_3(k+1,n+1,h).
\end{array}}
\end{equation*}
\caption{Querying a version without pairs\label{fig:ver_no_pair}}
\end{figure}

One may wonder why bother pairing~\(n\) and~\(h\) just to separate
them again, defeating the purpose of data abstraction. This example
demonstrates that abstraction is desirable for callers, but called
functions may break it due to pattern matching. We may also realise
that the choice of a stack for storing updates is not the best in
terms of memory usage. Instead, we can directly chain updates with an
additional argument denoting the previous update, so, for instance,
instead of equation~\eqref{ex:history1}:
\begin{equation*}
\pair{3}{\ufun{push}(4, \ufun{pop}(\ufun{push}(3, \ufun{push}(2, \ufun{push}(1,\el)))))}.
\end{equation*}
This new encoding\index{stack!encoding with tuples} closely mirrors
the call~\eqref{ex:history} on page~\pageref{ex:history} and saves
\(n\)~cons\hyp{}nodes in a history of length~\(n\). See
\fig~\ref{fig:ver}.
\begin{equation*}
\fun{push}_4(x,\pair{n}{h}) \rightarrow
\pair{n\!+\!1}{\ufun{push}(x,h)}.\quad
\fun{pop}_4(\pair{n}{h}) \rightarrow \pair{n\!-\!1}{\ufun{pop}(h)}.
\end{equation*}
\begin{figure}[t]
\begin{equation*}
\boxed{%
\begin{array}{r@{\;}l@{\;}l}
\fun{ver}_4(k,\pair{n}{h}) & \rightarrow & \fun{ver}_4(k,n,h).\\
\fun{ver}_4(0,n,h) & \rightarrow & \fun{lst}_4(0,n,h);\\
\fun{ver}_4(k,n,\ufun{pop}(h))
                      & \rightarrow & \fun{ver}_4(k-1,n+1,h);\\
\fun{ver}_4(k,n,\ufun{push}(x,h))
                      & \rightarrow & \fun{ver}_4(k-1,n-1,h).\\
\fun{lst}_4(\pair{n}{h}) & \rightarrow & \fun{lst}_4(0,n,h).\\
\fun{lst}_4(k,0,h) & \rightarrow & \el;\\
\fun{lst}_4(0,n,\ufun{push}(x,h)) & \rightarrow
                      & \cons{x}{\fun{lst}_4(0,n-1,h)};\\
\fun{lst}_4(k,n,\ufun{push}(x,h)) & \rightarrow
                      & \fun{lst}_4(k-1,n-1,h);\\
\fun{lst}_4(k,n,\ufun{pop}(h)) & \rightarrow 
                      & \fun{lst}_4(k+1,n+1,h).
\end{array}}
\end{equation*}
\caption{Querying a version without a stack\label{fig:ver}}
\end{figure}

There is now a minimum and maximum cost. The worst case is when the
bottommost item of the last version is the first item pushed in the
history, so \fun{lst\(_4\)/3}\index{lst4@\fun{lst\(_4\)/3}} has to
recur until the origin of time. In other words, the historical line
never reaches the abscissa axis after the first push. We have same
cost as before: \(\W{\fun{lst}_4}{n} = n +
1\). \index{lst4@$\W{\fun{lst}_4}{n}$} The best case happens when the
last version is empty. In this case, \(\B{\fun{lst}_4}{n} =
1\)\index{lst4@$\B{\fun{lst}_4}{n}$}, and this is an occurrence of the
kind of improvement we sought.

%\mypar{Average cost}

\paragraph{Full persistence}
\index{persistence!full $\sim$}

The update\hyp{}based approach to history is fully persistent by
enabling the modification of the past as follows: traverse history
until the required moment, pop up the update at that point, push
another one and simply put back the previously traversed updates,
which must have been kept in some accumulator. But changing the past
must not create a history with a non\hyp{}constructible version in it,
to wit, the historical line must not cross the absissa axis after the
modification. If the change consists in replacing a pop by a push,
there is no need to worry, as this will raise by~\(2\) the end point
of the line. It is the converse change that requires special
attention, as this will lower by~\(2\) the end point. The \(\pm
2\)~offset comes from the vertical difference between the end points
of a push and a pop update of same origin, as can be easily seen in
\fig~\vref{fig:push_pop1}. As a consequence, in
\fig~\ref{fig:history1}, the last version has length~\(1\), which
implies that it is impossible to replace a push by a pop, anywhere in
the past.

Let us consider the history in \fig~\vref{fig:history2}.
\begin{figure}
\centering
\includegraphics[bb=69 578 272 726,scale=0.9]{history2}%[]
\caption{$\ufun{pop}(\ufun{push}(4,\ufun{push}(3,
          \ufun{pop}(\ufun{push}(2,\ufun{push}(1,\el))))))$
\label{fig:history2}}
\end{figure}
Let \(\fun{chg}(k,u,\pair{n}{h})\)\index{chg@\fun{chg/3}} be the
changed history, where \(k\)~is the index of the update we want to
change, indexing the last one at~\(0\); \(u\)~is the new update we
want to set; \(n\)~is the length of the last version of
history~\(h\). The call
\begin{equation*}
\fun{chg}(3,\ufun{push}(5),
  \pair{2}{\ufun{pop}(\ufun{push}(4,\ufun{push}(3,
           \ufun{pop}(\ufun{push}(2,\fun{push}(1,\el))))))})
\end{equation*}
results in
\(\pair{4}{\ufun{pop}(\ufun{push}(4,\ufun{push}(3,\ufun{push}(5,
  \ufun{push}(2,\ufun{push}(1,\el))))))}\). \!\!This call succeeds
because in \fig~\ref{fig:pop2push}
\begin{figure}
\centering
\subfloat[A pop becomes a push\label{fig:pop2push}]{
  \includegraphics[bb=69 521 270 726,scale=0.8]{pop2push}}
\quad
\subfloat[A push becomes a pop\label{fig:push2pop}]{
  \includegraphics[bb=69 577 270 726,scale=0.8]{push2pop}}
\caption{Changing updates}
\end{figure}
the new historical line does not cross the absissa axis. We can see in
\fig~\ref{fig:push2pop} the result of the call
\begin{equation*}
\fun{chg}(2,\ufun{pop}(),
            \pair{2}{\ufun{pop}(\ufun{push}(4,\ufun{push}(3,
                     \ufun{pop}(\ufun{push}(2,\ufun{push}(1,\el))))))}).
\end{equation*}
All these examples help in guessing the characteristic property for a
replacement to be valid:
\begin{itemize}

  \item the replacements of a pop by a push, a pop by a pop, a push by
    a push are always valid;

  \item the replacement of a push by a pop as update~\(k>0\) is valid
    if and only if the historical line between updates
    \(0\)~and~\(k-1\) remains above or reaches without crossing the
    horizontal line of ordinate~\(2\).

\end{itemize}
We can divide the algorithm in two phases. First, the update to be
replaced must be found, but, the difference with \fun{ver\(_4\)/3} is
that we may need to know if the historical line before reaching the
update lies above the horizontal line of ordinate~\(2\). This is easy
to check if we maintain across recursive calls the lowest ordinate
reached by the line. The second phase substitute the update and checks
if the resulting history is valid.

Let us implement the first phase. First, we project \(n\)~and~\(h\)
out of \(\pair{n}{h}\) in order to save some memory, and the lowest
ordinate is~\(n\), which we pass as an additional argument to another
function \fun{chg/5}\index{chg@\fun{chg/5}|(}:
\begin{equation*}
\fun{chg}(k,u,\pair{n}{h}) \rightarrow \fun{chg}(k,u,n,h,n).
\end{equation*}
Function~\fun{chg/5} traverses~\(h\) while decrementing~\(k\), until
\(k=0\), which means that the update to be changed has been found. At
the same time, the length of the current version is computed (third
argument) and compared to the previous lowest ordinate (the fifth
argument), which is updated according to the outcome. We may try the
following canvas:
\begin{equation*}
\begin{array}{@{}r@{\;}l@{\;}l@{}}
\fun{chg}(0,u,n,h,m) & \rightarrow & \fbcode{CCCCCCC}\,;\\
\fun{chg}(k,u,n,\ufun{pop}(h),m) & \rightarrow
                                 & \fun{chg}(k-1,u,n+1,h,m);\\
\fun{chg}(k,u,n,\ufun{push}(x,h),m) & \rightarrow 
                   & \fun{chg}(k-1,u,n-1,h,m),\qquad \text{if \(m < n\)};\\
\fun{chg}(k,u,n,\ufun{push}(x,h),m) & \rightarrow
                                    & \fun{chg}(k-1,u,n-1,h,n-1).
\end{array}
\end{equation*}
\index{chg@\fun{chg/5}|)} The problem is that we forget the history
down to the update we are looking for. There are two methods to record
it: either we use an accumulator\index{functional
  language!accumulator} and stick with a definition in tail
form\index{functional language!tail form} (small\hyp{}step
design\index{design!small-step $\sim$}), or we put back a visited
update after returning from a recursive call (big\hyp{}step
design\index{design!big-step $\sim$}). The latter is faster, as there
is no need to reverse the accumulator when we are done; the former
allows us to share the history up to the update when it leaves history
structurally invariant, at the cost of an extra parameter being the
original history. This same dilemma was encountered when we compared
\fun{sfst/2}\index{sfst@\fun{sfst/2}} and
\fun{sfst\(_0\)/2}\index{sfst0@\fun{sfst\(_0\)/2}} in
section~\ref{sec:skipping} on page~\pageref{sec:skipping}. Let us opt
for a big\hyp{}step design in \fig~\ref{fig:chg}.
\begin{figure}[b]
\centering
\framebox[\columnwidth]{\vbox{%
\begin{gather*}
\inferrule*{}{\fun{chg}(0,u,n,h,m) \rightarrow \fun{rep}(u,h,m)}
\;\;
\inferrule
  {\fun{chg}(k-1,u,n+1,h,m) \twoheadrightarrow \pair{n'}{h'}}
  {\fun{chg}(k,u,n,\ufun{pop}(h),m) \twoheadrightarrow 
                                    \pair{n'}{\ufun{pop}(h')}}
\\
\inferrule
  {\fun{chg}(k-1,u,n-1,h,m) \twoheadrightarrow \pair{n'}{h'}\\
   m < n}
  {\fun{chg}(k,u,n,\ufun{push}(x,h),m) \twoheadrightarrow
                              \pair{n'}{\ufun{push}(x,h')}}
\\
\inferrule
  {\fun{chg}(k-1,u,n-1,h,n-1) \twoheadrightarrow \pair{n'}{h'}}
  {\fun{chg}(k,u,n,\ufun{push}(x,h),m) \twoheadrightarrow 
                               \pair{n'}{\ufun{push}(x,h')}}
\end{gather*}
}}
\caption{Changing a past version (big-step design)\label{fig:chg}}
\end{figure}
Notice how the length~\(n'\) of the new history is simply passed down
the inference rules. It can simply be made out:
\begin{itemize*}

  \item replacing a pop by a pop or a push by a push leaves the
    original length invariant;

  \item replacing a pop by a push increases the original length
    by~\(2\);

  \item replacing a push by a pop, assuming this is valid, decreases
    the original length by~\(2\).

\end{itemize*}
This task is up to the new
function~\fun{rep/3}\index{rep@\fun{rep/3}|(} (\emph{replace}), which
implements the second phase. The design is to return a pair made of
the differential in length~\(d\) and the new history~\(h'\):
\begin{equation*}
\begin{array}{@{}r@{\;}l@{\;}l@{}}
\fun{rep}(\ufun{pop}(),\ufun{pop}(h),m)
     & \rightarrow & \pair{0}{\ufun{pop}(h)};\\
\fun{rep}(\ufun{push}(x),\ufun{push}(y,h),m)
     & \rightarrow & \pair{0}{\ufun{push}(x,h)};\\
\fun{rep}(\ufun{push}(x),\ufun{pop}(h),m)
     & \rightarrow & \pair{2}{\ufun{push}(x,h)};\\
\fun{rep}(\ufun{pop}(),\ufun{push}(y,h),m)
     & \rightarrow & \pair{-2}{\ufun{pop}(h)}.
\end{array}
\end{equation*}
\index{rep@\fun{rep/3}|)} This definition implies that we need to
redefine~\fun{chg/3}\index{chg@\fun{chg/3}} as follows:
\begin{mathpar}
\inferrule{\fun{chg}(k,u,n,h,n) \twoheadrightarrow \pair{d}{h'}}
          {\fun{chg}(k,u,\pair{n}{h}) \twoheadrightarrow \pair{n+d}{h'}}.
\end{mathpar}
