\section{Regular expressions}

In \Pascal, an identifier is a letter followed by zero or more letters
or digits, that is, and identifier is a member of the set defined by
\(L(L \cup D)^{*}\). The notation we introduced so far is comfortable
for mathematics but not for computers. Let us introduce another
notation, called \emph{regular expressions}, for describing the same
languages and define its meaning in terms of the mathematical
notation. With this notation, we might define \Pascal identifiers as
\begin{center}
\term{letter} \lparen\term{letter} \disj \term{digit}\rparen\kleene
\end{center}
where the vertical bar means `or', the parentheses group
subexpressions, the star means `zero or more instances of' the
previous expression and juxtaposition means concatenation. A regular
expression~\(r\) is built up out of simpler regular expressions using
a set of rules, as follows. Let~\(\Sigma\) be an alphabet and \(L(r)\)
the language denoted by~\(r\). Then
\begin{enumerate*}

   \item \(\epsilon\) is a regular expression that denotes
     \(\{\varepsilon\}\);\label{regexp_empty}

   \item if \(a \in \Sigma\), then \(a\)~is a regular expression that
     denotes~\(\{a\}\). This is ambiguous: \(a\)~can denote a
     language, a word or a letter --~it depends on the
     context; \label{regexp_sym}

   \item assume \(r\)~and~\(s\) denote the languages \(L(r)\)~and
     \(L(s)\); \(a\)~denotes a letter. Then \label{regexp_rec}
   \begin{enumerate*}
    
     \item \(r\) \disj \(s\) is a regular expression
     denoting \(L(r) \cup L(s)\);

     \item \(r s\) is a regular expression denoting \(L(r) L(s)\);

     \item \(r\)\kleene{} is a regular expression
     denoting \((L(r))^{*}\);

     \item \lparen\(r\)\rparen{} is a regular expression
     denoting \(L(r)\);

     \item \(\overline{a}\) is a regular expression denoting
       \(\Sigma\backslash \{a\}\).

   \end{enumerate*}

\end{enumerate*}
A language described by a regular expression is a \emph{regular
  language}. Rules~\ref{regexp_empty} and~\ref{regexp_sym} form the
base of the definition. Rule~\ref{regexp_rec} provides the inductive
step. Unnecessary parentheses can be avoided in regular expressions if
\begin{itemize*}

  \item the unary operator \kleene{} has the highest precedence and
  is left associative,

  \item concatenation has the second highest precedence and is left
  associative,

  \item \disj{} has the lowest precedence and is left associative.

\end{itemize*}
Under those conventions, \lparen\(a\)\rparen{} \disj
\lparen\lparen\(b\)\rparen\kleene\lparen\(c\)\rparen\rparen{} is
equivalent to \(a\) \disj \(b\)\kleene\(c\). Both expressions denote
the language containing either the string \(a\) or zero or more
\(b\)'s followed by one \(c\): \(\{a, c, bc, bbc, bbbc, \dots\}\). For
example,
\begin{itemize*}

  \item the regular expression \(a\) \disj \(b\) denotes the set
    \(\{a, b\}\);

  \item the regular expression \lparen\(a\) \disj
    \(b\)\rparen\lparen\(a\) \disj \(b\)\rparen{} denotes \(\{aa, ab,
    ba, bb\}\), the set of all strings of \(a\)'s and \(b\)'s of
    length two. Another regular expression for the set is \(aa\) \disj
    \(ab\) \disj \(ba\) \disj \(bb\);

  \item the regular expression \(a\)\kleene{} denotes the set of all
    strings of zero or more \(a\)'s, i.e. \(\{\varepsilon, a, aa, aaa,
    \dots\}\);

  \item the regular expression \lparen\(a\) \disj
    \(b\)\rparen\kleene{} denotes the set of all strings containing
    zero of more instances of an \(a\) or \(b\), that is the language
    of all words made of \(a\)'s and \(b\)'s. Another expression is
    \lparen\(a\)\kleene\(b\)\kleene\rparen\kleene.

\end{itemize*}
If two regular expressions \(r\)~and~\(s\) denote the same language,
we say that \(r\)~and~\(s\) are \emph{equivalent} and write \(r =
s\). In \fig~\vref{fig:laws},
\begin{figure}
\centering
\begin{tabular}{c|l}
\toprule
  \multicolumn{1}{c}{\textsc{Law}}
& \multicolumn{1}{c}{\textsc{Description}}\\
\midrule
  \(r\) \disj \(s\) = \(s\) \disj \(r\)
& \disj is commutative\\
\hline
  \(r\) \disj \lparen\(s\) \disj \(t\)\rparen{}
  = \lparen\(r\) \disj \(s\)\rparen{} \disj \(t\)
& \disj is associative\\
\hline
  \lparen\(rs\)\rparen \(t\) = \(r\)\lparen\(st\)\rparen
& concatenation is associative\\
\hline
  \(r\)\lparen\(s\) \disj \(t\)\rparen{} = \(rs\) \disj \(rt\)
& concatenation distributes over \disj\\
  \lparen\(s\) \disj \(t\)\rparen \(r\) = \(sr\) \disj \(tr\)
&\\
\hline
  \(\epsilon r = r\) 
& \(\epsilon\) is the identity element\\
  \(r \epsilon = r\)
& for the concatenation\\
\hline
  \(r\)\kleene\kleene = \(r\)\kleene
& Kleene closure is idempotent\\
\hline
  \(r\)\kleene = \(r\)\plus \disj \(\epsilon\)
& Kleene closure and positive closure\\
  \(r\)\plus = \(r r\)\kleene
& are closely linked\\
\bottomrule
\end{tabular}
\caption{Algebraic laws on regular languages\label{fig:laws}}
\end{figure}
we show useful algebraic laws on regular languages.

\subsection*{Regular definitions}

It is convenient to give names to regular expressions and define new
regular expressions using these names as if they were symbols. If
\(\Sigma\) is an alphabet, then a \emph{regular definition} is a
series of definitions of the form
\begin{align*}
    d_1 &\rightarrow r_1\\
    d_2 &\rightarrow r_2\\
    &\cdots\\
    d_n &\rightarrow r_n
\end{align*}
where each \(d_i\) is a distinct name and each \(r_i\) is a regular
expression over the alphabet \(\Sigma \cup \{d_1, d_2, \dots,
d_{i-1}\}\), \emph{i.e.,} the basic symbols and the previously defined
names. The restriction to \(d_j\) such \(j < i\) allows to construct a
regular expression over \(\Sigma\) only by repeatedly replacing all
the names in it. For instance, as we have stated, the set of \Pascal
identifiers can be defined by the regular definitions
\begin{align*}
\term{letter} & \rightarrow \text{\exc{A} \disj \exc{B} \disj
  \ldots \disj \exc{Z} \disj \exc{a} \disj \exc{b} \disj \ldots \disj
  \exc{z}} \\
\term{digit} & \rightarrow \text{\exc{0} \disj \exc{1} \disj \exc{2}
  \disj \exc{3} \disj \exc{4} \disj \exc{5} \disj \exc{6} \disj
  \exc{7} \disj \exc{8} \disj \exc{9}}\\
\term{id} & \rightarrow \text{\term{letter} \lparen\term{letter}
  \disj \term{digit}\rparen\kleene}
\intertext{Unsigned numbers in \Pascal are strings like
\texttt{5280}, \texttt{39.37}, \texttt{6.336E4}
or \texttt{1.894E-4}.}
\term{digit} & \rightarrow \text{ \exc{0} \disj \exc{1} \disj \exc{2}
  \disj \exc{3} \disj \exc{4} \disj \exc{5} \disj \exc{6} \disj
  \exc{7} \disj \exc{8} \disj \exc{9}}\\
\term{digits} & \rightarrow \text{\term{digit} \term{digit}\kleene}\\
\term{optional\_fraction} & \rightarrow \text{\exc{.} \term{digits}
  \disj} \, \epsilon\\
\term{optional\_exponent} & \rightarrow \text{\lparen\exc{E} \lparen
  \exc{+} \disj \exc{-} \disj} \, \epsilon \, \text{\rparen{}
  \term{digits}\rparen{} \disj} \, \epsilon\\
\term{num} & \rightarrow \text{\term{digits} \term{optional\_fraction}
  \term{optional\_exponent}}
\end{align*}
Certain constructs occur so frequently in regular expressions that it
is convenient to introduce notational shorthands for them:
\begin{itemize}

  \item \emph{Zero or one instance.} The unary operator `\opt{}' means
    `zero or one instance of.' Formally, by definition, if~\(r\) is a
    regular expression then \(r\)\opt = \(r\) \disj \(\epsilon\). In
    other words, \lparen\(r\)\rparen\opt{} denotes the language \(L(r)
    \cup \{\varepsilon\}\).
\begin{align*}
\term{digit} & \rightarrow \text{\exc{0} \disj \exc{1} \disj \exc{2}
  \disj \exc{3} \disj \exc{4} \disj \exc{5} \disj \exc{6} \disj
  \exc{7} \disj \exc{8} \disj \exc{9}}\\
\term{digits} & \rightarrow \text{\term{digit}\plus}\\
\term{optional\_fraction} & \rightarrow \text{\lparen\exc{.}
  \term{digits}\rparen\opt}\\
\term{optional\_exponent} & \rightarrow \text{\lparen\exc{E} \lparen
  \exc{+} \disj \exc{-}\rparen\opt{} \term{digits}\rparen\opt}\\
\term{num} & \rightarrow \text{\term{digits} \term{optional\_fraction}
  \term{optional\_exponent}}
\end{align*}

 \item It is also possible to write:
\begin{align*}
\term{digit} & \rightarrow \text{\exc{0} \disj \exc{1} \disj \exc{2}
  \disj \exc{3} \disj \exc{4} \disj \exc{5} \disj \exc{6} \disj
  \exc{7} \disj \exc{8} \disj \exc{9}}\\
\term{digits} & \rightarrow \text{\term{digit}\plus}\\
\term{fraction} & \rightarrow \text{\exc{.} \term{digits}}\\
\term{exponent} & \rightarrow \text{\exc{E} \lparen \exc{+} \disj
  \exc{-}\rparen\opt{} \term{digits}}\\
\term{num} & \rightarrow \text{\term{digits} \term{fraction}\opt{}
  \term{exponent}\opt}
\end{align*}

\end{itemize}
If we want to specify the characters `\texttt{?}', `\texttt{*}',
`\texttt{+}', `\texttt{|}', we write them with a preceding backslash,
\emph{e.g.,} `\verb+\?+', or between double-quotes, \emph{e.g.,}
\verb+"?"+. Then, of course, the character double-quote must have a
backslash: \verb+\"+. It is also sometimes useful to match against end
of lines and end of files: \verb+\n+ stands for the control character
`end of line' and \term{\$} is for `end of file'.

\subsection*{Non-regular languages}

Some languages cannot be described by any regular expression. For
example, the language of balanced parentheses cannot be recognised by
any regular expression: \lparen\rparen, \lparen\lparen\rparen\rparen,
\lparen\rparen\lparen\rparen,
\lparen\lparen\lparen\rparen\rparen\lparen\rparen\rparen{}
etc. Another example is the C programming language: it is not a
regular language because it contains embedded blocs between `\verb+{+'
and `\verb+}+'. Therefore, a lexer cannot recognise valid C programs:
we need a parser.

\paragraph{Exercises}

\paragraph{Question 1.}

Let the alphabet \(\Sigma = \{a, b\}\) and the following regular
expressions:
\begin{align*}
  r &= a \lparen a \, \disjM{} \, b \rparen\kleeneM{} ba,\\
  s &= \lparen ab \rparen\kleeneM{} \, \disjM{} \, \lparen ba
  \rparen\kleeneM{} \, \disjM{} \, \lparen a\kleeneM{} 
  \, \disjM{} \, b\kleeneM\rparen.
\end{align*}
The language denoted by~\(r\) is noted \(L(r)\) and the language
denoted by~\(s\) is noted \(L(s)\). Find a word~\(x\) such that
\begin{enumerate*}

  \item \(x \in L(r)\) and \(x \not\in L(s)\),

  \item \(x \not\in L(r)\) and \(x \in L(s)\),

  \item \(x \in L(r)\) and \(x \in L(s)\),

  \item \(x \not\in L(r)\) and \(x \not\in L(s)\).

\end{enumerate*}

\paragraph{Answer 1.}

The method to answer these questions is simply to try small words by
constructing them in order to satisfy the constraints.
\begin{enumerate}

  \item The \label{aba} shortest word \(x\) belonging to L(r) is found
    by taking \(\epsilon\) in place of \(\lparen a \disjM{}
    b\rparen\kleeneM\). So \(x = aba\). Let us check if \(x \in L(s)\)
    or not. \(L(s)\) is made of the union of four sub-languages
    (subsets). To make this clear, let us remove the useless
    parentheses on the right side:
    \begin{equation*}
    s = \lparen ab \rparen\kleeneM{} \, \disjM{} \, \lparen ba
    \rparen\kleeneM{} \, \disjM{} \, a\kleeneM{} \, \disjM{} \,
    b\kleeneM.
    \end{equation*}
    Therefore, membership tests on \(L(s)\) have to be split
    into four: one membership test on \(\lparen ab
    \rparen\kleeneM\), one on \(\lparen ba \rparen\kleeneM\),
    one on \(a\kleeneM\) and another one on
    \(b\kleeneM\). In other words, \(x \in L(s)\) is equivalent to
    \begin{equation*}
    x \in L(\lparen ab \rparen\kleeneM) \;
     \text{or} \; x \in L(\lparen ba \rparen\kleeneM) \;
     \text{or} \; x \in L(a\kleeneM) \; \text{or} \; x
     \in L(b\kleeneM).
    \end{equation*}
    Let us test the membership with \(x = aba\):
    \begin{enumerate}
  
      \item The words in \(L(\lparen ab \rparen\kleeneM)\) are
        \(\epsilon\), \(ab\), \(abab\ldots\) Thus \(aba \not\in
        L(\lparen ab \rparen\kleeneM)\).

      \item The words in \(L(\lparen ba \rparen\kleeneM)\) are
        \(\epsilon\), \(ba\), \(baba\ldots\) Hence \(aba \not\in
        L(\lparen ba \rparen\kleeneM)\).

      \item The words in \(L(a\kleeneM)\) are \(\epsilon\), \(a\),
        \(aa\ldots\) Therefore \(aba \not\in L(a\kleeneM)\).

      \item The words in \(L(b\kleeneM)\) are \(\epsilon\), \(b\),
        \(bb\ldots\) So \(aba \not\in L(b\kleeneM)\).
  
    \end{enumerate}
    The conclusion is \(aba \not\in L(s)\).

    \item What is the shortest word belonging to \(L(s)\)?  Since the
      four sub-languages composing \(L(s)\) are starred, it means that
      \(\epsilon \in L(s)\). Since we showed at the item~(\ref{aba})
      that \(aba\) is the shortest word of \(L(r)\), it means that
      \(\epsilon \not\in L(r)\) because \(\epsilon\) is of length
      \(0\).

    \item This question is a bit more difficult. After a few tries, we
      cannot find any~\(x\) such that \(x \in L(r)\) and \(x \in
      L(s)\). Then we may try to prove that \(L(r) \cap L(s) =
      \varnothing\), \emph{i.e.,} there is no such~\(x\). How should we
      proceed? The idea is to use the decomposition of \(L(s)\) into
      for sub-languages and try to prove
      \begin{align*}
        L(r) \cap L(\lparen ab \rparen\kleeneM) &= \varnothing,\\
        L(r) \cap L(\lparen ba \rparen\kleeneM) &= \varnothing,\\
        L(r) \cap L(a\kleeneM) &= \varnothing,\\
        L(r) \cap L(b\kleeneM) &= \varnothing.
      \end{align*}
      If all these four equations are true, they imply
        \(L(r) \cap L(s) = \varnothing\).
      \begin{enumerate}

        \item Any word in \(L(r)\) ends with \(a\) whereas any word in
          \(L(\lparen ab \rparen\kleeneM)\) finishes with \(b\) or is
          \(\epsilon\). Thus \(L(r) \cap L(\lparen ab \rparen\kleeneM)
          = \varnothing\).

        \item For the same reason, \(L(r) \cap L(b\kleeneM) =
          \varnothing\).
 
        \item Any word in \(L(r)\) contains both \(a\) and \(b\)
          whereas any word in \(L(a\kleeneM)\) contains only \(b\) or
          is \(\epsilon\). Therefore \(L(r) \cap L(a\kleeneM) =
          \varnothing\).

        \item Any word in \(L(r)\) starts with \(a\) whereas any word
          in \(L(\lparen ba \rparen\kleeneM)\) starts with \(b\) or is
          \(\epsilon\). Thus \(L(r) \cap L(\lparen ba \rparen\kleeneM)
          = \varnothing\).

      \end{enumerate}
      Finally, since all the four equations are false, they imply
      that
      \begin{equation*}
        L(r) \cap L(s) = \varnothing.
      \end{equation*}

    \item Let us construct letter by letter a word \(x\) which does
      not belong neither to \(L(r)\) not \(L(s)\). First, we note that
      all words in \(L(r)\) start with \(a\), so we can try to start
      \(x\) with \(b\): this way \(x \not\in L(r)\). So we have \(x =
      b\ldots\) and we have to fill the dots with some letters in such
      a way that \(x \not\in L(s)\).

      We use again the decomposition of~\(L(s)\) into four
      sub-languages and make sure that~\(x\) does not belong to any of
      those sub-languages. First, because \(x\) starts with \(b\), we
      have \(x \not\in L(a\kleeneM)\) and \(x \not\in L(\lparen
      ab \rparen\kleeneM)\). Now, we have to add some more letters
      such that \(x \not\in L(b\kleeneM)\) and \(x \not\in L(\lparen
      ba \rparen\kleeneM)\). Since any word in \(L(b\kleeneM)\) has a
      letter \(b\) as second letter or is \(\epsilon\), we can choose
      the second letter of \(x\) to be \(a\). This
      way \(x=ba\ldots \not\in L(b\kleeneM)\). Finally, we have to
      add more letters to make sure that
      \begin{equation*}
      x=ba\ldots \not\in L(\lparen ba\rparen\kleeneM).
      \end{equation*}
      Any word in \(L(\lparen ba\rparen\kleeneM)\) is
      either \(\epsilon\) or \(ba\) or \(baba\ldots\), hence the third
      letter is \(b\). Therefore, let us choose the letter \(a\) as
      the third letter of \(x\) and we thus have \(x=baa \not\in
      L(\lparen ba\rparen\kleeneM)\). In summary, \(baa \not\in L(r),
      baa \not\in L(b\kleeneM), baa \not\in L(\lparen
      ba\rparen\kleeneM), baa \not\in L(a\kleeneM), baa \not\in
      L(\lparen ab\rparen\kleeneM)\), which is equivalent
      to \(baa \not\in L(r)\) and \(baa \not\in L(\lparen
      ab \rparen\kleeneM) \cup L(\lparen ba \rparen\kleeneM) \cup
      L(a\kleeneM) \cup L(b\kleeneM) = L(s)\). Therefore, \(x=baa\) is
      one possible answer.

\end{enumerate}

\paragraph{Question 2.}

Given the binary alphabet \(\Sigma = \{a, b\}\) and the order on
letters \(a < b\), write regular definitions for the following
languages.
\begin{enumerate*}

  \item All words starting and ending with \(a\).

  \item All non-empty words.

  \item All words in which the third last letter is \(a\).

  \item All words containing exactly three \(a\).

  \item All words containing at least one \(a\) before a \(b\).

  \item All words in which the letters are in increasing order.

  \item All words with no letter following the same one.

\end{enumerate*}

\paragraph{Answer 2.}

When answering these questions, it is important to keep in mind that
the language of words made up on the alphabet~\(\Sigma\)
is~\(\Sigma^{*}\) and that there are, in general, several regular
expressions describing one language.

\begin{enumerate}

  \item The constraint on the words is that they must be of the shape
    \(a \ldots a\) where the dots stand for `any combination of \(a\)
    and \(b\).' In other words, one answer is \(a \lparen a \;
    \disjM{} \, b\rparen\kleeneM \, a \, \disjM \, a\).

  \item This question is very simple since the language of all words
    is \(\lparen a \; \disjM{} \, b\rparen\kleeneM\), we have to
    remove \(\epsilon\), \emph{i.e.,} one simple answer is \(\lparen a \;
    \disjM{} \, b\rparen\plusM\).

  \item The question implies that the words we are looking for are of
    the form \(\ldots a \, \_ \, \_\) where the dots stand for `any
    sequence of \(a\) and \(b\)' and each `\_' stands for a regular
    expression denoting any letter. Any letter is described
    by \(\lparen a \, \disjM{} \, b\rparen\); therefore one possible
    answer is \(\lparen a \, \disjM{} \, b\rparen\kleeneM{}
    a \, \lparen a \, \disjM{} \, b\rparen \, \lparen a \, \disjM{} \,
    b\rparen\).

  \item The words we search contain, at any place, exactly three
    \(a\), so are of the form \(\ldots a \ldots a \ldots a \ldots\),
    where the dots stand for `any letter except \(a\)', \emph{i.e.,} `any
    number of \(b\).' In other words: \(b\kleeneM a b\kleeneM a
    b\kleeneM a b\kleeneM\).

  \item Because the alphabet contains only two letters, the question
    is equivalent to: 'All words containing the substring
    ab', \emph{i.e.,} the words are of the form \(\ldots ab \ldots\)
    where the dots stand for `any sequence of \(a\) and \(b\).' It is
    then easy to understand that a short answer is \(\lparen
    a \, \disjM{} \, b\rparen\kleeneM ab \lparen a \, \disjM{} \,
    b\rparen\kleeneM\).

  \item Because the alphabet is made only of two letters, the answer
    is easy: we put first all the \(a\) and then all the \(b\):
    \(a\kleeneM b\kleeneM\).

  \item Since the alphabet contains only two letters, the only way to
    not repeat a letter is to only have substrings \(ab\) or \(ba\) in
    the words we look for. In other words: \(abab\ldots ab\) or
    \(abab\ldots aba\) or \(baba\ldots ba\) or \(baba\ldots bab\). In
    short: \(\lparen ab \rparen\kleeneM a\opt \, \disjM{} \, \lparen
    ba \rparen\kleeneM b\opt\) or, even shorter: \(a\opt \lparen ba
    \rparen\kleeneM b\opt\).

\end{enumerate}



\paragraph{Question 3.}

Try to simplify the regular expressions
\(\lparen \epsilon \, \disjM{} \, a\kleeneM{} \, \disjM{} \,
  b\kleeneM{} \, \disjM{} \, a \, \disjM{} \, b \rparen\kleeneM\) and
\(a \lparen a \, \disjM{} \, b \rparen\kleeneM{} b 
  \, \disjM{} \, \lparen a b \rparen\kleeneM{} \, \disjM{} \, \lparen
  b a \rparen\kleeneM\).

\paragraph{Answer 3.}

\begin{enumerate}

  \item The first regular expression can be simplified in the
    following way:
   \begin{align*}
     \lparen \epsilon \, \disjM{} \, a\kleeneM{} \, \disjM{} \,
     b\kleeneM{} \, \disjM{} \, a \, \disjM{} \, b \rparen\kleeneM
     &= \lparen \epsilon \, \disjM{} \, a\kleeneM{} \, \disjM{} \,
     b\kleeneM{} \, \disjM{} \, b\rparen\kleeneM, & \text{since}
     \, L(a) \subset L(a\kleeneM);\\
     &= \lparen \epsilon \, \disjM{} \, a\kleeneM{} \, \disjM{} \,
     b\kleeneM\rparen\kleeneM, & \text{since} \,
     L(b) \subset L(b\kleeneM);\\
     &= \lparen \epsilon \, \disjM{} \, a\plusM{} \, \disjM{} \,
     b\plusM\rparen\kleeneM, & \text{since} \, \{\epsilon\}
     \subset L(x\kleeneM);\\
     &= \lparen a\plusM{} \, \disjM{} \, b\plusM\rparen\kleeneM, &
     \text{since} \, \lparen\epsilon \, \disjM{} \,
     x\rparen\kleeneM = x\kleeneM.
   \end{align*}
   Words in \(L(\lparen a\plusM{} \, \disjM{} \,
   b\plusM\rparen\kleeneM)\) are of the form \(\epsilon\) or
   \((a\ldots a)\) \((b \ldots b)\) \((a\ldots a)\) \((b\ldots
    b)\ldots\), where the ellipsis stands for `none or many times'.
   So we recognise \(\lparen a \, \disjM{} \,
   b\rparen\kleeneM\). Therefore \(\lparen \epsilon \, \disjM{}
   \, a\kleeneM{} \, \disjM{} \, b\kleeneM{} \, \disjM{} \, a \,
   \disjM{} \, b \rparen\kleeneM = \lparen a \, \disjM{} \,
   b\rparen\kleeneM\).

  \item The second regular expression can be simplified in the
    following way. We note first that the expression is made of the
    disjunction of three regular sub-expressions (\emph{i.e.,} it is a
    union of three sub-languages). The simplest idea is then to check
    whether one of these sub-languages is redundant, \emph{i.e.,} if
    one is included in another. If so, we can simply remove it from
    the expression.
  \begin{align*}
     a \lparen a \, \disjM{} \, b \rparen\kleeneM{} b 
     \, \disjM{} \, \lparen a b \rparen\kleeneM{} 
     \, \disjM{} \, \lparen b a \rparen\kleeneM
     &= a \lparen a \, \disjM{} \, b \rparen\kleeneM{} b 
     \, \disjM{} \, \epsilon
     \, \disjM{} \, \lparen a b \rparen\plusM{} 
     \, \disjM{} \, \lparen b a \rparen\kleeneM,\\
     & \qquad \text{since} \, \lparen ab \rparen\kleeneM =
     \epsilon \, \disjM{}\, \lparen ab \rparen\plusM;\\
     &= a \lparen a \, \disjM{} \, b \rparen\kleeneM{} b 
     \, \disjM{} \, \lparen a b \rparen\plusM{} 
     \, \disjM{} \, \lparen b a \rparen\kleeneM,\\
     & \qquad \text{since} \, \{\epsilon\} \subset L(\lparen ba
     \rparen\kleeneM).
  \end{align*}
  We have:
  \begin{align*}
    \lparen ab\rparen\plusM
   &= \lparen ab\rparen \lparen ab\rparen \ldots \lparen ab \rparen\\
   &= a\lparen ba\rparen\lparen ba\rparen\ldots\lparen  ba\rparen b \;
    \disjM{} \; ab\\
   &= a\lparen ba \rparen\kleeneM b.
   \end{align*}
   Also \(L(\lparen ba \rparen) \subset L(\lparen a \, \disjM{}
   \, b\rparen\kleeneM)\) and then \(L(\lparen ba
   \rparen\kleeneM) \subset L(\lparen a \, \disjM{} \,
   b\rparen\kleeneM)\), because \(\lparen a \, \disjM{} \,
   b\rparen\kleeneM\) denotes all the words. Therefore
   \begin{align*}
      L(a\lparen ba \rparen\kleeneM b) 
    &\subset L(a \lparen a \, \disjM{} \, b\rparen\kleeneM b)\\
      L(\lparen ab\rparen\plusM) 
    &\subset L(a \lparen a \, \disjM{} \, b\rparen\kleeneM b)
   \end{align*}

   As a consequence, one possible answer is
   \begin{equation*}  
     a \lparen a \, \disjM{} \, b \rparen\kleeneM{} b 
     \, \disjM{} \, \lparen a b \rparen\kleeneM{} 
     \, \disjM{} \, \lparen b a \rparen\kleeneM
     = a \lparen a \, \disjM{} \, b\rparen\kleeneM b 
     \, \disjM{} \, \lparen ba \rparen\kleeneM.
   \end{equation*}
   The intersection between \(L(a \lparen a \, \disjM{} \,
   b\rparen\kleeneM b)\) and \(L(\lparen ba \rparen\kleeneM)\)
   is empty because all the words of the former start with
     \(a\), while all the words of the other start with \(b\) (or is
     \(\epsilon\)).  Therefore we cannot simply further this way.

\end{enumerate}

