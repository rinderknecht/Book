\chapter{Word Factoring}
\label{chap:factoring}
\index{exact string matching|see{word fac\-toring}}
\index{string matching|see{word factoring}}
\index{word factoring|(}

Let us call \emph{alphabet}\index{word factoring!alphabet} a
non\hyp{}empty, finite set of symbols, called
\emph{letters}\index{word factoring!letter} and set in a
sans\hyp{}serif type, for example \word{a}, \word{b} etc. A
\emph{word}\index{word factoring!word} is a finite series of letters,
like \word{word}; in particular, a letter is a word, as in English. We
denote repetition of a letter or word with an exponent, for instance,
\(\word{a}^3 = \word{aaa}\). Just as letters can be joined to make up
words, so can words: the word \(u \cdot v\) is made of the letters of
word~\(u\) followed by the letters of word~\(v\), for instance, if
\(u=\word{back}\) and \(v=\word{up}\), then \(u \cdot v =
\word{backup}\). This operation is associative: \((u \cdot v) \cdot w
= u \cdot (v \cdot w)\). As a shorthand, the operator may be omitted:
\((uv)w = u(vw)\). Catenation on words behave like a
non\hyp{}commutative product, so it has a neutral element
\(\varepsilon\), called the \emph{empty word}: \(u \cdot \varepsilon =
\varepsilon \cdot u = u\).

A word~\(x\) is a \emph{factor}\index{word factoring!factor} of a
word~\(y\) if there exists two words \(u\)~and~\(v\) such that \(y =
uxv\). The word~\(x\) is a \emph{prefix}\index{word factoring!prefix}
of~\(y\), noted \(x \prefeq y\), if~\(u=\varepsilon\), that is, if \(y
= xv\). Moreover, it is a \emph{proper prefix}, noted \(x \pref y\),
if \(v \neq \varepsilon\). Given \(y = uxv\), the word~\(x\) is a
\emph{suffix}\index{word factoring!suffix} of~\(y\) if \(v =
\varepsilon\). Furthermore, it is a \emph{proper suffix} if \(u \neq
\varepsilon\). Let \(a\)~be any letter and \(x\), \(y\)~any word, then
the prefix relation is easy to define by an inference system as
\begin{mathpar}
\inferrule*{}{\varepsilon \prefeq y}
\qquad
\inferrule{x \prefeq y}{a \cdot x \prefeq a \cdot y}
\end{mathpar}

The purpose being to write a functional program for factoring, we need
to translate words and operations on them into terms of the
language. A letter is translated into a constant data constructor; for
example, \word{a} becomes \(\fun{a}()\). A word of more than one
letter is mapped to a stack of mapped letters, such as \word{hi} in
\([\fun{h}(),\fun{i}()]\). The catenation of a letter and a word is
translated as a push, like \(\word{a} \cdot \word{bed}\) becomes
\([\fun{a}(),\fun{b}(),\fun{e}(),\fun{d}()]\). The catenation of two
words is associated to stack catenation, so \(\word{ab} \cdot
\word{cd}\) leads to\index{cat@\fun{cat/2}}
\(\fun{cat}([\fun{a}(),\fun{b}()],[\fun{c}(),\fun{d}()])\).

% The transposition of a word is the reversal of the stack implementing
% the word.

As usual, the translation of the inference system defining
\((\prefeqName)\) into a function \fun{pre/2}\index{pre@\fun{pre/2}}
requires that the cases corresponding to the axioms evaluate in
\(\fun{true}()\) and the cases left unspecified \((\nprefeqName)\)
evaluate in \(\fun{false}()\):
\begin{equation*}
\fun{pre}(\el,y) \rightarrow \fun{true}();\quad
\fun{pre}(\cons{a}{x},\cons{a}{y}) \rightarrow \fun{pre}(x,y);\quad
\fun{pre}(x,y) \rightarrow \fun{false}().
\end{equation*}
The inference system is now a formal specification for the program.

A letter in a word can be uniquely characterised by a natural number,
called \emph{index}\index{word factoring!index}, assuming that the
first letter has index~\(0\) \citep{Dijkstra_1982}. If
\(x=\word{top}\), then the letter at index~\(0\) is written
\(\ind{x}{0}=\word{t}\) and the one at index~\(2\) is
\(\ind{x}{2}=\word{p}\). A factor~\(x\) of~\(y\) can be identified by
the index of~\(\ind{x}{0}\) in~\(y\). The end of the factor can also
be given; for example, \(x=\word{sit}\) is a factor of
\(y=\word{curiousity}\) at index~\(6\), written \(\ind{y}{6,8} = x\),
meaning \(\ind{y}{6} = \ind{x}{0}\), \(\ind{y}{7} = \ind{x}{1}\) and
\(\ind{y}{8} = \ind{x}{2}\). Given two words \(p\)~and~\(t\),
determining whether \(p\)~is a factor of~\(t\) is called
\emph{factoring~\(p\) in~\(t\)}.

Factor matching is common in text editing, although it is usually
better known as \emph{exact string matching} in the academic field of
\emph{stringology} or \emph{text
  algorithmics} \citep{CharrasLecroq_2004,CrochemoreHancartLecroq_2007}
\citep[\S{}32]{CLRS_2009}. Because of the asymmetric nature of
factoring, the word~\(p\) is called the \emph{pattern}\index{word
  factoring!pattern} and the word~\(t\) is the \emph{text}\index{word
  factoring!text}.

\section{Na\"{\i}ve factoring}
\label{sec:naive_factoring}
\index{word factoring!na\"{\i}ve $\sim$|(}

In section~\ref{def:linear_search}, \vpageref{def:linear_search}, we
introduced the linear search\index{linear search}, that is, the
stepwise search for the occurrence of an item in a stack. We can
generalise it to search for a series of items occurring consecutively
in a stack, that is, to solve the factoring problem. This approach is
qualified as being na\"{\i}ve because it is a simple extension of a
simple idea and it is implied that it is not the most efficient.

Everything starts with~\(\ind{p}{0}\) and~\(\ind{t}{0}\) being
compared, then, assuming \(\ind{p}{0} = \ind{t}{0}\), letters
\(\ind{p}{1}\) and~\(\ind{t}{1}\) are, in turn, compared etc. until
one of the words is exhausted or a mismatch occurs. Assuming that
\(p\)~is shorter than~\(t\), the former case means that \(p\)~is a
prefix of~\(t\). In the latter case, \(p\)~is shifted so
\(\ind{p}{0}\) is aligned with \(\ind{t}{1}\) and the comparisons are
resumed from there. If \(p\)~cannot be shifted anymore because its end
would surpass the end of~\(t\), then it is not a factor. The essence
of this procedure is summed up in \fig~\vref{fig:naive},
\begin{figure}[t]
\centering
\includegraphics[bb=75 621 352 715]{naive}
\caption{Na\"{\i}vely matching pattern~\(p\) against text~\(t\)
(failure in grey)\label{fig:naive}}
\end{figure}
where \(\ind{p}{i} \neq \ind{t}{j}\) (the letters
\word{a}~and~\word{b} are not relevant in themselves).

% Wrapping figure better declared before a paragraph
%
%\begin{wrapfigure}[15]{r}[0pt]{0pt}
% [15] vertical lines
% {r} right placement
% [0pt] of margin overhang
%\end{wrapfigure}

\Fig~\vref{fig:loc0}
\begin{figure}[t]
\centering 
\includegraphics[bb=71 548 253 721]{loc0}
\caption{Na\"{\i}ve factoring with \fun{loc\(_0\)/2}
\label{fig:loc0}}
\end{figure}
shows an abstract program implementing this scheme. The call
\(\fun{loc}_0(p,t)\)\index{loc0@\fun{loc\(_0\)/2}} evaluates in
\(\fun{absent}()\) if the pattern~\(p\) is not a factor of the
text~\(t\), otherwise in \(\fun{factor}(k)\), where \(k\)~is the index
in~\(t\) where~\(p\) occurs first. Conceptually, this design consists
in combining a linear search for the first letter of the pattern and a
prefix check for the rest of the pattern and text. It is important to
verify whether the invariants implicit in general do not break in the
presence of limit cases. For instance, in stack processing, set the
different stacks to be empty and interpret the result of single
rewrites and entire evaluations. We have \(\fun{pre}(\el,t)
\twoheadrightarrow \fun{true}()\), because \(t = \varepsilon \cdot
t\). Accordingly, \(\fun{loc}_0(\el,t) \twoheadrightarrow
\fun{factor}(0)\).

\paragraph{Refinements}

While this program composition is intuitive, it is too long. We may
remark that, after a call to \fun{pre/2}\index{pre@\fun{pre/2}}
evaluates in \(\fun{true}()\), the interpretation ends with
\(\fun{factor}(j)\). Dually, a value \(\fun{false}()\) is followed by
the call
\(\fun{loc}_0(p,t,j+1)\)\index{loc0@\fun{loc\(_0\)/3}}. Therefore,
instead of calling \fun{pre/2} and then inspecting the resulting value
to decide what to do next, we could have \fun{pre/2} take the
lead. This entails that it needs to receive additional arguments to be
able to end with \(\fun{factor}(j)\) or resume with
\(\fun{loc}_0(p,t,j+1)\), as expected. The corresponding code is shown
in \fig~\ref{fig:loc1}.
\begin{figure}[t]
\begin{equation*}
\boxed{%
\begin{array}{r@{\;}l@{\;}l}
\fun{loc}_1(p,t)   & \rightarrow & \fun{loc}_1(p,t,0).\\
\\
\fun{loc}_1(\cons{a}{p},\el,j) & \rightarrow & \fun{absent}();\\
\fun{loc}_1(p,t,j) & \rightarrow & \fun{pre}_1(p,t,p,t,j).\\
\\
\fun{pre}_1(\el,t,p',t',j) & \rightarrow & \fun{factor}(j);\\
\fun{pre}_1(\cons{a}{p},\cons{a}{t},p',t',j)
                   & \rightarrow & \fun{pre}_1(p,t,p',t',j);\\
\fun{pre}_1(p,t,p',\cons{a}{t'},j) & \rightarrow & \fun{loc}_1(p',t',j+1).
\end{array}}
\end{equation*}
\caption{Refinement of \fig~\vref{fig:loc0}
\label{fig:loc1}}
\end{figure}

Further examination reveals that we can merge
\fun{loc\(_1\)/3}\index{loc1@\fun{loc\(_1\)/3}} and
\fun{pre\(_1\)/5}\index{pre1@\fun{pre\(_1\)/5}} into
\fun{pre/5}\index{pre@\fun{pre/5}} in\index{word factoring!na\"{\i}ve
  $\sim$!program} \fig~\vref{fig:loc}. This kind of progressive
design, where a program is transformed into a guided series of
equivalent programs is called a \emph{refinement}. Here, each
refinement is more efficient than the preceding, but less legible than
the original, so each step must be cautiously checked.
\begin{figure}[h]
\begin{equation*}
\boxed{%
\begin{array}{r@{\;}l@{\;}l}
\fun{loc}(p,t)   & \xrightarrow{\smash{\pi}} & \fun{pre}(p,t,p,t,0).\\
\\
\fun{pre}(\el,t,p',t',j) & \xrightarrow{\smash{\rho}}
                         & \fun{factor}(j);\\
\fun{pre}(p,\el,p',t',j) & \xrightarrow{\smash{\sigma}}
                         & \fun{absent}();\\
\fun{pre}(\cons{a}{p},\cons{a}{t},p',t',j)
                         & \xrightarrow{\smash{\tau}}
                         & \fun{pre}(p,t,p',t',j);\\
\fun{pre}(p,t,p',\cons{b}{t'},j) & \xrightarrow{\smash{\upsilon}}
                         & \fun{pre}(p',t',p',t',j+1).
\end{array}}
\end{equation*}
\caption{Refinement of \fig~\vref{fig:loc1}
\label{fig:loc}}
\end{figure}

\paragraph{Termination}
\index{word factoring!na\"{\i}ve $\sim$!termination|(}

We want to show that the index in the text always increases, whether a
comparison fails or not, so we choose a lexicographic order on the
dependency pairs of \fun{pre/5}\index{termination!dependency pair}
made of the fourth and second arguments (definition~\eqref{def:lexico}
\vpageref{def:lexico}), where \(s \succ t\) if~\(t\) is the immediate
substack of~\(s\). The third rule satisfies \((t',\cons{a}{t}) \succ
(t',t)\). The fourth rule is also ordered, because \((\cons{b}{t'},t)
\succ (t',t')\).\index{word factoring!na\"{\i}ve
  $\sim$!termination|)}\hfill\(\Box\)

\paragraph{Completeness}
\index{word factoring!na\"{\i}ve $\sim$!completeness|(}
\index{induction!example|(}

Note how, in rule~\(\sigma\), the pattern~\(p\) can not be empty
because the rules are ordered and that case would always match
rule~\(\rho\). The completeness of the definition of \fun{pre/5}
deserves some attention and we need to justify why the call
\(\fun{pre}(\cons{a}{p},\cons{b}{t},p',\el,j)\), with \(a \neq b\),
cannot happen. Perhaps surprisingly, a more general statement is
easier to establish:
\begin{center}
\(\fun{loc}(p,t) \twoheadrightarrow \fun{pre}(p_0,t_0,p_0',t_0',j)\)
\textsl{implies} \(t_0' \succcurlyeq t_0\),
\end{center}
where (\(\succcurlyeq\)) is the reflexive substack relation. Let us
prove this property by \emph{induction on the length of the
  derivation}\index{induction!$\sim$ on the length of the
  derivation}. More precisely, we want to establish the proposition
\begin{equation*}
\pred{Comp}{n} \colon \fun{loc}(p,t) \xrightarrow{\smash{n}}
\fun{pre}(p_0,t_0,p_0',t_0',j) \Rightarrow t_0' \succcurlyeq t_0.
\index{Comp@\predName{Comp}}
\end{equation*}
\begin{itemize}

  \item The basis \(\pred{Comp}{0}\) is easy to prove without
    induction by means of rule~\(\pi\): \(\fun{loc}(p,t)
    \xrightarrow{\smash{\pi}} \fun{pre}(p,t,p,t,0)\) and \(t
    \succcurlyeq t\) trivially holds.

    \item The induction hypothesis is \(\pred{Comp}{n}\) and we want
      to show that, under this assumption, \(\pred{Comp}{n+1}\) holds
      as well. In other words, let us suppose that \(\fun{loc}(p,t)
      \xrightarrow{\smash{n}} \fun{pre}(p_0,t_0,p_0',t_0',j)\) implies
      \(t_0' \succcurlyeq t_0\) and we want to prove that
      \(\fun{pre}(p_0,t_0,p_0',t_0',j) \rightarrow
      \fun{pre}(p_1,t_1,p_1',t_1',k)\) implies \(t_1' \succcurlyeq
      t_1\). This rewrite can only be by means of~\(\tau\)
      or~\(\upsilon\). 
      \begin{itemize}

      \item If~\(\tau\), the induction hypothesis on the
        left\hyp{}hand side entails \(t' \succcurlyeq \cons{a}{t}\),
        so \(t' \succcurlyeq t\) in the right\hyp{}hand side;

      \item otherwise, the right\hyp{}hand side of~\(\upsilon\)
        trivially satisfies \(t' \succcurlyeq t'\).

      \end{itemize}
\end{itemize}
In summary, \(\pred{Comp}{0}\) is true and \(\pred{Comp}{n}
\Rightarrow \pred{Comp}{n+1}\). Therefore, the induction principle
yields \(\forall n.\pred{Comp}{n}\), which, in turn, entails our
formulation with~(\(\twoheadrightarrow\)). Note how, in this case,
this proof technique reduces to mathematical induction
on~\(n\).\hfill\(\Box\)
\index{induction!example|)}
\index{word factoring!na\"{\i}ve $\sim$!completeness|)}

\addcontentsline{toc}{subsection}{Cost}
\paragraph{Cost}

In the following cost analysis, let~\(m\) be the length of the
pattern~\(p\) and \(n\)~be the length of the text~\(t\). Moreover, as
it is common with search algorithms, we discriminate on \(p\)~being a
factor of~\(t\) or not.

\paragraph{Minimum cost}
\index{word factoring!na\"{\i}ve $\sim$!minimum cost}

If \(m \leqslant n\), the best case happens when the pattern is a
prefix of the text, so the evaluation trace is \(\pi\tau^m\rho\) and
\(\B{\fun{loc}}{m,n} = m + 2\). If \(m > n \), the minimum cost is
\(\B{\fun{loc}}{m,n} = \len{\pi\tau^n\sigma} = n + 2\). We can gather
these two cases in one formula:
\begin{equation*}
\B{\fun{loc}}{m,n} = \min\{m,n\} + 2.\index{loc@$\B{\fun{loc}}{m,n}$}
\end{equation*}

\paragraph{Maximum cost}
\index{word factoring!na\"{\i}ve $\sim$!maximum cost|(}

To find the maximum cost, let us investigate the cases where the
pattern is a factor of the text and when it is not.
\begin{itemize}

  \item \emph{The text contains the pattern.} The discovery of the
    pattern must be delayed as much as possible, therefore the worst
    case is when \(w\)~is a suffix of~\(t\) and every mismatch
    involves the last letter of the pattern. An example is \(p =
    \word{a}^{m-1}\word{b}\) and \(t = \word{a}^{n-1}\word{b}\). The
    evaluation trace corresponding to this case is \(\pi
    (\tau^{m-1}\upsilon)^{n-m} \tau^m\rho\), whose length is
    \(mn-m^2+m+2\).

  \item \emph{The text does not contain the pattern.} The pattern is
    not the prefix of any suffix of the text. The most delayed
    comparison failure should occur at the last letter of the pattern,
    like \(p = \word{a}^{m-1}\word{b}\) and \(t = \word{a}^{n}\). The
    cost is \(\len{\pi(\tau^{m-1}\upsilon)^{n-m+1}\tau^{m-1}\sigma} =
    mn - m^2 + 2m + 1\).

\end{itemize}
Therefore, the maximum cost is \(\W{\fun{loc}}{m,n} = mn - m^2 + 2m +
1\) \index{loc@$\W{\fun{loc}}{m,n}$}, when the pattern is not a factor
of the text and \(m \geqslant 1\). The previous analysis suggests an
improvement for that case, but would make the case when the text
contains the pattern the worst: just after rule~\(\tau\), let us add
\index{word factoring!na\"{\i}ve $\sim$!maximum cost|)}
\begin{equation*}
\belowdisplayskip=0pt
\fun{pre}([a],[b],p',t',j) \rightarrow \fun{absent}();\index{pre@\fun{pre/5}}
\end{equation*}

\paragraph{Average cost}
\index{word factoring!na\"{\i}ve $\sim$!average cost|(}

Let us suppose that \(0 < m \leqslant n\) and that the letters of
\(p\)~and~\(t\) are chosen from the same alphabet, whose cardinal
is~\(\breve{a} > 1\). Na\"{\i}ve factoring consists in matching a
pattern against the prefixes of the suffixes of a text, by decreasing
lengths. Let~\(\OM{\breve{a}}{m}\) \index{loc@$\OM{\breve{a}}{n}$} be
the average number of letter comparisons for comparing two words of
length~\(m\) over the alphabet~\(\breve{a}\). The average number
\(\OM{\fun{loc}}{m,n}\) of comparisons for na\"{\i}ve factoring is
\begin{equation}
\OM{\fun{loc}}{m,n} = (n-m+1)\OM{\breve{a}}{m} + \OM{\breve{a}}{m-1},
\label{eq:loc}
\end{equation}
because there are \(n-m+1\) suffixes of length at least~\(m\) and
\(1\)~suffix of length~\(m-1\), against which the pattern is matched.

The determination of \(\OM{\breve{a}}{m}\) is achieved by fixing the
pattern~\(p\) and letting the text~\(t\) vary over all possible
letters. There are \(\breve{a}^{m}\)~comparisons between
\(\ind{p}{0}\) and~\(\ind{t}{0}\), as much as there are different
texts; if \(\ind{p}{0}=\ind{t}{0}\), there are
\(\breve{a}^{m-1}\)~comparisons between \(\ind{p}{1}\)
and~\(\ind{t}{1}\), as much as there are different \(\ind{t}{1,m-1}\)
etc. In total, there are
\begin{equation*}
\breve{a}^{m} + \breve{a}^{m-1} + \dots +
\breve{a} = \breve{a}(\breve{a}^{m} - 1)/(\breve{a}-1)
\end{equation*}
comparisons. There are \(\breve{a}^m\)~possible texts, hence the
average is
\begin{equation*}
\OM{\breve{a}}{m}
 = \frac{\breve{a}(\breve{a}^{m}-1)}{\breve{a}^m(\breve{a}-1)}
 = \frac{\breve{a}}{\breve{a}-1}\left(1
                          - \frac{1}{\breve{a}^{m}}\right)
 < \frac{\breve{a}}{\breve{a}-1}
 \leqslant 2.
\end{equation*}
Since \(\OM{\breve{a}}{1} = 1\), we draw the following bounds from
equation~\eqref{eq:loc}:
\begin{equation*}
n - m + 2 \leqslant \OM{\fun{loc}}{m,n} < 2(n-m+2) \leqslant 2n + 4.
\end{equation*}
Na\"{\i}ve factoring is thus efficient in average, but its hypothesis
is unlikely to apply to random English texts. Moreover, notice how the
average cost gets down as the alphabet grows since \(\lim_{\breve{a}
  \to \infty}\OM{\breve{a}}{m} = 1\).\index{word
  factoring!na\"{\i}ve $\sim$!average cost|)} \index{word
  factoring!na\"{\i}ve $\sim$|)}


\section{Morris-Pratt algorithm}
\index{word factoring!Morris-Pratt|(}

In case of mismatch, the na\"{\i}ve algorithm resumes comparing the
first letters of~\(p\) without using the information of the partial
success, to wit, we know \(\ind{p}{0,i-1} = \ind{t}{j-i,j-1}\) and
\(\ind{p}{i} \neq \ind{t}{j}\) (see \fig~\vref{fig:naive}).  The
attempt at matching~\(p\) with \(\ind{t}{j-i+1,j-1}\) could reuse
\(\ind{t}{j-i+1,j-1} = \ind{p}{1,i-1}\), in other words,
\(\ind{p}{0,i-2}\)~is compared to~\(\ind{p}{1,i-1}\), i.e., the
pattern~\(p\) is compared to a part of itself. If we know an
index~\(k\) such that \(\ind{p}{0,k-1} = \ind{p}{i-k,i-1}\), that is,
\(\ind{p}{0,k-1}\)~is a \emph{border} of~\(\ind{p}{0,i-1}\) (also
known as a \emph{side}\index{side|see{word factoring, border}}), then
we can resume by comparing~\(\ind{t}{j}\)
with~\(\ind{p}{k}\). Clearly, the greater~\(k\), the more comparisons
are skipped, so we want to find the \emph{maximum borders} of the
prefixes of~\(p\).

\paragraph{Border}
\index{word factoring!border|(}

The border of a non\hyp{}empty word~\(y\) is a proper prefix of~\(y\)
which is also a suffix. For example, the word \word{abacaba} has three
borders: \(\varepsilon\), \word{a}~and~\word{aba}. The last one is the
maximum border and we write
\(\Border{}{\underline{\word{aba}}\word{c}\underline{\word{aba}}} =
\word{aba}\). Another example is \(\Border{}{\word{abac}} =
\varepsilon\), because \(\word{abac} =
\underline{\varepsilon}\word{abac}\underline{\varepsilon}\). Maximum
borders can overlap; consider, for example,
\(\Border{}{\underline{\word{aaa}}\word{a}} =
\Border{}{\word{a}\underline{\word{aaa}}} = \word{aaa}\).

The speed\hyp{}up brought by Morris and Pratt to the na\"{\i}ve search
is depicted in \fig~\vref{fig:mp}.
\begin{figure}[b]
\centering
\includegraphics[bb=74 621 367 717]{mp}
\caption{Morris-Pratt algorithm (failure in grey)
\label{fig:mp}}
\end{figure}
Notice that, contrary to na\"{\i}ve factoring, letters in the text are
compared in a strictly increasing order (never having to
backtrack). Consider the complete run in \fig~\vref{fig:mp_ex} where,
in the end,
\begin{figure}[t]
\centering
\includegraphics{mp_ex}
\caption{Morris\hyp{}Pratt algorithm at work (no match found)
\label{fig:mp_ex}}
\end{figure}
\(p\)~is not found to be a factor of~\(t\). As usual, letters on a
grey background correspond to mismatches. It is clear that
\(\Border{}{a} = \varepsilon\), for all letters~\(a\).

We now have the choice of finding either \(\Border{}{ay}\) or
\(\Border{}{ya}\), where \(y\)~is a non\hyp{}empty word. Since we are
interested in knowing the maximum borders of all the prefixes of a
given pattern, the latter is more suitable (\(y \pref ya\)). The idea
is to recursively consider \(\Border{}{y} \cdot a\): if it is a prefix
of~\(y\), then \(\Border{}{ya} = \Border{}{y} \cdot a\); otherwise, we
seek the maximum border of the maximum border of~\(y\), namely,
\(\Border{2}{y} \cdot a\) etc. until \(\Border{q}{y} \cdot a\) is a
prefix of~\(y\) or else \(\Border{q}{y} = \varepsilon\). For example,
\(\Border{}{y \cdot \word{a}} = \Border{3}{y} \cdot \word{a}\) in
\fig~\vref{fig:max_B}.
\begin{figure}[b]
\centering
\includegraphics{max_B}
\caption{\(\Border{}{y \cdot \word{a}}
   = \Border{}{\Border{}{y} \cdot \word{a}}
   = \Border{}{\Border{2}{y} \cdot \word{a}}
   = \Border{3}{y} \cdot \word{a}\).
\label{fig:max_B}}
\end{figure}

\par\vskip\baselineskip

Formally, for all words~\({y \neq \varepsilon}\) and any
letter~\(a\),
\begin{equation}
  \Border{}{a}         := \varepsilon;\qquad
  \Border{}{y \cdot a} := \left\{
    \begin{aligned}
      & \Border{}{y} \cdot a,
      && \text{if \(\Border{}{y} \cdot a \prefeq y\)};\\
      & \Border{}{\Border{}{y} \cdot a},
      && \text{otherwise.}
    \end{aligned}
  \right.
\label{eq:Border}
\end{equation}
Consider the following examples where \(y\)~and~\(\Border{}{y}\) are
given:
\begin{align*}
  y             &= \word{abaabb},
& \Border{}{y}  &= \varepsilon,
& \Border{}{y \cdot \word{b}}
                &= \Border{}{\Border{}{y} \cdot \word{b}}
                 = \Border{}{\word{b}} = \varepsilon;\\
  y             &= \word{baaaba},
& \Border{}{y}  &= \word{ba},
& \Border{}{y \cdot \word{a}}
                &= \Border{}{y} \cdot \word{a}
                 = \word{baa};\\
  y             &= \word{abbbab},
& \Border{}{y}  &= \word{ab},
& \Border{}{y \cdot \word{a}}
                &= \Border{}{\Border{}{y} \cdot \word{a}}
                 = \Border{2}{y} \cdot \word{a}
                 = \word{a}.
\end{align*}
\index{word factoring!border|)}

\paragraph{Failure function}
\index{word factoring!failure function|(}

Let us note~\(\wlen{y}\) the length of a word~\(y\). For a given
word~\(x\), let us define a function \(\MPfailureName_{x}\) on all its
prefixes as
\begin{equation}
  \MPfailure{x}{}{\wlen{y}}
:= \wlen{\Border{}{y}},\,\; \text{for
    all \(x\) and \(y \neq \varepsilon\) such
    that \(y \prefeq x\)}.\label{eq:MP_failure}
\end{equation}
For reasons which will be clear soon, this function is called the
\emph{failure function} of~\(x\). An equivalent definition is
\begin{equation*}
  \MPfailure{x}{}{i}
= \wlen{\Border{}{\ind{x}{0,i-1}}}, \,\;
\text{for all \(x\) and  \(i\) such that
\(0 < i \leqslant \wlen{x}\)}.
\end{equation*}
For example, \fig~\vref{fig:mp_fail} shows the table of the maximum
borders for the prefixes of the word \word{abacabac}.
\begin{figure}
\centering
\includegraphics{mp_fail}
\caption{Failure function of \word{abacabac}
\label{fig:mp_fail}}
\end{figure}
In \fig~\vref{fig:mp}, the length of the maximum border is~\(k\), so
\(k=\MPfailure{p}{}{i}\) and \(\smash{\Ind{p}{\MPfailure{p}{}{i}}}\)
is the first letter to be compared with~\(\ind{t}{j}\) after the
shift. Also, the figure assumes that \({i>0}\), so the border in
question is defined. Equations \eqref{eq:Border} \vpageref{eq:Border}
defining the maximum border can be unfolded as follows:
\begin{align*}
   \Border{}{ya}
&= \Border{}{\Border{}{y} \cdot a},
& \Border{}{y} \cdot a &\nprefeq y;\\
   \Border{}{\Border{}{y} \cdot a}
&= \Border{}{\Border{2}{y} \cdot a},
&  \Border{2}{y} \cdot a &\nprefeq \Border{}{y};\\
&\;\;\smash{\vdots} & &\;\;\smash{\vdots}\\
   \smash{\Border{}{\Border{p-1}{y} \cdot a}}
&= \smash{\Border{}{\Border{p}{y} \cdot a}},
& \smash{\Border{p}{y} \cdot a} &\nprefeq \smash{\Border{p-1}{y}};
\end{align*}
and \(\varepsilon \not\in \{y,\Border{}{y}, \dots,
\Border{p-1}{y}\}\). By transitivity, the equations entail
\(\Border{}{ya} = \Border{}{\Border{p}{y} \cdot a}\). Two cases are
possible: either \(\Border{p}{y} = \varepsilon\), so \(\Border{}{ya} =
\Border{}{a} = \varepsilon\), or the unfolding resumes until we find
the smallest \(q > p\) such that \(\Border{}{\Border{q-1}{y} \cdot a}
= \Border{}{\Border{q}{y} \cdot a}\) with \(\Border{q}{y} \cdot a
\prefeq \Border{q-1}{y}\). Because a border is a proper prefix, that
is to say, \(\Border{}{y} \pref y\), we have \(\Border{2}{y} =
\Border{}{\Border{}{y}} \pref \Border{}{y}\), yielding \(\Border{q}{y}
\cdot a \prefeq \Border{q-1}{y} \pref \dots \pref \Border{}{y} \pref
y\). Therefore \(\Border{q}{y} \cdot a \prefeq y\), since \({q > 0}\),
and \(\Border{}{ya} = \Border{q}{y} \cdot a\). This reasoning
establishes that
\begin{equation*}
\Border{}{ya} = 
\left\{
  \begin{aligned}
   & \Border{q}{y} \cdot a,
   && \text{if \(\Border{q}{y} \cdot a \prefeq y\)};\\
   & \varepsilon, 
   && \text{otherwise;}
  \end{aligned}
\right.
\end{equation*}
with the additional constraint that \(q\)~must be as small as
possible. This form of the definition of~\(\BorderName\) is simpler
because it does not contain an embedded call like
\(\Border{}{\Border{}{y} \cdot a}\). We can now take the lengths of
each sides of the equations, leading to
\begin{equation*}
\wlen{\Border{}{ya}} = 
\left\{
  \begin{aligned}
   & \wlen{\Border{q}{y} \cdot a} = 1 + \wlen{\Border{q}{y}},
   && \text{if \(\Border{q}{y} \cdot a \prefeq y\)};\\
   & \wlen{\varepsilon} = 0,
   && \text{otherwise.}
  \end{aligned}
\right.
\end{equation*}
If~\({ya \prefeq x}\), then \(\wlen{\Border{}{ya}} =
\MPfailure{x}{}{\wlen{ya}} = \MPfailure{x}{}{\wlen{y}+1}\). Let \(i :=
\wlen{y} > 0\).
\begin{equation*}
\MPfailure{x}{}{i+1} =
\left\{
  \begin{aligned}
   & 1 + \wlen{\Border{q}{y}},
   && \text{if \(\Border{q}{y} \cdot a \prefeq y\)};\\
   & 0,
   && \text{otherwise.}
  \end{aligned}
\right.
\end{equation*}
We need to work on~\(\wlen{\Border{q}{y}}\) now. From the definition
of~\(\MPfailureName\) by equation~\eqref{eq:MP_failure}
\vpageref{eq:MP_failure}, we deduce
\begin{equation}
\MPfailure{x}{q}{\wlen{y}} = \wlen{\Border{q}{y}},
\,\; \text{with \(y \prefeq x\)},
\label{eq:MP_failure_p}
\end{equation}
which we can prove by complete induction on~\(q\). Let us call this
property \(\pred{P}{q}\). Trivially, we have \(\pred{P}{0}\). Let us
suppose \(\pred{P}{n}\) for all \(n \leqslant q\): this is the
induction hypothesis. Let us suppose \(y \prefeq x\) and prove now
\(\pred{P}{q+1}\):
\begin{equation*}
  \MPfailure{x}{q+1}{\wlen{y}}
= \MPfailure{x}{q}{\MPfailure{x}{}{\wlen{y}}} 
= \MPfailure{x}{q}{\wlen{\Border{}{y}}}
\doteq \wlen{\Border{q}{\Border{}{y}}}
= \wlen{\Border{q+1}{y}},
\end{equation*}
where \((\smash{\doteq})\)~is a valid application of the induction
hypothesis because \(\Border{}{y} \pref y \prefeq x\). This proves
\(\pred{P}{q+1}\) and the induction principle entails that
\(\pred{P}{n}\) holds for all \(n \geqslant 0\). Therefore, equation
\eqref{eq:MP_failure_p} allows us to refine our definition of
\(\MPfailure{x}{}{i+1}\) as follows, with \(i>0\):
\begin{equation*}
\MPfailure{x}{}{i+1} =
\left\{
  \begin{aligned}
   & 1 + \MPfailure{x}{q}{i},
   && \text{if \(\Border{q}{y} \cdot a \prefeq y\)};\\
   & 0,
   && \text{otherwise.}
  \end{aligned}
\right.
\end{equation*}
There is one part of the definition \eqref{eq:Border}
\vpageref{eq:Border} that we did not use: \({\Border{}{a} :=
  \varepsilon}\). It implies \(\MPfailure{x}{}{1} =
\MPfailure{x}{}{\wlen{a}} = \wlen{\Border{}{a}} = \wlen{\varepsilon} =
0\) and, since the definition of~\(\MPfailureName\) implies
\(\MPfailure{x}{}{1} = 1 + \MPfailure{x}{}{0}\), so
\(\MPfailure{x}{}{0} = -1\). Property `\(\Border{q}{y} \cdot a \prefeq
y\) and \({ya \prefeq x}\) and \({\wlen{y} = i}\)' implies any of the
equalities \(\smash{\Ind{y}{\wlen{\Border{q}{y}}}} = a \Leftrightarrow
\smash{\Ind{y}{\MPfailure{x}{q}{i}}} = a \Leftrightarrow
\smash{\Ind{x}{\MPfailure{x}{q}{i}}} = \smash{\Ind{x}{\wlen{y}}}
\Leftrightarrow \smash{\Ind{x}{\MPfailure{x}{q}{i}}} = \ind{x}{i}\).
We now know
\begin{equation*}
 \MPfailure{x}{}{0}   = -1\quad\text{and}\quad
 \MPfailure{x}{}{i+1} =
   \left\{
     \begin{aligned}
       & 1 + \MPfailure{x}{q}{i},
       && \text{if \(\Ind{x}{\MPfailure{x}{q}{i}} = \ind{x}{i}\)};\\
       & 0,
       && \text{otherwise;}
     \end{aligned}
   \right.
\end{equation*}
where \(q\)~is the smallest nonzero natural satisfying the
condition. This can be further simplified into
\begin{equation*}
\MPfailure{x}{}{0} = -1
\quad \text{and} \quad
\MPfailure{x}{}{i+1} = 1 + \MPfailure{x}{q}{i},
\end{equation*}
where~\(i \geqslant 0\) and \(q>0\)~is the smallest natural such that
\(\MPfailure{x}{q}{i} = -1\) or \(\smash{\Ind{x}{\MPfailure{x}{q}{i}}}
= \ind{x}{i}\).

\mypar{Preprocessing}
\index{word factoring!Morris-Pratt preprocessing!definition|(}

The function call
\(\fun{fail}_0(x,i)\)\index{fail0@\fun{fail\(_0\)/2}}, defined in
\fig~\vref{fig:fail0}, implements \(\MPfailure{x}{}{i}\).
\begin{figure}[t]
\abovedisplayskip=0pt
\belowdisplayskip=0pt
\framebox[\columnwidth]{\vbox{%
\begin{gather*}
\fun{fail}_0(x,0) \rightarrow -1;\quad
\fun{fail}_0(x,i) \rightarrow
          1 + \fun{fp}(x,\fun{nth}(x,i-1),\fun{fail}_0(x,i-1)).\\
\fun{nth}(\cons{a}{x},0) \rightarrow a;\quad
\fun{nth}(\cons{a}{x},i) \rightarrow \fun{nth}(x,i-1).
\\
\inferrule*{}{\fun{fp}(x,a,-1) \rightarrow -1;}
\quad
\inferrule
  {\fun{nth}(x,k) \twoheadrightarrow a}
  {\fun{fp}(x,a,k) \twoheadrightarrow k}
\,;\quad
\fun{fp}(x,a,k) \rightarrow \fun{fp}(x,a,\fun{fail}_0(x,k)).
\end{gather*}
}}
\caption{The failure function \(\MPfailureName\) as \fun{fail\(_0\)/2}
\label{fig:fail0}}
\end{figure}
The function \fun{fp/3}\index{fp@\fun{fp/3}} (\emph{fixed point})
computes \(\MPfailure{x}{q}{i-1}\), starting with
\(\fun{fail}_0(x,i-1)\) and
\(\fun{nth}(x,i-1)\)\index{nth@\fun{nth/2}}, which denotes
\(\ind{x}{i-1}\) and is needed to check the condition
\(\smash{\Ind{x}{\MPfailure{x}{q}{i-1}}} = \ind{x}{i-1}\). The
equality test \(\MPfailure{x}{q}{i-1} = -1\) is performed by the first
rule of \fun{fp/3}.\index{word factoring!failure function|)}

The algorithm of Morris and Pratt requires that \(\MPfailure{x}{}{i}\)
be computed for all indexes~\(i\) of the pattern~\(x\) and, since it
depends on the values of some calls \(\MPfailure{x}{}{j}\), with \(j <
i\), it is more efficient to compute \(\MPfailure{x}{}{i}\) for
increasing values of~\(i\) and store them, so they can be reused
instead of being recomputed. This technique is called
\emph{memoisation}\index{memoisation} (not to be confused with
memorisation). In this instance, the evaluation of
\(\MPfailure{x}{}{i}\) relies on the memo
\([\pair{\ind{x}{i-1}}{\MPfailure{x}{}{i-1}},
\pair{\ind{x}{i-2}}{\MPfailure{x}{}{i-2}}, \dots,
\pair{\ind{x}{0}}{\MPfailure{x}{}{0}}]\). The memoising version of
\fun{fail\(_0\)/2} is named \fun{fail/2}\index{fail@\fun{fail/2}} in
\fig~\ref{fig:fail}.
\begin{figure}[b]
\centering
\abovedisplayskip=0pt
\belowdisplayskip=0pt
\framebox[0.85\columnwidth]{\vbox{%
\begin{gather*}
\fun{fail}(p,0) \rightarrow -1;
\quad
\fun{fail}(\cons{\pair{a}{k}}{p},i) \rightarrow 
\fun{fp}(p,a,k,i-1).
\\
\inferrule*{}{\fun{fp}(p,a,-1,i) \rightarrow 0;}\quad
\inferrule
  {\fun{suf}(p,i-k-1) \twoheadrightarrow \cons{\pair{a}{k'}}{p'}}
  {\fun{fp}(p,a,k,i) \twoheadrightarrow k+1}\,;\\
\inferrule
  {\fun{suf}(p,i-k-1) \twoheadrightarrow \cons{\pair{b}{k'}}{p'}}
  {\fun{fp}(p,a,k,i) \twoheadrightarrow \fun{fp}(p',a,k',k)}\,.
\\
\fun{suf}(p,0) \rightarrow p;\quad
\fun{suf}(\cons{a}{p},i) \rightarrow \fun{suf}(p,i-1).
\end{gather*}
}}
\caption{The failure function with memoisation\label{fig:fail}}
\end{figure}
Here, we work with the memo~\(p\), which is a reversed prefix, instead
of~\(x\), so we need to know its length~\(i\) in order to know how
many letters must be discarded by \fun{suf/2}\index{suf@\fun{suf/2}}
(\emph{suffix}): \(\fun{suf}(x,i-k-1)\) instead of
\(\fun{fail}_0(x,k)\). Thanks to the memo, \fun{fp/4} does not need to
call \fun{fail/2}, just to look in~\(p\) with \fun{suf/2}. Note that,
as a small improvement, we also moved the increment: instead of \(1 +
\fun{fp}(\dots)\) and \(\dots \twoheadrightarrow k\), we do now
\(\fun{fp}(\dots)\) and \(\dots \twoheadrightarrow k+1\).

Let us name \fun{pp/1}\index{pp@\fun{pp/1}} (\emph{preprocessing}) the
function computing the stack \([\pair{\ind{x}{0}}{\MPfailure{x}{}{0}},
\pair{\ind{x}{1}}{\MPfailure{x}{}{1}}, \dots,
\pair{\ind{x}{m-1}}{\MPfailure{x}{}{m-1}}]\) for a pattern~\(x\) of
length~\(m\). Its definition is shown in \fig~\vref{fig:pp},
\begin{figure}[t]
\begin{equation*}
\boxed{%
\begin{array}{r@{\;}l@{\;}l}
\fun{pp}(x) & \rightarrow & \fun{pp}(x,\el,0).\\
\\
\fun{pp}(\el,p,i) & \rightarrow & \fun{rev}(p);\\
\fun{pp}(\cons{a}{x},p,i)
  & \rightarrow
  & \fun{pp}(x,\cons{\pair{a}{\fun{fail}(p,i)}}{p},i+1).
\end{array}}
\end{equation*}
\caption{Preprocessing of a pattern~\(y\) by \fun{pp/1}
\label{fig:pp}}
\end{figure}
where \fun{rev/1}\index{rev@\fun{rev/1}} is the reversal function
(definition~\eqref{def:rev} \vpageref{def:rev}), and
\fun{pp/1}\index{pp@\fun{pp/1}} simply calls the failure function
\fun{fail/2} for each new index~\(i\) on the current memo~\(p\) and
creates a new memo by pairing the failure index with the current
letter and pushing on the current memo
(\(\cons{\pair{a}{\fun{fail}(p,i)}}{p}\)). The stack reversal at the
end is necessary because the memo contains the letters in reversed
order with respect to the pattern. For example, the example in
\fig~\vref{fig:mp_fail} leads to the evaluation
\begin{equation*}
\fun{pp}(x) \twoheadrightarrow
[\pair{\word{a}}{-1}, \pair{\word{b}}{0},
\pair{\word{a}}{0}, \pair{\word{c}}{1},
\pair{\word{a}}{0}, \pair{\word{b}}{1},
\pair{\word{a}}{2}, \pair{\word{c}}{3}],
\end{equation*}
where \(x=\word{abacabac}\). If \(x=\word{ababaca}\), then
\begin{equation*}
\belowdisplayskip=0pt
\fun{pp}(x) \twoheadrightarrow
[\pair{\word{a}}{-1}, \pair{\word{b}}{0},
\pair{\word{a}}{0}, \pair{\word{b}}{1},
\pair{\word{a}}{2}, \pair{\word{c}}{3},
\pair{\word{a}}{0}].
\end{equation*}
\index{word factoring!Morris-Pratt preprocessing!definition|)}

%\addcontentsline{toc}{subsection}{Cost}
\paragraph{Minimum cost} 
\index{word factoring!Morris-Pratt preprocessing!minimum cost|(}

It is clear from the definition~\eqref{eq:Border} \vpageref{eq:Border}
that the determination of the maximum border of a non\hyp{}empty word
requires finding the maximum borders of some or all proper prefixes,
so, if the word contains \(n\)~letters, at least \(n-1\) comparisons
are needed, as the border of the first letter alone needs no
comparison. This lower bound is tight, as the following reasoning
shows. Let us call \emph{positive comparison}\index{word
  factoring!comparison!positive $\sim$} a successful prefix test as
found in the definition of~\(\BorderName\), that is, \(\Border{}{y}
\cdot a \prefeq y\). Dually, a \emph{negative comparison}\index{word
  factoring!comparison!negative $\sim$} is a failed prefix test. In
order to minimise the number of calls to evaluate \(\Border{}{ya}\),
we may notice that a positive comparison only entails the evaluation
of~\(\Border{}{y}\), whilst a negative comparison requires two:
\(\Border{}{\Border{}{y} \cdot a}\). Therefore, the first idea may be
to assume that only positive comparisons occur:
\begin{equation*}
\Border{}{x} \!\eqn{n-2}\! 
\Border{}{\ind{x}{0,n-2}} \cdot \ind{x}{n-1}
\!\eqn{n-1}\!\! \dots
\eqn{0}\! \Border{}{\ind{x}{0}} \cdot \ind{x}{1,n-1}
\!=\! \ind{x}{1,n\!-\!1},
\end{equation*}
where \((\smash{\eqn{i}})\)~implies \(\Border{}{\ind{x}{0,i}} \cdot
\ind{x}{i+1} \prefeq \ind{x}{0,i}\), for \(0 \leqslant i \leqslant
n-2\). Firstly, \(i=0\) and the corresponding positive comparison
yields \(\ind{x}{0} = \ind{x}{1}\). Unfolding the other comparisons
yields \(\ind{x}{0} = \ind{x}{1} = \dots = \ind{x}{n-1}\), so a best
case is \(x=a^n\), for any letter~\(a\).

But there is another case,
because the outermost call to~\(\BorderName\) after a negative
comparison does not imply a comparison if its argument is a single
letter:
\begin{align*}
\Border{}{x}
&\eqn{n-2} \Border{}{\Border{}{\ind{x}{0,n-2}} \cdot \ind{x}{n-1}}\\
&\eqn{n-3} \Border{}{\Border{}{\Border{}{\ind{x}{0,n-3}} \cdot
    \ind{x}{n-2}} \cdot \ind{x}{n-1}}\\
&\;\;\;\smash{\vdots}\\
&\;\mathrel{\stackrel{\smash[t]{0}}{=}}
  \Border{}{\Border{}{\ldots \Border{}{\Border{}{\ind{x}{0}} \cdot
        \ind{x}{1}} \dots} \cdot \ind{x}{n-1}}\\
&\,\,\doteq
  \Border{}{\Border{}{\ldots \Border{}{\Border{}{\ind{x}{1}} \cdot
            \ind{x}{2}} \dots} \cdot \ind{x}{n-1}}\\
&\;\;\;\smash{\vdots}\\
&\,\,\doteq \Border{}{\ind{x}{n-1}} = \varepsilon.
\end{align*}
where \((\smash{\eqn{i}})\)~implies \(\Border{}{\ind{x}{0,i}} \cdot
\ind{x}{i+1} \nprefeq \ind{x}{0,i}\), for \(0 \leqslant i \leqslant
n-2\) and \((\smash{\doteq})\) involves no comparisons. Starting with
\(i=0\) yields \(\ind{x}{1} \neq \ind{x}{0}\), then \(i=1\) leads to
\(\ind{x}{2} \neq \ind{x}{0}\) etc. so the consequences of all these
negative comparisons are \(\ind{x}{0} \neq \ind{x}{i}\), for \(1
\leqslant i \leqslant n-2\). The number of negative comparisons is
\(n-1\), thus is minimal, but the shape of the word is different than
previously, as the first letter must differ from all the
following. Let \(\OB{\fun{pp}}{n}\)\index{pp@$\OB{\fun{pp}}{n}$} be
the minimum number of comparisons involved in the evaluation of
\(\fun{pp}(x)\)\index{pp@\fun{pp/1}}, where the length of the
pattern~\(x\) is~\(n\). It is the same as the number of comparisons to
evaluate \(\Border{}{\ind{x}{0,n-2}}\) when \(\ind{x}{0,n-2}\)~is a
best case. Therefore,
%\begin{equation*}
\(\OB{\fun{pp}}{n} = n - 2\).
%\end{equation*}
\index{word factoring!Morris-Pratt preprocessing!minimum cost|)}

\paragraph{Maximum cost}
\index{word factoring!Morris-Pratt preprocessing!maximum cost|(}

The determination of the maximum border of a word implies finding the
maximum borders of some or all proper prefixes, so, if we want to
maximise the number of comparisons, we may want to compute as many
borders as possible. In order to do so, evaluating \(\Border{}{x}\)
would lead to finding the maximum border of a factor of
length~\(n-1\), where \(n\)~is the length of~\(x\). The best case
\(x=a^{n}\) showed that \(\Border{}{x} = \ind{x}{1,n-1}\), which fits
our purpose, except we would like \(\Border{}{\ind{x}{1,n-1}}\). In
other words, we add the constraint that the first comparison is
negative:
\begin{align*}
\Border{}{x}
&\eqn{n-1} \Border{}{\Border{}{\ind{x}{0,n-2}} \cdot \ind{x}{n-1}}\\
&\eqn{n-2} \Border{}{\Border{}{\ind{x}{0,n-3}} \cdot \ind{x}{n-2,n-1}}\\
&\;\;\;\smash{\vdots}\\
&\;\mathrel{\stackrel{\smash[t]{1}}{=}}
  \Border{}{\Border{}{\ind{x}{0}} \cdot \ind{x}{1,n-1}}
= \Border{}{\ind{x}{1,n-1}},
\end{align*}
where \((\smash{\eqn{n-1}})\) supposes \(\Border{}{\ind{x}{0,n-2}}
\cdot \ind{x}{n-1} \nprefeq \ind{x}{0,n-2}\), and
\((\smash{\eqn{i}})\), with \(1 \leqslant i \leqslant n-2\),
corresponds to \(\Border{}{\ind{x}{0,i}} \cdot \ind{x}{i+1} \prefeq
\ind{x}{0,i}\). These constraints imply \(\ind{x}{0} = \ind{x}{1} =
\dots = \ind{x}{n-2} \neq \ind{x}{n-1}\), that is to say,
\(x=a^{n-1}b\), with \(a \neq b\). Up to now, the number of
comparisons is~\(n-1\), as in the minimal case, but the evaluation
continues as follows:
\begin{equation*}
\Border{}{a^{i}b}
\eqn{i} \Border{}{\Border{}{a^{i}} \cdot b}\\
\doteq \Border{}{\Border{}{a^{i-1}} \cdot ab}\\
\doteq \dots \doteq \Border{}{a^{i-1}b},
\end{equation*}
for \(1 \leqslant i \leqslant n-2\) and \((\smash{\eqn{i}})\) entails
the negative comparisons \(\Border{}{a^{i}} \cdot b \nprefeq a^{i}\)
and the positive comparisons \((\smash{\doteq})\), which we do not
count because we have in mind to find
\(\OW{\fun{pp}}{n}\)\index{pp@$\OW{\fun{pp}}{n}$}, so repeated
evaluations of the same border do not entail repeated comparisons
thanks to memoisation\index{memoisation}. Thus, we have \(n-2\)
negative comparisons until \(\Border{}{b} = \varepsilon\), which, with
the \(n-1\) earlier positive comparisons, sum up \(2n-3\). Since
\(\OW{\fun{pp}}{n}\) is the number of comparisons to compute
\(\Border{}{\ind{x}{0,n-2}}\) without repetitions, we have
\begin{equation*}
\belowdisplayskip=0pt
\OW{\fun{pp}}{n} = 2(n-1)-3 = 2n - 5.
\end{equation*}
\index{word factoring!Morris-Pratt preprocessing!maximum cost|)}

\vspace*{-20pt}

\mypar{Search}
\index{word factoring!Morris-Pratt!search|(}

We found above that \(n-2 \leqslant \OC{\fun{pp}}{n} \leqslant 2n -
5\), where the bounds are tight if \(n \geqslant 3\). To make use of
the value of \(\fun{pp}(p)\), we could start by modifying the linear
search in section~\ref{sec:naive_factoring}, in particular the program
in \fig~\ref{fig:loc} on page~\pageref{fig:loc}, while keeping an eye
on \fig~\vref{fig:mp}. The result is displayed in
\fig~\vref{fig:mp_def}. Note how the first argument of
\fun{mp/5}\index{mp@\fun{mp/5}}, \(p\), is the working copy and the
third, \(p'\), is the original which remains invariant (it is used to
reset~\(p\) after a letter mismatch). Indexes~\(i\), \(j\) and~\(k\)
are the same as in \fig~\vref{fig:mp}. The latter is none other than
the value computed by the failure function; variables \(i\)~and~\(j\)
are incremented each time a letter in the pattern is successfully
matched against a letter in the text (third rule of \fun{mp/5}) and
\(j\)~is also incremented each time there is a mismatch of the first
letter of the pattern (fourth rule of \fun{mp/5}).\index{word
  factoring!Morris-Pratt!search|)}
\begin{figure}[h]
\centering
\abovedisplayskip=0pt
\belowdisplayskip=0pt
\framebox[0.8\columnwidth]{\vbox{%
\begin{gather*}
\inferrule
  {\fun{pp}(p) \twoheadrightarrow p'}
  {\fun{mp}(p,t) \twoheadrightarrow \fun{mp}(p',t,p',0,0)}.
\\
\begin{array}{r@{\;}l@{\;}l}
\fun{mp}(\el,t,p',i,j)   & \rightarrow & \fun{factor}(j-i);\\
\fun{mp}(p,\el,p',i,j)   & \rightarrow & \fun{absent}();\\
\fun{mp}(\cons{\pair{a}{k}}{p},\cons{a}{t},p',i,j)
                         & \rightarrow & \fun{mp}(p,t,p',i+1,j+1);\\
\fun{mp}(\cons{\pair{a}{-1}}{p},\cons{b}{t},p',0,j)
                         & \rightarrow & \fun{mp}(p',t,p',0,j+1);\\
\fun{mp}(\cons{\pair{a}{k}}{p},t,p',i,j)
                         & \rightarrow
                         & \fun{mp}(\fun{suf}(p',k),t,p',k,j).
\end{array}
\end{gather*}}}
\caption{Morris-Pratt algorithm (search phase)\label{fig:mp_def}}
\end{figure}


\addcontentsline{toc}{subsection}{Cost}
\paragraph{Minimum cost} 
\index{word factoring!Morris-Pratt!minimum cost|(}

Let
\(\smash[t]{\OB{\fun{mp/5}}{m,n}}\)\index{mp@$\OB{\fun{mp/5}}{m,n}$}
be the minimum number of comparisons performed during the evaluation
of \fun{mp/5}, where \(m\)~is the length of the pattern and \(n\)~is
the length of the text. Just as with na\"{\i}ve factoring, the best
case is when the pattern is a prefix of the text, so
\(\OB{\fun{mp/5}}{m,n} = m\). Taking into account the preprocessing
stage, the minimum number of comparisons
\(\OB{\fun{mp}}{m,n}\)\index{mp@$\OB{\fun{mp}}{m,n}$} of
\fun{mp/2}\index{mp@\fun{mp/2}} is
\begin{equation*}
\belowdisplayskip=0pt
\OB{\fun{mp}}{m,n} = \OB{\fun{pp}}{m} + \OB{\fun{mp/5}}{m,n} = (m-2)
+ m = 2m - 2.
\end{equation*}
\index{word factoring!Morris-Pratt!minimum cost|)}

\vspace*{-10pt}

\paragraph{Maximum cost}
\index{word factoring!Morris-Pratt!maximum cost|(}

Since the Morris\hyp{}Pratt algorithm only reads the text forwards,
the worst case must maximise the number of times the letters of the
text~\(t\) are compared with a letter in the pattern~\(p\). Therefore,
the first letter of the pattern cannot differ from all the letters of
the text, otherwise each letter of the text would be compared exactly
once. Let us assume the exact opposite: \(\ind{p}{0} = \ind{t}{i}\),
with \(i \geqslant 0\). But this would also imply one comparison per
letter in the text. The way to force the pattern to shift as little as
possible is to further impose \(\ind{p}{1} \neq \ind{t}{i}\), for
\(i>0\). In short, this means that \(ab \prefeq p\), with letters
\(a\)~and~\(b\) such that \(a \neq b\) and \(t=a^n\). A simple
drawing is enough to reveal that this configuration leads to the
maximum number of comparisons \(\OW{\fun{mp/5}}{m,n} = 2n -
1\)\index{mp@$\OW{\fun{mp/5}}{m,n}$}, as each letter in the text is
compared twice, except the first, which is compared once. Taking into
account the preprocessing stage, the maximum number of comparisons
\(\OW{\fun{mp}}{m,n}\) \index{mp@$\OW{\fun{mp}}{m,n}$} of
\fun{mp/2}\index{mp@\fun{mp/2}} satisfies
\begin{equation*}
\belowdisplayskip=0pt
\OW{\fun{mp}}{m,n} = \OW{\fun{pp}}{m} + \OB{\fun{mp/5}}{m,n} = (2m-5)
+ (2n-1) = 2(n + m - 3).
\end{equation*}
\index{word factoring!Morris-Pratt!maximum cost|)}

\vspace*{-10pt}

\mypar{Metaprogramming}
\index{word factoring!Morris-Pratt!metaprogramming|(}

The previous study leads to programs for the preprocessing and search
phases that somewhat obscure the main idea supporting the algorithm of
Morris and Pratt, to wit, the use of the maximum borders of the proper
prefixes of the pattern and the forward\hyp{}only reading of the
text. The reason for that somewhat unfortunate situation is that, for
efficiency imperatives, we have to memoise\index{memoisation} the
values of the failure function and, instead of working with the
original pattern, we proceed with a version of it augmented with these
values. Also, the utilisation of stacks for modelling the pattern
slows down and obfuscates the reading of the letters and the shifts.

If the pattern is fixed, a more legible approach is available,
consisting in the modification of the preprocessing stage so that a
dedicated program is output. This kind of taylored method, where a
program is the result of the execution of another, is called
\emph{metaprogramming}\index{metaprogramming}. Of course, it is an
option only if the time needed to output, compile and execute a
program is amortised in the long run, which implies for the problem at
hand that the pattern and the text are expected to be significantly
long or that the search is likely to be repeated with the same pattern
on other texts (or the remainder of the same text after an occurrence
of the pattern has been found).

% Wrapping figure better declared before a paragraph
%
\setlength{\intextsep}{0pt}
\begin{wrapfigure}[29]{r}[0pt]{0pt}
% 29 vertical lines
% {r} mandatory right placement
% [0pt] of margin overhang
\centering
\includegraphics[bb=71 360 238 723]{metaprog}
\caption{Factoring \word{abacabac} %in~\(t\) with a metaprogram
\label{fig:metaprog}}
\end{wrapfigure}
There is a graphical way to represent the contents of the table in
\fig~\vref{fig:mp_fail} called \emph{deterministic finite automaton}
\index{word factoring!Morris-Pratt!automaton} \index{finite
  automaton!deterministic $\sim$} and shown in
\fig~\vref{fig:abacabac}.
\begin{figure}[b]
\centering
\includegraphics[bb=80 651 390 710]{abacabac}
\caption{Morris-Pratt automaton for the pattern \word{abacabac}
\label{fig:abacabac}}
\end{figure}
We will here only describe informally automata; for a full treatment,
see \cite{VanLeeuwen_1990c, HopcroftMotwaniUllman_2003,
  Sakarovitch_2003}. Consider that the circles, called
\emph{states}\index{states}, contain the values of~\(i\) from the
table. The edges, called \emph{transitions}\index{transition}, between
two states are of two kinds: either solid and carrying a letter,
called \emph{label}\index{label}, or dotted and going backwards. The
succession of states throughout solid edges make the word
\(x=\word{abacabac}\). The rightmost state is distinguished by a
double circling because it marks the end of~\(x\). There is a back
edge between state~\(i\) and~\(j\) only if \(\MPfailure{x}{}{i} = j\).
The leftmost state has an incoming, solid edge without a source and a
dotted, outgoing edge. The former simply denotes the beginning of the
word and the latter corresponds to the special value
\(\MPfailure{x}{}{0} = -1\).  What matters for us is that the
intuitive support brought by an automaton also can be implemented
intuitively, with each state corresponding to one function and the
outgoing transitions associated with different rules of the definition
of the state function. The example \word{abacabac} is shown in
\fig~\ref{fig:metaprog}. The automaton states, \(0\), \(1\), through
\(7\), correspond to the functions \fun{zero/2}, \fun{one/2}
etc. through \fun{seven/2}. Notice how
\fun{mp\(_0\)/1}\index{mp0@\fun{mp\(_0\)/1}} sets the index to~\(0\)
when initialising the first state, that is, calling
\fun{zero/2}\index{zero@\fun{zero/2}}. The index~\(j\) plays the same
role as in \fig~\ref{fig:mp} on page~\pageref{fig:mp}. The first rule
of each function corresponds to a rightwards transition in the
automaton in \fig~\vref{fig:abacabac} and the second rule is a
backwards transition, that is, a failure, except in \fun{zero/2},
where it means that the pattern is shifted by one letter. The function
\fun{zero/2} has a third rule handling the case when the pattern is
absent in the text. We could add a similar rule to the other
functions, as an optimisation, but we opt for brevity and let
successive failure rules bring us back to
\fun{zero/2}\index{zero@\fun{zero/2}}. The first rule of
\fun{seven/2}\index{seven@\fun{seven/2}} is special as well, because
it is used when the pattern has been found. Note the index \(j-6\),
clearly showing that the length of the pattern is part of the program,
which is hence a metaprogram.\index{word
  factoring!Morris-Pratt!metaprogramming|)} \index{word
  factoring!Morris-Pratt|)}

\mypar{Knuth's variant}
\index{word factoring!Knuth-Morris-Pratt|(}

In \fig~\vref{fig:mp}, if \(a = \word{a}\), then the sliding would
immediately lead to a comparison failure. Hence let us compare
\(\smash{\Ind{p}{\MPfailure{p}{}{i}}}\) to~\(\ind{t}{j}\) only if
\(\smash{\Ind{p}{\MPfailure{p}{}{i}}} \neq \ind{p}{i}\). Else, we
consider the maximum border of the maximum border etc. until we find
the smallest~\(q\) such that \(\smash{\Ind{p}{\MPfailure{p}{q}{i}}}
\neq \ind{p}{i}\). This is an improvement by
\cite{KnuthMorrisPratt_1977}. There is an updated reprint by
\cite{Knuth_2010} and a treatment based on automata theory by
\cite{CrochemoreHancartLecroq_2007}, in its section~2.6. See also an
interesting derivation of the program by algebraic refinements in the
book by \cite{Bird_2010}. In terms of the search automaton, when a
failure occurs at state~\(i\) on the letter~\(a\), we follow the back
edge to state \(\MPfailure{x}{}{i}\), but, if the normal transition
is~\(a\) again, we follow another back edge etc. until there is a
transition different from~\(a\) or we have to shift the pattern. The
improvement proposed by Knuth consists in replacing 
\begin{figure}[b]
\centering
\includegraphics[bb=82 646 388 720]{kmp_abac}
\caption{Knuth-Morris-Pratt automaton for the pattern \word{abacabac}
\label{fig:kmp_abac}}
\end{figure}
all these successive failure transitions by only one. For example, the
automaton in \fig~\ref{fig:kmp_abac} is Knuth's optimisation of the
one in \fig~\ref{fig:abacabac}.

\index{word factoring!Knuth-Morris-Pratt|)}
\index{word factoring|)}

\paragraph{Exercises}

\begin{enumerate}

  \item Find \(\M{\fun{loc}}{m,n}\).

  \item Prove that \fun{pp/1} and \fun{mp/2} terminate.

  \item Prove \(\fun{loc/2} = \fun{mp/2}\) (correctness of the
    algorithm of Morris and Pratt).

  \item Find \(\B{\fun{pp}}{m}\) and \(\W{\fun{pp}}{m}\). (Mind
    the cost of \fun{suf/2}.)

  \item Find \(\B{\fun{mp}}{m,n}\) and \(\W{\fun{mp}}{m,n}\).

  \item Find a simple modification to avoid calling \fun{rev/1} in
  \fig~\vref{fig:pp}.\label{factoring:trick}

  \item Modify \fun{fail/2} so that \fun{mp/2} implements the
    Knuth\hyp{}Morris\hyp{}Pratt algorithm. Study the best and worst
    cases of this variant and show that \(\OW{\fun{pp}}{m} = 2m-6\),
    for \(m \geqslant 3\).

  \item Write the metaprogram corresponding to the automaton in
    \fig~\vref{fig:kmp_abac}.

  \item Write a function \fun{rlw/2}\index{rlw@\fun{rlw/2}}
    (\emph{remove the last word}) such that \(\fun{rlw}(w,t)\) is
    rewritten into the text~\(t\) if the word~\(w\) is absent,
    otherwise into~\(t\) without the last occurrence of~\(t\) in it.

\end{enumerate}
