\section{Specification of tokens}

\Fig~\vref{fig:lexer_parser}
\begin{figure}[b]
\centering
\includegraphics[bb=71 640 368 721]{lexer_parser}
\caption{Lexer and parser\label{fig:lexer_parser}}
\end{figure}
shows that the lexical analyser is the first phase of a compiler. Its
main task is to read the input characters and produce a sequence of
tokens that the syntax analyser uses. Upon receiving a request for a
token (\emph{get token}) from the parser, the lexical analyser reads
input characters until a lexeme is identified and returned to the
parser together with the corresponding token. Usually, a lexical
analyser is in charge of
\begin{itemize*}

  \item stripping out from the source program comments and white
    spaces, in the form of blank, tabulation and newline characters;

  \item keeping trace of the position of the lexemes in the source
    program, so the error handler can refer to exact positions in
    error messages.

\end{itemize*}
A \emph{token} is a set of strings which are interpreted in the same
way, for a given source language. For instance, \tokenName{id} is a
token denoting the set of all possible identifiers. A \emph{lexeme} is
a string belonging to a token. For example, \verb+5.4+ is a lexeme of
the token \tokenName{num}. Tokens are defined by means of
\emph{patterns}. A pattern is a kind of compact rule describing all
the lexemes of a token. A pattern is said to \emph{match} each lexeme
in the token. For example, in the \Pascal statement
\begin{verbatim}
const pi = 3.14159;
\end{verbatim}
the substring \texttt{pi} is a lexeme for the token \tokenName{id}
(\emph{identifier}).

Most recent programming languages distinguish a finite set of strings
that match the identifiers but are not part of the identifier token:
the \emph{keywords}. For example, in \Ada, \texttt{function} is a
keyword and, as such, is not a valid identifier. In~C, \texttt{int} is
a keyword and, as such, cannot be used as an identifier, for instance
to declare a variable. Nevertheless, it is common not to create
explicitly a \tokenName{keyword} token and let each keyword lexeme be
the only one of its own token, as displayed in
\fig~\vref{fig:tokens_lexemes}.
\begin{figure}
\centering
\begin{tabular}{@{}>{\bfseries\sffamily}l|>{\tt}c|l@{}}
\toprule
  \multicolumn{1}{c}{\textsc{Token}}
& \multicolumn{1}{c}{\textsc{Sample lexemes}} 
& \multicolumn{1}{c}{\textsc{Informal pattern}}\\
\midrule
id       & pi count D2 \dots & letter followed by letters and digits\\
relop    & < <= = >= >   & \texttt{<} or \texttt{<=} or \texttt{<} or \texttt{=} or \texttt{>=} or \texttt{>}\\
const    & const         & \texttt{const}\\
if       & if            & \texttt{if}\\
num      & 3.14 4 .2E2 \ldots  & numeric constant\\
literal  & "message" ""  \ldots & characters between \texttt{"} and \texttt{"} except \texttt{"}\\
\bottomrule
\end{tabular}
\caption{Examples of tokens and lexemes\label{fig:tokens_lexemes}}
\end{figure}

Regular expressions are an important notation for specifying
patterns. Each pattern matches a set of strings, so regular
expressions will serve as names for sets of strings. The term
\emph{alphabet} denotes any finite set of symbols. Typical examples of
symbols are letters and digits, for example, the set \(\{0, 1\}\) is
the \emph{binary alphabet}; \textsc{ascii} is another example of
computer alphabet.

A \emph{string} over some alphabet is a finite sequence of symbols
drawn from that alphabet. The terms \emph{sentence} and \emph{word}
are often used as synonyms. The length of string~\(s\), usually
noted~\(\strlen{s}\), is the number of occurrences of symbols
in~\(s\). For example, \texttt{banana} is a string of
length~\(6\). The \emph{empty string}, denoted~\(\varepsilon\), is a
special string of length zero. More informal definitions are given in
the table in \fig~\vref{fig:table}.
\begin{figure}
\centering
\setlength\tymin{70pt}
\begin{tabulary}{\linewidth}{@{}CL@{}}
\toprule
  \textsc{Term}
& \multicolumn{1}{c}{\textsc{Informal definition}}\\
\midrule
  \emph{prefix} of \(s\)
& A string obtained by removing zero or more trailing symbols of
  string~\(s\); \emph{e.g.,} \texttt{ban} is a prefix of \texttt{banana}.\\
\hline
  \emph{suffix} of \(s\)
& A string formed by deleting zero or more of the leading symbols
  of~\(s\); \emph{e.g.,} \texttt{nana} is a suffix of \texttt{banana}.\\
\hline
  \emph{substring} of \(s\)
& A string obtained by deleting a prefix and a suffix from~\(s\);
\emph{e.g.,} \texttt{nan} is a substring a \texttt{banana}. Every prefix and
every suffix of~\(s\) is a substring~\(s\), but not every substring
of~\(s\) is a prefix or a suffix of~\(s\). For every string~\(s\),
both \(s\)~and~\(\varepsilon\) are prefixes, suffixes and substrings
of~\(s\).\\
\hline
  \emph{proper} prefix, suffix or substring of~\(s\)
& Any non-empty string~\(x\), that is, respectively, a prefix, suffix,
  substring of~\(s\) such that \(s \neq x\); \emph{e.g.,}
  \(\varepsilon\)~and \texttt{banana} are \emph{not} proper prefixes
  of \texttt{banana}.\\
\hline
  \emph{subsequence} of~\(s\)
& Any string formed by deleting zero or more not necessarily
  contiguous symbols from~\(s\); \emph{e.g.,} \texttt{baaa} is a subsequence
  of \texttt{banana}.\\
\bottomrule
\end{tabulary}
\caption{Formal language glossary\label{fig:table}}
\end{figure}

The term \emph{language} denotes any set of strings over some fixed
alphabet. The \emph{empty set}, noted~\(\varnothing\),
or~\(\{\varepsilon\}\), the set containing only the empty word are
languages. The set of valid C~programs is an infinite
language. If~\(x\) and~\(y\) are strings, then the
\emph{concatenation} of~\(x\) and~\(y\), written \(xy\) or \(x \cdot
y\), is the string formed by appending \(y\)~to~\(x\). For example, if
\(x = \texttt{dog}\) and \(y = \texttt{house}\), then \(xy =
\texttt{doghouse}\). The empty string is the identity element under
concatenation: \(s \varepsilon = \varepsilon s = s\). If we think of
concatenation as a product, we can define string exponentiation as
follows:
\begin{itemize*}

  \item \(s^0 = \varepsilon\),

  \item \(s^{n} = s^{n-1} s\), if \(n > 0\).

\end{itemize*}
Since \(\varepsilon s = s\) and \(s^1 = s\), we have \(s^2 = ss\) and
\(s^3 = sss\), etc.

We can now revisit in \fig~\vref{fig:table_formal} the definitions we
gave in \fig~\vref{fig:table} using a formal notation, where \(L\)~is
the language under consideration.
\begin{figure}
\centering
\begin{tabular}{c>{$}c<{$}}
\toprule
  \textsc{Term}
& \multicolumn{1}{c}{\textsc{Formal definition}}\\
\midrule
  \(x\)~is a \emph{prefix} of~\(s\)
& \exists y \in L. s = x y\\
  \(x\)~is a \emph{suffix} of~\(s\)
& \exists y \in L.s = y x\\
  \(x\)~is a \emph{substring} of~\(s\)
& \exists u, v \in L. s = u x v\\
  \(x\)~is a \emph{proper prefix} of~\(s\)
& \exists y \in L. y \neq \varepsilon \; \text{and} \; s = x y\\
  \(x\)~is a \emph{proper suffix} of~\(s\)
& \exists y \in L. y \neq \varepsilon \; \text{and} \; s = y x\\
  \(x\)~is a \emph{proper substring} of~\(s\)
& \exists u, v \in L. u v \neq \varepsilon \; \text{and} \; s = u x
v\\
\bottomrule
\end{tabular}
\caption{Formal definitions of \fig~\vref{fig:table}\label{fig:table_formal}}
\end{figure}

\paragraph{Operations on languages}

It is possible to define operations on languages. For lexical
analysis, we are interested mainly in \emph{union},
\emph{concatenation} and \emph{closure}. Consider
\fig~\vref{fig:operations}, where let~\(L\) and~\(M\) be two
languages.
\begin{figure}[b]
\centering
\begin{tabular}{c>{$}c<{$}}
\toprule
  \textsc{Operation}
& \multicolumn{1}{c}{\textsc{Formal definition}}\\
\midrule
  \emph{union} of \(L\) and \(M\)
& L \cup M = \{ s \mid s \in L \; \text{or} \; s \in M\}\\
  \emph{concatenation} of \(L\) and \(M\)
& L M = \{s t \mid s \in L \; \text{and} \; t \in M\}\\
  \emph{Kleene closure} of \(L\)
& L^{*} = \bigcup_{i=0}^{\infty}{L^{i}} \; \text{where} \; L^0
  = \{\varepsilon\}\\
  \emph{positive closure} of \(L\)
& L^{+} = \bigcup_{i=1}^{\infty}{L^{i}}\\
\bottomrule
\end{tabular}
\caption{Operations on formal languages\label{fig:operations}}
\end{figure}
\(L^{*}\) means `zero or more concatenations of~\(L\)', and
\(L^{+}\)~means `at least one concatenation of~\(L\).' Clearly,
\(L^{*} = \{\varepsilon\} \cup L^{+}\). For instance, let \(L =
\{\texttt{A}, \texttt{B}, \ldots, \texttt{Z}, \texttt{a}, \texttt{b},
\ldots, \texttt{z}\}\) and \(D = \{\texttt{0}, \texttt{1}, \ldots,
\texttt{9}\}\). Then
\begin{enumerate*}

  \item \(L\)~is the alphabet consisting of the set of upper and lower
    case letters and \(D\)~is the alphabet of the decimal digits;

  \item since a symbol is a string of length one, the sets
    \(L\)~and~\(D\) are finite languages too.

\end{enumerate*}
These two ways of considering \(L\)~and~\(D\) and the operations on
languages allow us to create new languages from other languages
defined by their alphabet. Here are some examples of new languages
created from~\(L\) and~\(D\): 
\begin{itemize*}

  \item \(L \cup D\) is the language of letters and digits;

  \item \(L D\) is the language whose words consist of a letter followed
  by a digit;

  \item \(L^{4}\) is the language whose words are four\hyp{}letter
  strings;

  \item \(L^{*}\) is the language made up on the alphabet~\(L\),
  \emph{i.e.,} the set of all strings of letters, including the empty
  string~\(\varepsilon\);

  \item \(L(L \cup D)^{*}\) is the language whose words consist of
  letters and digits and beginning with a letter;

  \item \(D^{+}\) is the language whose words are made of one or more
  digits, \emph{i.e.,} the set of all decimal integers.

\end{itemize*}
