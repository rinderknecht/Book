\section{Skipping}
\label{sec:skipping}

Here we envisage how to copy a stack while ignoring a given
item. [\emph{The progression is very slow on purpose.}]

\paragraph{First occurrence}
\label{def:linear_search}
\index{stack!skipping an item!first occurrence}

Let us suppose that \(\fun{sfst}(s,x)\)\index{sfst@\fun{sfst/2}|(}
(\emph{skip the first occurrence}) evaluates in a stack identical
to~\(s\) but without the first occurrence of~\(x\), starting from the
top. In particular, if~\(x\) is absent in~\(s\), then the value of the
call is identical to~\(s\). This is our
\emph{specification}.\index{specification} For instance, we expect the
following evaluations:\label{sfst_ex}
\begin{equation*}
\begin{array}{r@{\;}l@{\;}l}
\fun{sfst}(\el,3) & \twoheadrightarrow & \el;\\
\fun{sfst}(\el,\el) & \twoheadrightarrow & \el;\\
\fun{sfst}([3,\el],[5,2]) & \twoheadrightarrow & [3,\el];\\
\fun{sfst}([\el,[1,2],4,\el,4],4) & \twoheadrightarrow & 
  [\el,[1,2],\el,4];\\
\fun{sfst}([4,[1,2],\el,\el,4],\el)
 & \twoheadrightarrow & [4,[1,2],\el,4].
\end{array}
\end{equation*}

\paragraph{First attempt}

Let us try a direct approach. In particular, at this point, it is
important \emph{not} to seek a definition in tail form. Tail form must
be considered as an optimisation and early optimisation is opening
Pandora's jar. The first idea that may come to mind is to define an
auxiliary function \fun{mem/2} such that the call \(\fun{mem}(s,x)\)
checks whether a given item \(x\)~is in a given stack~\(s\), because
that notion of membership is implicit in the wording of the
specification. But two problems then arise. Firstly, what would be the
result of such a function? Secondly, what would be the additional cost
for using it? For the sake of the argument, let us follow this track
and find out how where it leads. A stack can either be empty or not,
so let us make two rules:
\begin{equation*}
\fun{mem}(\el,x) \rightarrow \fbcode{CCCCC}\,;\quad
\fun{mem}(\cons{y}{s},x) \rightarrow \fbcode{CCCCC}\,.
\end{equation*}
Note that we introduced a variable~\(y\), distinct from
variable~\(x\). \emph{Two different variables may or may not denote
  the same value, but two occurrences of the same variable always
  denote the same value.} Had we written instead
\begin{equation*}
\fun{mem}(\el,x) \rightarrow \fbcode{CCCCC}\,;\quad
\fun{mem}(\cons{\underline{x}}{s},x) \rightarrow \fbcode{CCCCC}\,.
\end{equation*}
one case would be missing, namely when the top of the stack is not the
item sought for, for instance, \(\fun{mem}(3,[4])\) would fail due to
a match failure. Now, what is the first right\hyp{}hand side? The
first pattern matches only if the stack is empty. In particular, this
means that the item is not in the stack, since, by definition, an
empty stack is a stack containing no item. How do we express that?
Since the original problem is silent on the matter, it is said to be
\emph{underspecified}.\index{underspecification} We may think that
zero would be a token of choice to denote the absence of the item in
the stack:
\begin{equation*}
\fun{mem}(\el,x) \rightarrow 0;\quad
\fun{mem}(\cons{y}{s},x) \rightarrow \fbcode{CCCCC}\,.
\end{equation*}
But this would be a mistake because there is no natural and necessary
relationship between the concept of emptiness and the number
zero. Zero is best understood algebraically as the number noted~\(0\)
such that \(0 + n = n + 0 = n\), for any number~\(n\). Then, let us
try the empty stack:
\begin{equation*}
\fun{mem}(\el,x) \rightarrow \el;\quad
\fun{mem}(\cons{y}{s},x) \rightarrow \fbcode{CCCCC}\,.
\end{equation*}
The next step is to find a way to actually compare the value of~\(x\)
to the value of~\(y\). We can use the rule above about variables: two
occurrences of the same variable mean that they hold the same
value. Therefore
\begin{equation*}
\fun{mem}(\el,x) \rightarrow \el;\quad
\fun{mem}(\cons{x}{s},x) \rightarrow \fbcode{CCCCC}\,.
\end{equation*}
was not so bad, after all? Indeed, but we know now that a case is
missing, so let us add it at the end, where \(x \neq y\):
\begin{equation*}
\fun{mem}(\el,x) \rightarrow \el;\;\;
\fun{mem}(\cons{x}{s},x) \rightarrow \fbcode{CCCCC};\;\:
\underline{\fun{mem}(\cons{y}{s},x)} \rightarrow \fbcode{CCCCC}.
\end{equation*}
Now, what is the second right\hyp{}hand side? It is evaluated if the
item we were looking for is present at the top of the stack. How do we
express that? We may think of ending with the item itself, the
rationale being that if the result is the empty stack, then the item
is not in the input stack, otherwise the result is the item itself:
\begin{equation*}
\fun{mem}(\el,x) \rightarrow \el;\quad
\fun{mem}(\cons{x}{s},x) \rightarrow x;\quad
\fun{mem}(\cons{y}{s},x) \rightarrow \fbcode{CCCCC}\,.
\end{equation*}
The last right\hyp{}hand side is easier to guess since it deals with
the case where the top of the stack (\(y\)) is not the item we seek
(\(x\)), so a recursive call which ignores~\(y\) should come to mind:
\begin{equation*}
\fun{mem}(\el,x) \rightarrow \el;\quad
\fun{mem}(\cons{x}{s},x) \rightarrow x;\quad
\fun{mem}(\cons{y}{s},x) \rightarrow \fun{mem}(s,x).
\end{equation*}

Some tests would increase the confidence that this definition is
correct\index{soundness} and complete\index{completeness} with respect
to the specification. Let us label the rules first:
\begin{equation*}
\fun{mem}(\el,x)         \smashedrightarrow{\zeta} \el;\quad
\fun{mem}(\cons{x}{s},x) \smashedrightarrow{\eta}   x;\quad
\fun{mem}(\cons{y}{s},x) \smashedrightarrow{\theta} \fun{mem}(s,x).
\end{equation*}
Then we could try the following cases:
\begin{equation*}
\begin{array}{r@{\;}l@{\;}l@{\;}l@{\;}l}
\fun{mem}(\el,3) & \smashedrightarrow{\zeta} & \el,\\
\fun{mem}([1],3) & \smashedrightarrow{\theta} & \fun{mem}(\el,3)
& \smashedrightarrow{\zeta} & \el,\\
\fun{mem}([1,3,2],3) & \smashedrightarrow{\theta} & \fun{mem}([3,2],3)
& \smashedrightarrow{\eta} & 3.
\end{array}
\end{equation*}
The code seems to work: item~\(x\) is in stack~\(s\) if the result
is~\(x\), otherwise it is~\(\el\). However, this function is not
correct.\index{soundness} The hidden and flawed assumption is `items
can not be stacks,' in spite of the counter\hyp{}examples given at the
beginning for illustrating the expected behaviour of \fun{sfst/2}. In
particular, an item can be the empty stack and this situation leads to
an ambiguity with our definition of \fun{mem/2}:
\begin{equation*}
  \fun{mem}(\el,\el) \smashedrightarrow{\zeta} \el \xleftarrow{\smash{\eta}} \fun{mem}([\el],\el).
\end{equation*}
It is impossible to discriminate the two cases, the first one meaning
absence of the item and the second presence, because they both end
with the empty stack. In fact, we should have distinguished two data
constructors to denote the outcomes `the item was found' and `the item
was not found'. For example,
\begin{equation*}
\begin{array}{r@{\;}l@{\;}l}
\fun{mem}(\el,x)         & \smashedrightarrow{\zeta}  & \fun{false}();\\
\fun{mem}(\cons{x}{s},x) & \smashedrightarrow{\eta}   & \fun{true}();\\
\fun{mem}(\cons{y}{s},x) & \smashedrightarrow{\theta} & \fun{mem}(s,x).
\end{array}
\end{equation*}
But, for now, let us backtrack and ask ourselves again whether using
\fun{mem/2} is really a good idea.

\paragraph{Better approach}

Let us suppose that the input stack contains the item at the
bottom. Using \fun{mem/2} to find it leads to a complete traversal of
the input stack. Then another traversal from the beginning (the top of
the stack) is needed to copy the stack without its last item, so, in
total, two complete traversals are performed.

A better idea consists in \emph{interleaving} these two passes into
one because the problem stems from the fact that \fun{mem/2} forgets
about the items which are not the item of interest, thus, when it is
found or known to be absent, there is no way to build a copy to make
the result. By interleaving, we mean that during the traversal, the
concepts of membership and of copying are combined, instead of being
used sequentially as two function calls. A similar situation was
encountered in the design of a function reversing a stack:
\fun{rev\(_0\)/1}, which calls \fun{cat/2}, is much slower than
\fun{rev/1}, which uses an auxiliary stack.

Here, our algorithm consists in memorising all visited items and, if
the item is not found, the resulting stack is rebuilt from them; if
found, the result is built from them \emph{and} the remaining,
unvisited, items. There are usually two ways to keep visited items:
either in an accumulative parameter, called \emph{accumulator}, or in
the context of recursive calls. At this point, it is important to
recall a cardinal guideline: Do not try first to design a definition
in tail form, but opt instead for a direct approach. In some simple
cases, a direct approach may actually be in tail form, but the point
is methodological: a priori ignore all concerns about tail
forms. Accordingly, let us use the context of a recursive call to
record the visited items. A stack being either empty or not, it is
natural to start with
\begin{equation*}
\begin{array}{r@{\;}l@{\;}l}
\fun{sfst}(\el,x) & \rightarrow & \fbcode{CCCCC}\,;\\
\fun{sfst}(\cons{y}{s},x) & \rightarrow & \fbcode{CCCCC}\,.
\end{array}
\end{equation*}
Then, just as we tried with \fun{mem/2}, we must distinguish the
case when \(x\)~is the same as~\(y\):
\begin{equation*}
\begin{array}{r@{\;}l@{\;}l}
\fun{sfst}(\el,x) & \rightarrow & \fbcode{CCCCC}\,;\\
\underline{\fun{sfst}(\cons{x}{s},x)} & \rightarrow & \fbcode{CCCCC}\,;\\
\fun{sfst}(\cons{y}{s},x) & \rightarrow & \fbcode{CCCCC}\,.
\end{array}
\end{equation*}
This method is a \emph{linear search}\index{linear search}: the items
in the stack are compared one by one to~\(x\), starting from the top,
until the bottom or an item equal to~\(x\) is reached. Since we know
that the last rule deals with the case when \(x \neq y\), we must
memorise~\(y\) and go on comparing~\(x\) with the other items in~\(s\)
(if any). This is where the recursive call with a context, discussed
above, is set as \(\cons{y}{\text{\textvisiblespace}}\):
\begin{equation*}
\begin{array}{r@{\;}l@{\;}l}
\fun{sfst}(\el,x) & \rightarrow & \fbcode{CCCCC}\,;\\
\fun{sfst}(\cons{x}{s},x) & \rightarrow & \fbcode{CCCCC}\,;\\
\fun{sfst}(\cons{y}{s},x) & \rightarrow & \cons{y}{\fun{sfst}(s,x)}.
\end{array}
\end{equation*}
Very importantly, let us remark that the position of~\(y\) in the
result is the same as in the input (the top). The second rule
corresponds to the case where the item we are looking for,
namely~\(x\), is found to be the top of the current stack, which is a
substack of the original input. A stack made of successive items from
the beginning of a given stack is called a \emph{prefix} of the
latter. When a stack is a substack of another, that is, it is made of
successive items including the last, it is called a \emph{suffix}. We
know that the~\(x\) in \(\cons{x}{s}\) is the first occurrence
of~\(x\) in the original stack (the one in the first call), because we
wouldn't be dealing with this case \emph{again}: the specification
states that this first occurrence must be absent from the resulting
stack; since it is now at the top of a suffix, we just need to end
with~\(s\), \emph{which we do not visit}:
\begin{equation*}
\begin{array}{r@{\;}l@{\;}l}
\fun{sfst}(\el,x)         & \rightarrow & \fbcode{CCCCC}\,;\\
\fun{sfst}(\cons{x}{s},x) & \rightarrow & \underline{s}\,;\\
\fun{sfst}(\cons{y}{s},x) & \rightarrow & \cons{y}{\fun{sfst}(s,x)}\,.
\end{array}
\end{equation*}
The first rule handles the case where we traversed the whole original
stack (up to \(\el\)) without finding~\(x\). Thus the result is simply
the empty stack because the empty stack without~\(x\) is the empty
stack:
\begin{equation*}
\fun{sfst}(\el,x)         \rightarrow \el;\quad
\fun{sfst}(\cons{x}{s},x) \rightarrow s;\quad
\fun{sfst}(\cons{y}{s},x) \rightarrow \cons{y}{\fun{sfst}(s,x)}.
\end{equation*}

Let us run some tests now and, in order avoid mistakes, it is handy to
label the rules with some Greek letters:
\begin{equation*}
\fun{sfst}(\el,x)          \xrightarrow{\smash{\theta}} \el;\quad
\fun{sfst}(\cons{x}{s},x)  \xrightarrow{\smash{\iota}}  s;\quad
\fun{sfst}(\cons{y}{s},x)  \xrightarrow{\smash{\kappa}}
\cons{y}{\fun{sfst}(s,x)}.
\end{equation*}
The item is absent in~\(\theta\), the item is found in~\(\iota\) and
the search continues with~\(\kappa\). Note that equality is implicitly
meant in non\hyp{}linear rules like~\(\iota\); in other words, the
cost of such equality test is~\(0\) in our model. Also remark how
important is for rule~\(\iota\) to be written before~\(\kappa\),
otherwise \(\iota\)~would be useless (so\hyp{}called \emph{dead
  code}\index{functional language!dead code}). Here is an
example\index{stack!skipping an item!example} of a successful search:
\begin{equation*}
\fun{sfst}([3,0,1,2],1) \xrightarrow{\smash{\kappa}}
\cons{3}{\fun{sfst}([0,1,2],1)} \xrightarrow{\smash{\kappa}}
\cons{3,0}{\fun{sfst}([1,2],1)} \xrightarrow{\smash{\iota}} [3,0,2].
\end{equation*}
Now an example of an unsuccessful\index{stack!skipping an
  item!example} search:
\begin{equation*}
\fun{sfst}([3,0],4) \xrightarrow{\smash{\kappa}}
\cons{3}{\fun{sfst}([0],4)} \xrightarrow{\smash{\kappa}}
\cons{3,0}{\fun{sfst}(\el,4)} \xrightarrow{\smash{\theta}}
[3,0].
\end{equation*}
More complicated examples, given \vpageref{sfst_ex}, yield
\begin{equation*}
\begin{array}{r@{\;}l@{\;}l}
\fun{sfst}([4,[1,2],\el,\el,4],\el)
& \smashedrightarrow{\kappa} &
  [4|\fun{sfst}([[1,2],\el,\el,4],\el)]\\
& \smashedrightarrow{\kappa} &
  [4|[[1,2]|\fun{sfst}([\el,\el,4],\el)]]\\
& = & [4,[1,2]|\fun{sfst}([\el,\el,4],\el)]\\
& \smashedrightarrow{\iota} & [4,[1,2]|[\el,4]]\\
& = & [4,[1,2],\el,4].
\end{array}
\end{equation*}
\begin{equation*}
\begin{array}{r@{\;}l@{\;}l}
\fun{sfst}([3,\el],[5,2])
& \smashedrightarrow{\kappa} & [3|\fun{sfst}([\el],[5,2])]\\
& \smashedrightarrow{\kappa} & [3|[\el|\fun{sfst}(\el,[5,2])]]\\
& = & [3,\el|\fun{sfst}(\el,[5,2])]\\
& \smashedrightarrow{\theta} & [3,\el|\el]\\
& = & [3,\el].
\end{array}
\end{equation*}
Once we are convinced that our definition is correct\index{soundness}
and complete with respect to the specification, there is a little
extra worth testing: we can check \emph{what happens for inputs which
  are not expected by the specification.} Our specification says at
one point that the second argument of \fun{sfst/2} is a stack. What
happens if we supply an integer instead? For example, we have
\(\fun{sfst}(3,\el) \nrightarrow\). We have a match failure, that is,
the rewrites are stuck, so that our definition is not
\emph{robust},\index{robustness} in other words, it fails abruptly on
unspecified inputs.

When programming in the small, as we do here, robustness is usually
not a concern because we want to focus on learning a language by
expressing simple algorithms, but when developing large applications,
we must take care of making the code robust by catching and signalling
errors. Notice that a program can be complete\index{completeness} but
not robust,\index{robustness} because completeness is relative to
what is specified behaviour (all valid inputs must be accepted and not
lead to an error), whereas robustness is relative to what is left
unspecified.

These considerations are germane to discussing the merits and
weaknesses of scripting languages, whose semantics try hard to ignore
errors by defaulting on special values (like the empty string) to keep
running. In the setting of our abstract functional language, we can
use a data constructor, that is, a function without evaluation rules,
like \(\fun{error}()\), to signal an error or notify some piece of
information about the arguments. For instance, here is the definition
of a function which distinguishes between stacks and non\hyp{}stack
arguments:
\begin{equation*}
\begin{array}{r@{\;}l@{\;}l}
\fun{is\_a\_stack}(\el) & \rightarrow & \fun{yes}();\\
\fun{is\_a\_stack}(\cons{x}{s}) & \rightarrow & \fun{yes}();\\
\fun{is\_a\_stack}(s) & \rightarrow & \fun{no}().
\end{array}
\end{equation*}
Data constructors come handy in signalling errors because they are
like unique identifiers, therefore they cannot be confused with any
other kind of data the function computes and so can be detected easily
by the caller. Consider this robust\index{robustness} version of
\fun{sfst/2} which discriminates errors:
\begin{equation*}
\begin{array}{r@{\;}l@{\;}l}
\fun{sfst}(\el, x) & \rightarrow & \el;\\
\fun{sfst}(\cons{x}{s},x) & \rightarrow & s;\\
\fun{sfst}(\cons{y}{s},x) & \rightarrow & \cons{y}{\fun{sfst}(s,x)};\\
\fun{sfst}(s,x) & \rightarrow & \underline{\fun{error}()}.
\end{array}
\end{equation*}
Then a function calling \fun{sfst/2} can make the difference between a
normal rewrite and an error by using a data constructor in a pattern:
\begin{equation*}
\begin{array}{r@{\;}l@{\;}l}
\fun{caller}(s,x) & \rightarrow & \fun{check}(\fun{sfst}(s,x)).\\
\fun{check}(\fun{error}()) & \rightarrow & \fbcode{CCCCC}\,;\\
\fun{check}(r) & \rightarrow & \fbcode{CCCCC}\,.
\end{array}
\end{equation*}
%%% END

\paragraph{Cost}

In general, the cost
\(\C{\fun{sfst}}{n}\)\index{sfst@$\C{\fun{sfst}}{n}$|(} of
\(\fun{sfst}(s,x)\), where \(n\)~is the length of~\(s\), depends
on~\(x\) being present in~\(s\) or not. In the latter case, the trace
is \(\kappa^n\theta\), so \(\C{\fun{sfst}}{n} = \len{\kappa^n\theta} =
n + 1\). If the former, the cost depends on the position of~\(x\)
in~\(s\). Let us set that the top of~\(s\) is at position~\(0\) and
\(x\)~occurs at position~\(j\). We then have \(\C{\fun{sfst}}{n,j} =
\len{\kappa^j\iota} = j + 1\)\index{sfst@$\C{\fun{sfst}}{n}$|)}. If we
decide that position~\(n\) (or greater) means absence, we can actually
retain the last formula for both cases.

The minimum cost \(\B{\fun{sfst}}{n}\)\index{sfst@$\B{\fun{sfst}}{n}$}
is then the minimum\index{stack!skipping an item!minimum cost} value
of \(\C{\fun{sfst}}{n,j}\), for \(j\)~ranging from \(0\)~to~\(n\),
therefore \(\B{\fun{sfst}}{n} = \C{\fun{sfst}}{n,0} = 1\), that is,
when the item occurs at the top, and, dually, the maximum
cost\index{stack!skipping an item!maximum cost}
is\index{sfst@$\W{\fun{sfst}}{n}$} \(\W{\fun{sfst}}{n} =
\C{\fun{sfst}}{n,n} = n + 1\), that is, when the item is absent. The
average cost
\(\M{\fun{sfst}}{n}\)\index{sfst@$\M{\fun{sfst}}{n}$}\index{stack!skipping
  an item!average cost} of a successful search assumes that \(j\)~can
take all the positions in the stack:
\begin{equation*}
  \M{\fun{sfst}}{n} = \frac{1}{n}\sum_{j=0}^{n-1}\C{\fun{sfst}}{n,j} =
  \frac{1}{n}\sum_{j=0}^{n-1}(j+1) = \frac{1}{n}\sum_{j=0}^{n}j
  = \frac{n+1}{2} \sim \frac{n}{2},
\end{equation*}
by equation~\eqref{eq:sum_k} \vpageref{eq:sum_k}.

Notice that rule~\(\kappa\) implies the creation of a
(\texttt{|})\hyp{}node, which we call
\emph{cons\hyp{}node}\index{cons-node}, as shown in
\fig~\vref{fig:sfst_dag}.
\begin{figure}
\centering
\includegraphics[bb=71 664 345 721]{sfst_dag}
\caption{Directed acyclic graphs for \fun{sfst/2}
\label{fig:sfst_dag}}
\end{figure}
Hence, whilst the contents of the new stack is shared with the
original stack, \(j\)~nodes are newly allocated if~\(x\) occurs at
position~\(j\) in~\(s\). The worst case happens when \(x\)~is absent
so the memory\index{stack!skipping an item!memory} needed amounts to
\(n\)~nodes, all of which being useless because in this case
\(\fun{sfst}(s,x) \equiv s\). If we want to avoid this situation,
another definition of \fun{sfst/2} has to be devised, one that
discards all constructed nodes and allows us to reference the input
when~\(x\) is missing.

The crux of the matter is embodied in the construct
\(\cons{y}{\texttt{\textvisiblespace}}\) of rule~\(\kappa\), called
the \emph{context}\index{functional language!context of a call} of the
call \(\fun{sfst}(s,x)\)\index{sfst@\fun{sfst/2}|)}, which we want
if~\(x\) is present, but not otherwise. To resolve this conflicting
requirement, we opt for removing the context and store the information
it contains (\(y\)) into an accumulator\index{functional
  language!accumulator}, in a new rule~\(\xi\) derived
from~\(\kappa\). We use the accumulator in a new rule~\(\nu\) derived
from~\(\iota\). The new \fun{sfst/2} is called
\fun{sfst\(_0\)/2}\index{sfst0@\fun{sfst\(_0\)/2}} and is shown in
\fig~\ref{fig:sfst0}.\index{sfst@\fun{sfst/4}}
\begin{figure}[b]
\begin{equation*}
\boxed{%
\begin{array}{@{}r@{\;}l@{\;}lr@{\;}l@{\;}l@{}}
\fun{rcat}(\el,t)              & \xrightarrow{\smash{\zeta}} & t; &
\fun{sfst}(\el,x,t,u)          & \xrightarrow{\smash{\mu}} & u;\\
\fun{rcat}(\cons{x}{s},t)      & \xrightarrow{\smash{\eta}} 
                               & \fun{rcat}(s,\cons{x}{t}). & 
\fun{sfst}(\cons{x}{s},x,t,u)  & \xrightarrow{\smash{\nu}}
                               & \fun{rcat}(t,s);\\
\fun{sfst}_0(s,x)              & \xrightarrow{\smash{\lambda}} 
                               & \fun{sfst}(s,x,\el,s). &
\fun{sfst}(\cons{y}{s},x,t,u)  & \xrightarrow{\smash{\xi}}
                               & \fun{sfst}(s,x,\cons{y}{t},u).
\end{array}}
\end{equation*}
\caption{Skipping the first occurrence with maximum sharing
\label{fig:sfst0}}
\end{figure}
Of course, whilst in~\(\iota\) we just referenced~\(s\), the
construction corresponding to the now missing context of~\(\kappa\)
must be performed by~\(\nu\). Also, we must add another argument which
refers to the original stack, so we can use it in a new rule~\(\mu\),
the pendant of~\(\theta\). Note the shapes of the right\hyp{}hand
sides: they are either a value~(\(\zeta\) and~\(\mu\)) or a function
call whose arguments contain values. In other words, no function call
has a context. This syntactic property of a definition is named
\emph{tail form}\index{functional language!tail form}. Intuitively,
the practical consequence of such a form is that terminating calls
unfold until a value is reached and nothing else is left to be done:
\emph{the value of the last call is the value of the first call}. This
kind of definition enables the sharing in rule~\clause{\mu}, where
\(u\) (the reference to the original stack) becomes the value, instead
of \(\fun{rev}(t)\).

Implementations of functional language often use this property to
optimise the evaluation, as we shall see in the last part of this
book. The downside of \fun{sfst\(_0\)/2} with respect to \fun{sfst/2}
is the additional cost incurred by having to reverse~\(t\) in
rule~\clause{\nu}, that is, the call \(\fun{rcat}(t,s)\). More
precisely, there are two complementary cases: either \(x\)~is missing
in~\(s\) or \(x\)~occurs in~\(s\). Let us assume that \(s\)~contains
\(n\)~items and \(x\)~is absent in~\(s\). The evaluation trace
\index{functional language!evaluation!trace} of the call
\(\fun{sfst}_0(s,x)\) is \(\lambda\xi^n\mu\), so \(\C{\fun{sfst}_0}{n}
= \len{\lambda\xi^n\mu} = \len{\lambda} + n \len{\xi} + \len{\mu} = n
+ 2\). Let us now assume that \(x\)~occurs at position \(k\) in~\(s\),
with the first item having position~\(0\). The evaluation trace is
then \(\lambda \xi^{k}\nu\eta^k\zeta\), hence \(\C{\fun{sfst}_0}{n,k}
= \len{\lambda \xi^{k}\nu\eta^k\zeta} = 2k + 3\). Clearly now,
\begin{gather*}
\B{\fun{sfst}_0}{0} = 2,\quad
\B{\fun{sfst}_0}{n} = \min_{0 \leqslant k < n}\{\C{\fun{sfst}_0}{n}, \C{\fun{sfst}_0}{n,k}\}
                    = \min_{0 \leqslant k < n}\{n+2,2k+3\} = 3,\\
\W{\fun{sfst}_0}{n} = \max_{0 \leqslant k < n}\{n+2,2k+3\} = 2n+1,
\end{gather*}
where the minimum cost occurs when the item is the top of the stack;
the maximum cost happens when the sought item is last in the stack (at
the bottom).

Since calling \fun{rcat/2}\index{rcat@\fun{rcat/2}} to reverse the
visited items is the source of the extra cost, we might try to
maintain the order of these items in \fig~\vref{fig:sfst1}\index{sfst1@\fun{sfst\(_1\)/1}}
\begin{figure}[b]
\begin{equation*}
\boxed{%
\begin{array}{r@{\;}l@{\;}l}
\fun{sfst}_1(s,x)               & \rightarrow
                                & \fun{sfst}_2(s,x,\el,s).\\
\fun{sfst}_2(\el,x,t,u)         & \rightarrow & u;\\
\fun{sfst}_2(\cons{x}{s},x,t,u) & \rightarrow & \fun{cat}(t,s);\\
\fun{sfst}_2(\cons{y}{s},x,t,u) & \rightarrow
                                & \fun{sfst}_2(s,x,\fun{cat}(t,[y]),u).
\end{array}}
\end{equation*}
\caption{Skipping the first occurrence (bad design)
\label{fig:sfst1}}
\end{figure}
but using stack catenation instead of pushing. The problem is that the
last rule of \fun{sfst\(_2\)/4}\index{sfst@\fun{sfst\(_2\)/4}}
\index{cat@\fun{cat/2}} yields
\(\fun{cat}(\dots\fun{cat}(\fun{cat}(\el,[x_1]),[x_2])\dots)\), whose
cost we know to be quadratic \index{cost!quadratic $\sim$} as in the
rewrite~\eqref{eq:rev0}
\vpageref{eq:rev0}\index{rev0@\fun{rev\(_0\)/1}}, from which we can
quickly conclude that \(\W{\fun{sfst}_1}{n} \sim \tfrac{1}{2}n^2\).

\paragraph{Last occurrence}
\index{stack!skipping an item!last occurrence}

Let us suppose that \(\fun{slst}(s,x)\)\index{slst@\fun{slst/2}}
(\emph{skip the last occurrence}) evaluates in a stack identical
to~\(s\) but without the last occurrence of~\(x\). In particular,
if~\(x\) is absent in~\(s\), then the value of the call is identical
to~\(s\). The first design that may come to mind is to see this
problem as the dual problem of ignoring the first occurrence:
\begin{equation}
\fun{slst}_0(s,x) \xrightarrow{\smash{\pi}}
                  \fun{rev}(\fun{sfst}(\fun{rev}(s),x)).
\label{eq:slst0}
\end{equation}
If~\(x\) is missing in~\(s\), we
have\index{slst0@$\C{\fun{slst}_0}{n}$}\index{rev@$\C{\fun{rev}}{n}$}
\(\C{\fun{slst}_0}{n} = 1 + \C{\fun{rev}}{n} + \W{\fun{sfst}}{n} +
\C{\fun{rev}}{n} = 3n+6\). If~\(x\) occurs in~\(s\) at position~\(k\),
\(\C{\fun{slst}_0}{n,k} = 1 + \C{\fun{rev}}{n} +
\C{\fun{sfst}}{n,n-k-1} + \C{\fun{rev}}{n-1} = 3n - k +
4\). Therefore, we can derive the minimum and
maximum\index{stack!skipping an item!minimum
  cost}\index{stack!skipping an item!maximum cost} costs:
\begin{equation*}
\abovedisplayskip=0pt
\belowdisplayskip=0pt
\B{\fun{slst}_0}{n} = \min_{k < n}\{3n+6,3n-k+4\} = 2n + 5,
\end{equation*}
when\index{slst@$\B{\fun{slst}_0}{n}$} \(x\)~is last in~\(s\), and
\begin{equation*}
\abovedisplayskip=0pt
\belowdisplayskip=0pt
\W{\fun{slst}_0}{n} = \max_{k < n}\{3n+6,3n-k+4\} = 3n+6,
\end{equation*}
when\index{slst@$\W{\fun{slst}_0}{n}$} \(x\)~is missing in~\(s\).  The
mean cost\index{stack!skipping an item!average
  cost}\index{slst@$\M{\fun{slst}_0}{n}$} when \(x\)~is present is
\begin{equation*}
\abovedisplayskip=0pt
\belowdisplayskip=0pt
\M{\fun{slst}_0}{n} = \frac{1}{n}\sum_{k=0}^{n-1}\C{\fun{slst}_0}{n,k}
                    = \frac{1}{n}\sum_{k=0}^{n-1}(3n - k + 4)
                    = \frac{5n+9}{2} \sim \frac{5}{2}n.
\end{equation*}
When \(x\)~is present, the worst case is when it is the top of the
stack: \(\W{\fun{slst}_0}{n} = \max_{k<n}\{3n - k + 4\} = 3 n+ 4
\leqslant 3n + 6\).

In any case, the maximum cost is asymptotically equivalent to~\(3n\),
that is, three complete traversals of~\(s\) are performed, whilst the
absence of~\(x\) could be detected with one. Dually, the minimum cost
is asymptotically equivalent to~\(2n\), accounting for two full
traversals, whilst \(x\)~being the last item could be assessed with
one. All this suggests that a better design is worth thinking about.

Consider \fig~\ref{fig:slst},
\begin{figure}[b]
\begin{equation*}
\boxed{%
\begin{array}{@{}r@{\;}l@{\;}lr@{\;}l@{\;}l@{}}
  \fun{slst}(\el,x)           & \xrightarrow{\smash{\rho}} & \el;
& \fun{slst}(\el,x,t)         & \xrightarrow{\smash{\upsilon}} & t;\\
  \fun{slst}(\cons{x}{s},x)   & \xrightarrow{\smash{\sigma}} & 
                                           \fun{slst}(s,x,s);
& \fun{slst}(\cons{x}{s},x,t) & \xrightarrow{\smash{\phi}} &
                                           \cons{x}{\fun{slst}(t,x)};\\
  \fun{slst}(\cons{y}{s},x)   & \xrightarrow{\smash{\tau}} &
                                           \cons{y}{\fun{slst}(s,x)}.
& \fun{slst}(\cons{y}{s},x,t) & \xrightarrow{\smash{\chi}} &
                                           \fun{slst}(s,x,t).
\end{array}}
\end{equation*}
\caption{Skipping the last occurrence with \fun{slst/2}
\label{fig:slst}\index{stack!skipping an item!last occurrence}}
\end{figure}
where, with a linear search\index{linear search} (rules
\(\rho\)~and~\(\tau\)), we find the first occurrence of~\(x\)
(rule~\(\sigma\)), but, in order to check whether it is also the last,
another linear search has to be run~(\(\chi\)). If it is successful
(\(\phi\)), we retain the occurrence find earlier (\(x\)) and resume
another search; if it is unsuccessful~(\(\upsilon\)), the~\(x\) found
earlier was indeed the last occurrence.  Notice how we have two
mutually recursive functions, \fun{slst/2}\index{slst@\fun{slst/2}}
and~\fun{slst/3}\index{slst@\fun{slst/3}}. The definition of the
latter features a third parameter, \(t\), which is a copy of the
stack~\(s\) when an occurrence of~\(x\) was found
by~\fun{slst/2}\index{slst@\fun{slst/2}} (\(\sigma\)). This copy is
used to resume~(\(\phi\)) the search from where the previous
occurrence was found. This is necessary as \(y\) in rule~\(\chi\) must
be discarded because we do not know at that point whether the
previous~\(x\) was the last. Consider \fig~\vref{fig:slst_27071}.
\begin{figure}
\begin{equation*}
\boxed{%
\begin{array}{r@{\;}l@{\;}l}
\fun{slst}([2,7,0,7,1],7)
& \xrightarrow{\smash{\tau}}
& \cons{2}{\fun{slst}([7,0,7,1],7)}\\
& \xrightarrow{\smash{\sigma}}
& \cons{2}{\fun{slst}([0,7,1],7,[0,7,1])}\\
& \xrightarrow{\smash{\chi}} 
& \cons{2}{\fun{slst}([7,1],7,[0,7,1])}\\
& \xrightarrow{\smash{\phi}}
& \cons{2,7}{\fun{slst}([0,7,1],7)}\\
& \xrightarrow{\smash{\tau}}
& \cons{2,7,0}{\fun{slst}([7,1],7)}\\
& \xrightarrow{\smash{\sigma}}
& \cons{2,7,0}{\fun{slst}([1],7,[1])}\\
& \xrightarrow{\smash{\chi}} 
& \cons{2,7,0}{\fun{slst}(\el,7,[1])}\\
& \xrightarrow{\smash{\upsilon}}
& \cons{2,7,0}{[1]} = [2,7,0,1].
\end{array}}
\end{equation*}
\caption{\(\fun{slst}([2,7,0,7,1],7) \twoheadrightarrow
  [2,7,0,1]\)\label{fig:slst_27071}\index{stack!skipping an item!example}}
\end{figure}
If the item is missing, the linear search fails as usual with a cost
of \(\len{\tau^n\rho} = n + 1\). Otherwise, let us name \( 0 \leqslant
x_1 < x_2 < \dots < x_p < n\) the positions of the \(p\)~occurrences
of~\(x\) in~\(s\). The evaluation trace\index{functional
  language!evaluation!trace} is
\begin{equation*}
\abovedisplayskip=0pt
\belowdisplayskip=0pt
\tau^{x_1} 
\cdot
\prod_{k=2}^{p}(\sigma\chi^{x_k-x_{k-1}-1})(\phi\tau^{x_k-x_{k-1}-1})
\cdot
(\sigma\chi^{n-x_p-1}\upsilon),
\end{equation*}
whose length is \(x_1 + 2\sum_{k=2}^{p}(x_k-x_{k-1}) + (n - x_p + 1) =
n + x_p - x_1 + 1\). In other words, if the position of the first
occurrence is noted~\(f\) and the position of the last is~\(l\), we
find that\index{slst@$\C{\fun{slst}}{n}$}
\begin{equation*}
\C{\fun{slst}}{n,f,l} = n + l - f + 1.
\end{equation*}
We deduce that the minimum cost\index{stack!skipping an item!minimum
  cost} happens when \(l-f+1 = p\), that is, when all the occurrences
are consecutive, so \(\B{\fun{slst}}{n,p} = n +
p\)\index{slst@$\B{\fun{slst}}{n}$}. The maximum
cost\index{stack!skipping an item!maximum cost} occurs when~\({f=0}\)
and~\({l=n-1}\), that is, when there is at least two occurrences
of~\(x\), one at the top and one at the bottom: \(\W{\fun{slst}}{n} =
2n\)\index{slst@$\W{\fun{slst}}{n}$}. We can check that when the stack
is entirely made of~\(x\), minimum and maximum costs concur
in~\(2n\). The average cost\index{stack!skipping an item!average cost}
\index{slst@$\M{\fun{slst}}{n}$} when \(x\)~is present requires
determining the cost for every possible pair \((f,l)\), with \(0
\leqslant f \leqslant l < n\):
\begin{align*}
\M{\fun{slst}}{n} 
  &= \frac{2}{n(n+1)} \sum_{f=0}^{n-1}\sum_{l=f}^{n-1}{\C{\fun{slst}}{n,f,l}}
   = \frac{2}{n(n+1)} \sum_{f=0}^{n-1}\sum_{l=f}^{n-1}{(n+l-f+1)}\\
  &= \frac{2}{n(n+1)} \sum_{f=0}^{n-1}\bigg(\!\!(n-f+1)(n-f)
     + \!\!\sum_{l=0}^{n-f-1}\!\!(l+f)\!\!\bigg)\\
  &= \frac{1}{n(n+1)}\sum_{f=0}^{n-1}(3n + 1 - f)(n-f)\\
  &= \frac{n(3n+1)}{n+1} - \frac{4n+1}{n(n+1)}\sum_{f=0}^{n-1}f
     + \frac{1}{n(n+1)}\sum_{f=0}^{n-1}f^2
  = \frac{4}{3}{n} + \frac{2}{3} \sim \frac{4}{3}n,
\end{align*}
where \(\sum_{f=0}^{n-1}{f} = n(n-1)/2\) is equation~\eqref{eq:sum_k},
\vpageref{eq:sum_k}, and the sum of the successive squares is obtained
as follows. We use the \emph{telescoping} or \emph{difference} method
on the series \((k^3)_{k>0}\). We start with the equality \((k+1)^3 =
k^3 + 3k^2 + 3k + 1\), hence \((k+1)^3 - k^3 = 3k^2 + 3k + 1\). Then
we can sum these differences, whose terms cancel out, leaving the
first and the last:
\begin{align}
  && (1+1)^3 - \boxed{1^3} &= 3 \cdot 1^2 + 3 \cdot 1 + 1\notag\\
+ && (2+1)^3 - 2^3         &= 3 \cdot 2^2 + 3 \cdot 2 + 1\notag\\
+ &&&\;\;\vdots\notag\\
+ && \boxed{(n+1)^3} - n^3 &= 3n^2 + 3n + 1\notag
\intertext{\rule{\linewidth}{0.4pt}}
\Rightarrow 
  &&\boxed{(n+1)^3} - \boxed{1^3}
  &= 3 \sum_{k=1}^{n}{k^2} + 3 \sum_{k=1}^{n}{k} + n\notag\\
  && n^3 + 3n^2 + 3n
  &= 3 \sum_{k=1}^{n}{k^2} + 3 \cdot \frac{n(n+1)}{2} + n\notag\\
\Leftrightarrow && \sum_{k=1}^{n}{k^2} &= \frac{n(n+1)(2n+1)}{6}.
\label{eq:sum_of_squares}
\end{align}

\paragraph{Exercises}
\begin{enumerate*}

  \item Prove that \(\fun{sfst/2} = \fun{sfst\(_0\)/2}\).
    \index{sfst@\fun{sfst/2}}\index{sfst0@\fun{sfst\(_0\)/2}}

  \item Show that \(\B{\fun{sfst}_0}{n} = 3\), \(\W{\fun{sfst}_0}{n} =
    2n+1\) and \(\M{\fun{sfst}_0}{n} = n+2\) (successful
    search). 

  \item Prove \fun{slst/2} = \fun{slst\(_0\)/2}.

  \item Show that, in a worst case to be identified,
    \(\fun{slst}_0(s,x)\) creates \(3n\)~useless nodes if~\(s\)
    contains \(n\)~items. Compare the memory usage of
    \fun{slst\(_0\)/2}\index{slst@\fun{slst\(_0\)/2}} with that of
    \fun{slst/2}\index{slst@\fun{slst/2}}.

\end{enumerate*}
