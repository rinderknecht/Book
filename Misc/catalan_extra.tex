\begin{wrapfigure}[9]{r}[0pt]{0pt}
% [9] vertical lines
% {r} mandatory right placement
% [0pt] of margin overhang
\centering
\includegraphics[bb=71 646 150 721]{ctree2}
\caption{Catalan tree \label{fig:ctree2}}
\end{wrapfigure}
With these preliminaries in mind, let us consider an example with the
Catalan tree of size \(n=16\) in \fig~\vref{fig:ctree2}. The node
\((\circ)\) is a node at level \(l=2\) and of degree \(d=3\). We have
seen that we can construct a Dyck path in bijection with that tree by
performing a preorder traversal which yields an upward step, or rise,
when going down an edge, and a rightward step, or fall, when going up
an edge. We are interested here in distinguishing the three edges
connecting our node \((\circ)\) to its children, on the way up.  The
equivalent Dyck path\index{Dyck path} is shown in
\fig~\vref{fig:Catalan_path_length},
\begin{figure}[t]
\centering
\includegraphics[bb=28 452 302 721]{path_length}
\caption{Dyck path equivalent to \fig~\vref{fig:ctree2} \label{fig:Catalan_path_length}}
\end{figure}
from \((0,0)\) to \(C(n-1,n-1)\). Recall that if~\(h\) is the height
of the Catalan tree, the corresponding Dyck path has height~\(h-1\),
therefore, the three oriented edges in the tree are mapped to the
three oriented falls ending at level~\(l\) in the Dyck
path. Precisely, the end points are \((i_1,i_1+l)\), \((i_2,i_2+l)\)
and \(B(i_d,i_d+l)\), where \(d=3\), on the dotted diagonal
\((A,B)\). The subtree rooted at \((\circ)\) in \fig~\vref{fig:ctree2}
corresponds here to the lattice path from~\(A\) to~\(B\), which stays
above the diagonal \((A,B)\), plus the preceding and following steps,
which are right below it. Therefore, we divide the Dyck path in three
successive paths: from the origin \((0,0)\) to~\(A\), from~\(A\)
to~\(B\), and from~\(B\) to~\(C\). To map the node~\((\circ)\) to a
unique path, we operate on these three paths successively as follows:
\begin{enumerate*}

  \item the initial path is left unchanged;

  \item the oriented falls are removed;

  \item the remaining path is reversed.

\end{enumerate*}
Note that there are actually \(d+1\) oriented falls, because we
include the one right after~\(B\). The resulting lattice path goes
from \((0,0)\) to~\(A\), then to~\(B'\), and ends
at~\(C'(n-l-d-1,n+l-2)\). (The new steps that differ from the original
Dyck path are drawn with a dashed line.) The straight line distance
from~\(B\) to~\(B'\) is~\(d\), because we deleted the~\(d\) falls
corresponding to the three oriented edges in the tree. The mapping is
injective because the resulting path is unique. It is also
surjective. First, let us note that, given~\(n\) and a lattice path
ending at \((a,b)\), then \(l=b-n+2\) and \(d=2n-a-b-3\). Second, we
remark that the reversed path stays above, or touches, the diagonal
passing at~\(B'\), drawn with dots, whose equation is
\(y=x+l+d\). Therefore, when we delete a fall \((i_k-1,i_k+l)
\rightarrow (i_k,i_k+l)\), for \(1 \leqslant k \leqslant d\), we raise
the rest of the path by one, which will remain above the diagonal
\(y=x+l+k\). In other words, to recover the Dyck path, we start from
\((0,0)\) and look for the first point at level~\(l+1\) after which
the path remains above the diagonal at that point: this is where we
restore a fall, and we resume the process \(d-1\) times.

% The idea is to find a bijection between these specific nodes and some
% combinatorial objects which are easy to count. Because we already know
% from section~\vref{sec:Catalan_enumeration} that Catalan trees of
% size~\(n\) are in bijection with Dyck paths \index{Dyck path} of
% length \(2n-2\), we want to find lattice paths in bijection with the
% nodes of interest: by counting these paths we will count the nodes.
